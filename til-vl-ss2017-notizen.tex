\documentclass[a4paper,10pt]{article}

\usepackage[utf8]{inputenc}
\usepackage[ngerman]{babel}
\usepackage[top=2.5cm,bottom=2.5cm,left=2.5cm,right=2.5cm]{geometry}
\usepackage[T1]{fontenc}
\usepackage{graphicx}
\usepackage[table]{xcolor}
\usepackage{fancyhdr}
\usepackage{tgpagella}
\usepackage{marginnote}
\usepackage{nameref}
\usepackage{amsmath,amssymb,amsfonts}
\usepackage{enumitem}
\usepackage{sectsty}
\usepackage{wrapfig}
\usepackage{listings}

\usepackage[
    pdfauthor={Dominik Pataky},
    pdftitle={Theoretische Informatik und Logik},
    pdfsubject={Notizen und Begriffsklärungen aus dem Sommersemester 2017, TU Dresden},
    colorlinks=true,linkcolor=black,urlcolor=link]{hyperref}

\definecolor{light-gray}{gray}{0.9}
\definecolor{light-red}{RGB}{255,105,105}
\definecolor{vl}{RGB}{84,200,70}
\definecolor{link}{RGB}{84,100,220}
\definecolor{sectionblue}{RGB}{0,64,114}
\definecolor{subsectionblue}{RGB}{0,94,167}
\definecolor{subsubsectionblue}{RGB}{0,110,195}

% set enumeration style
\setlist[enumerate, 1]{label=\textbf{\alph*)}, leftmargin=2em}
\setlist{itemsep=0em}

\newcommand{\vl}[1]{\colorbox{vl}{\textcolor{white}{\small\textbf{#1}}}}
\newcommand{\f}[1]{\textbf{#1}}
\newcommand{\blank}{\text{\textvisiblespace}}
\newcommand{\hili}[1]{\colorbox{light-gray}{\textcolor{black}{#1}}}
\newcommand{\verweis}[1]{\textit{\ref{#1} \nameref{#1}}}
\newcommand{\prob}[1]{\textbf{#1}}
\newcommand{\prspec}[1]{$~\prob{P}_{#1}~$}
\newcommand{\phalt}{$~\prob{P}_{Halt}~$}
\newcommand{\N}{\mathbb{N}}
\newcommand{\Q}{\mathbb{Q}}
\newcommand{\R}{\mathbb{R}}
\newcommand{\POT}{\mathcal{P}}
\newcommand{\LOES}{\f{Lösung:~}}
\newcommand{\TMM}[1]{\mathcal{M}_{#1}}
\newcommand{\LANG}{\mathcal{L}}



\begin{document}
    % Header and footer styles
    \lhead{\slshape Theoretische Informatik und Logik SS2017}
    \rhead{\slshape\nouppercase{\rightmark}}
    \cfoot{\thepage}
    \lfoot{\footnotesize Dozent: Prof. Krötzsch, Autor: Pataky}
    \rfoot{}

    \begin{titlepage}
        \centering

        {\scshape\Large Notizen zur Vorlesung \par}
        {\scshape\LARGE Theoretische Informatik und Logik \par}
        \vspace{1cm}

        \begin{abstract}
            Notizen zur Vorlesung \url{https://iccl.inf.tu-dresden.de/web/Theoretische_Informatik_und_Logik_(SS2017)}. \par
            Ich versuche möglichst ohne formelle Symbole und Definitionen zu arbeiten, daher verweisen die Markierungen jeweils auf die Vorlesungsnummer in \vl{FS} bzw. \vl{TIL}.
            Obwohl der Schwerpunkt auf TheoLog liegt, habe ich ein paar Definitionen aus Formale Systeme mit einbezogen, da TheoLog diese weiterverwendet. \\
            Einige Formulierungen habe ich aus den hervorragenden Folien von Prof. Krötzsch geliehen. Quellen dieser Folien sind auf Github zu finden unter \url{https://github.com/mkroetzsch} und sind unter der Lizenz CC BY 3.0 DE verwendbar. Für diese gilt: „(C) Markus Krötzsch, \url{https://iccl.inf.tu-dresden.de/web/TheoLog2017}, CC BY 3.0 DE“.
        \end{abstract}

        \tableofcontents

        \vfill
        \begin{tabular}{p{3cm} p{10cm}}
            Autor & Dominik Pataky \\
            Dozent & Prof. Markus Krötzsch \\
            Ort & Fakultät Informatik, TU Dresden \\
            Zeit & Sommersemester 2017 \\
            Letztes Update & \today \\
            Lizenz & CC BY-SA 4.0
        \end{tabular}

    \end{titlepage}

    %\section{Motivation}
    % Hier Motivationstext über theoretische Informatik einfügen
    % Hilbert wollte Anfang des 20. Jahrhunderts die Mathematik auf eine formelle Basis stellen.

    %\item[Logik] Eine kurze Motivationsübersicht zur Einleitung.
    %            \begin{itemize}
    %                \item „Wer rechnende Systeme verstehen und konstruieren will, der benötigt passende mathematische Modelle. Dieser Weg führt oftmals zur Logik.“ Ziel: Verifikation
    %                \item Logik ist die Wissenschaft vom folgerichtigen Denken. Ziel: Künstliche Intelligenz
    %                \item Logisches Schließen ist Problemlösen
    %                \item Logiken sind kodierte Beschreibungssprachen. Ziel: Wissensrepräsentation und Datenbanken
    %                \item „Logik“ ist ein allgemeiner Oberbegriff für viele mathematische und technische Formalismen
    %            \end{itemize}

    \setlength\parindent{0cm} % indentation of paragraphs
    \sectionfont{\color{sectionblue}\Large}
    \subsectionfont{\color{subsectionblue}\large}
    \subsubsectionfont{\color{subsubsectionblue}\normalsize}

    \section{Formale Systeme}
\subsection{Sprachen und Automaten}
\label{subsec:fs-sprachen-automaten}

\begin{description}
    \item[(formale) Sprache] Menge von Wörtern/Symbolen/Tokens, z.B. Programmiercode oder natürliche Sprache.
        Zusätzliche Begriffe: Konkatenation, Präfix/Suffix/Infix, leeres Wort \vl{FS 1}
        \begin{description}
            \item[Symbol] Token der Sprache, z.B. if/else, +/-, True/False, ''Hello World''-String
            \item[Alphabet] nichtleere, endliche Menge von Symbolen
            \item[Wort] endliche Sequenz von Symbolen
            \item[Grammatik] formelle Spezifikation einer Sprache. Aus einer Grammatik kann man widerum eine Sprache erzeugen \vl{FS 2}
            \item[Rechenoperationen] Vereinigung, Schnitt, Komplement, Produkt, Potenz, Kleene-Abschluss
            \item[Abschlusseigenschaft] Beispiel: Wenn Sprache A und Sprache B regulär sind, wäre dann auch der Schnitt der beiden Sprachen wieder regulär? \vl{FS 5}
        \end{description}

    \item[Automat]
        Beginnt von einem Startzustand und folgt je nach Eingabe seinen Übergängen in die jeweiligen Zustände.
        Akzeptiert, wenn letzter Zustand ein akzeptierter Endzustand.
        \begin{description}
            \item[Deterministischer endlicher Automat (DFA)] erkennen reguläre Sprachen \vl{FS 3}
            \item[nichtdeterministischer endlicher Automat (NFA)] „rät“ die richtigen Übergänge, arbeitet parallel. Nichtdeterminismus sinnvoll? Kompaktere Darstellungen, Start für Entwicklung DFA, kann bei Untersuchung Komplexität/Berechenbarkeit helfen \vl{FS 4}
            \item[Kellerautomat (PDA)]
                PDA erweitern endliche Automaten um einen unbeschränkt großen Speicher, der aber nur nach dem LIFO-Prinzip verwendet werden
                kann. PDAs erkennen genau die kontextfreien Sprachen. \vl{FS 15}
            \item[Turingmaschine (TM)]
                liefert allgemeines Modell der Berechnung. Liest und schreibt in einem Schritt, hat unendlichen
                Speicher, kann beliebig auf Speicher zugreifen (im Gegensatz zu LIFO bei PDA).
                Kann ein Band oder mehrere Bänder haben.
                Kann deterministisch (DTM) oder nichtdeterministisch (NTM) sein.
                Alle Varianten der TM können die selben Funktionen berechnen (Probleme lösen) - einzig der Aufwand ist unterschiedlich (NTM kann DTM darstellen, NTM kann durch DTM simuliert werden etc.).
                Siehe auch Church-Turing-These. \vl{FS 18}
        \end{description}


    \item[Kardinalität] Unterscheidung abzählbar (mit natürlichen Zahlen) und überabzählbar

    \item[Chomsky-Hierarchie]
        Kategorische Einteilung von Sprachen je nach Komplexität ihrer Grammatik. \\
        Hierarchie 0 > 1 > 2 > 3.
        These: „Die meisten Sprachen können nicht mit Grammatiken beschrieben
        werden (abzählbar viele Grammatiken vs. überabzählbar viele Sprachen)“. \vl{FS 2}
        \begin{description}
            \item[Typ 0] beliebige Grammatiken (Turingmaschinen)
            \item[Typ 1] kontextsensitive Grammatiken
            \item[Typ 2] kontextfreie Grammatiken (CYK, Kellerautomaten)
            \item[Typ 3] reguläre Grammatiken (DFA, NFA, Pumping Lemma)
        \end{description}

    \item[Probleme] Probleme formulieren Berechnungsfragen.
        \begin{description}
            \item[Wortproblem] Wortproblem für eine Sprache über einem Alphabet ist die Bestimmung der Ausgabe „ja, Wort ist in Sprache“ oder „nein, Wort ist nicht in Sprache“, für die Eingabe eines Wortes gebildet aus dem Alphabet \vl{FS 3}
            \item[Leerheitsproblem (DFA, NFA)] Entscheidung für „ja, Automat erzeugt Sprache“ oder „nein, durch den Automaten erzeugte Sprache ist leer“ (es wird nie ein Endzustand erreicht). \vl{FS 10}
            \item[Inklusionsproblem (DFA, NFA)] Entscheidung für „ja, Sprache A eines Automaten ist eine Teilmenge der Sprache B eines anderen Automaten“ oder „nein, Sprache A ist keine Teilmenge der Sprache B“. \vl{FS 10}
            \item[Äquivalenzproblem (DFA, NFA)] Entscheidung für „ja, Sprache A eines Automaten ist gleich der Sprache B eines anderen Automaten“ oder „nein, Sprache A unterscheidet sich von Sprache B“. \vl{FS 10}
            \item[Endlichkeitsproblem (DFA, NFA)] Entscheidung für „ja, erzeugte Sprache eines Automaten ist endlich“ oder „nein, erzeugte Sprache ist nicht endlich“ (z.B., wenn Zyklen auf dem Pfad von Start- zu Endzustand existieren). \vl{FS 10}
            \item[Universalitätsproblem (DFA, NFA)] Entscheidung für „ja, erzeugte Sprache eines Automaten ist $\Sigma^*$“ oder „nein, erzeugte Sprache ist nicht $\Sigma^*$“ (heißt, Komplement der erzeugten Sprache ist leer). \vl{FS 10}
            \item[Halteproblem (TM)] Entscheidet, ob eine Turingmaschine für eine Eingabe jemals hält oder nicht. Unentscheidbar. \vl{FS 19}
        \end{description}

    \item[Church-Turing-These]
        Die Turingmaschine kann alle Funktionen berechnen, die intuitiv berechenbar sind. Auch: „Eine Funktion ist genau dann
        im intuitiven Sinne berechenbar, wenn es eine Turingmaschine gibt, die für jede mögliche Eingabe den Wert der Funktion auf
        das Band schreibt und anschließend hält.“ \vl{FS 18}

    \item[Entscheidbarkeit]
        Eine Sprache L ist entscheidbar / berechenbar / rekursiv, wenn es eine Turingmaschine gibt, die das Wortproblem der Sprache L entscheidet.
        D.h. die Turingmaschine ist Entscheider und die Sprache L ist gleich der durch die Turingmaschine erkannten Sprache.
        Andernfalls heißt die Sprache L unentscheidbar. \\
        Die Sprache L ist semi-entscheidbar / Turing-erkennbar / rekursiv aufzählbar, wenn es eine Turingmaschine gibt, deren erkannte Sprache zwar L ist, jedoch die Turingmaschine kein Entscheider sein muss. \vl{FS 19}

    \item[Satz von Rice]
        %Informell: „Jede nicht-triviale Frage über die von einer TM ausgeführte Berechnung ist unentscheidbar.“ \\
        Sei E eine Eigenschaft von Sprachen, die für manche Turing-erkennbare Sprachen gilt und für manche Turing-erkennbare Sprachen nicht gilt (= „nicht-triviale Eigenschaft“). \\ Dann ist das folgende Problem unentscheidbar: die Eingabe besteht aus einer Turingmaschine. Wir wollen prüfen, ob die durch diese Turingmaschine erkannte Sprache die Eigenschaft E besitzt. Der Beweis für die Unentscheidbarkeit dieses Problems ist eine Reduktion vom Halteproblem. \vl{FS 20}

\end{description}

\newpage
\subsection{Aussagenlogik}
    Die Aussagenlogik untersucht logische Verknüpfungen von atomaren Aussagen. \vl{FS 21}

    \begin{description}
        \item[Atomare Aussage] Behauptungen, die wahr oder falsch sein können. \\ Auch: aussagenlogische Variablen, Propositionen, Atome
        \item[Operatoren, Junktoren] Verknüpfung von atomaren Aussagen. \\ Negation, Konjunktion, Disjunktion, Implikation, Äquivalenz. \\
            Können auch äquivalent durch andere Junktoren ausgedrückt werden (de Morgan). \vl{FS 22} \\
            Eine Disjunktion von Literalen nennt man \f{Klausel}. \\
            Eine Konjunktion von Literalen nennt man \f{Monom}.

        \item[Formel] Jedes Atom ist eine Formel, jede durch Junktoren verknüpfte Formeln sind wieder Formeln.
            Diese zusammengesetzten Formeln bestehen dann wieder aus Teilformeln (auch: Unterformeln, $Sub(F)$).
            Eine Formel, die nur aus einem Atom besteht, nennt man auch \f{Literal}. Literale können die Form $x$ oder $\neg x$ (für $x$ Atom) annehmen.
            \begin{description}
                \item[unerfüllbar] Formel hat keine Modelle
                \item[erfüllbar] Formel hat mindestens ein Modell
                \item[allgemeingültig] alle Interpretationen sind Modelle für Formel. Auch: \f{Tautologie}, $\models F$
                \item[widerlegbar] Formel ist nicht allgemeingültig
            \end{description}

        \item[Syntax] „Sprache einer Logik“ (Formeln mit logischen Operatoren). Wichtig: Klammerung.
        \item[Semantik] Definition der Bedeutung. Wertzuweisung von Wahrheitswerten zu Atomen mit Hilfe der Interpretation.
            „Die Bedeutung einer Formel besteht darin, dass sie uns Informationen darüber liefert, welche Wertzuweisungen möglich sind, wenn die Formel wahr sein soll.“
            \begin{description}
                \item[Interpretation] eine Funktion w, die von einer Menge Atome auf die Menge \{0, 1\} abbildet.
                \item[Wahrheitstabelle] Schrittweise Auflösung einer Formel durch Lösen ihrer Teilformeln.
                \item[Modell] eine Interpretation, dessen Abbildung eine Formel nach 1 löst.
                \item[Logische Konsequenz] eine Formel G ist eine logische Konsequenz einer Formel F ($F \models G$), wenn jedes Modell von F auch ein Modell für G ist.
                \item[Logische Äquivalenz] zwei Formeln F und G sind semantisch äquivalent ($F \equiv G$), wenn sie genau dieselben Modelle haben \vl{FS 22}
            \end{description}

        \item[Normalform] jede Formel lässt sich in eine äquivalente Formel in Normalform umformen. \\ Für die Umformungen gibt es Algorithmen, siehe \vl{FS 22}
            \begin{description}
                \item[Negationsnormalform (NNF)] enthält nur UND, ODER und Negation, wobei Negation nur direkt vor Atomen vorkommt.
                \item[Konjunktive Normalform (KNF)] Formel ist eine Konjunktion von Disjunktionen von Literalen.
                \item[Disjunktive Normalform (DNF)] Disjunktion von Konjunktionen von Literalen.
            \end{description}

        % Logisches Schließen: Resolution
        % Horn-Klauseln

    \end{description}


\newpage
\subsection{Komplexität}
\label{subsec:fs-komplexitaet}
Turingmaschinen sind begrenzt durch die Anzahl ihrer Speicherzellen (Speicher) und der Anzahl möglicher Berechnungsschritte (Zeit).
Schranken sind Funktionen gerichtet nach der Länge der Angabe. \vl{FS 24}

\begin{description}
    \item[$O$-Notation] charakterisiert Funktionen nach ihrem Verhalten und versteckt Summanden kleinerer Ordnung und lineare Faktoren. Beispiel: ein Polynom $n^{4} + 2n^{2} + 150$ wird zu $O(n^{4})$.
    \item[$O(f)$-zeitbeschränkt] es gibt eine Funktion $g \in O(f)$, so dass eine DTM/NTM bei beliebiger Eingabe der Länge $n$ nach einer maximalen Anzahl Schritte $g(n)$ anhält.
    \item[$O(f)$-speicherbeschränkt] es gibt eine Funktion $g \in O(f)$, so dass eine DTM/NTM bei beliebiger Eingabe der Länge $n$ nur eine maximale Anzahl Speicherzellen $g(n)$ verwendet.

    \item[Sprachklassen] Einteilung von Sprachen nach der Möglichkeit der Entscheidbarkeit. \\
        „Klasse aller Sprachen, welche…“
        \begin{description}
            \item[DTIME$(f(n))$] …durch eine $O(f)$-zeitbeschränkte DTM entschieden werden können
            \item[DSPACE$(f(n))$] …durch eine $O(f)$-speicherbeschränkte DTM entschieden werden können

            \item[NTIME$(f(n))$] …durch eine $O(f)$-zeitbeschränkte NTM entschieden werden können
            \item[NSPACE$(f(n))$] …durch eine $O(f)$-speicherbeschränkte NTM entschieden werden können
        \end{description}

    \item[Komplexitätsklassen] erfassen Sprachklassen je nach ihrer Komplexität.
        Stehen untereinander in Beziehung und bilden quasi Hierarchie. \vl{FS 24}
        \begin{description}
            \item[PTime (P)] deterministisch, polynomielle Zeit
            \item[ExpTime (Exp)] deterministisch, exponentielle Zeit
            \item[LogSpace (L)] deterministisch, logarithmischer Speicher
            \item[PSpace] deterministisch, polynomieller Speicher

            \item[NPTime (NP)] nichtdeterministisch, polynomielle Zeit
            \item[NExpTime (NExp)] nichtdeterministisch, exponentielle Zeit
            \item[NLogSpace (NL)] nichtdeterministisch, logarithmischer Speicher
            \item[NPSpace] nichtdeterministisch, polynomieller Speicher (gleich PSpace)
        \end{description}

    \item[SAT] Boolean Satisfiability Problem. Problem, welches ein Modell für eine Formel auf Erfüllbarkeit untersucht. In $NP$.
        Interessant für Untersuchung, da SAT ein Problem darstellt, für welches es wahrscheinlich schwierig ist eine Lösung zu finden,
        jedoch sehr einfach ist eine Lösung auf Korrektheit zu prüfen. \vl{FS 25}

    \item[Reduktion] Rückführung eines Problems auf ein anderes. Beispiel Drei-Farben-Problem ist auf SAT reduzierbar, da sich die Farb-Zustände als
        Formeln ausdrücken kodieren lassen. „Alle Probleme in NP können polynomiell auf SAT reduziert werden“ (\f{Cook, Levin})

    \item[Härte und Vollständigkeit] für P und NP \vl{FS 25}
        \begin{description}
            \item[NP-hart] Sprache ist NP-hart, wenn jede Sprache in NP polynomiell darauf reduzierbar ist \\
                (Beispiel Halteproblem und jedes weitere unentscheidbare Problem).
            \item[NP-vollständig] Sprache ist NP-hart und liegt selbst in NP (Beispiel SAT).

            \item[P-hart] Sprache ist P-hart, wenn jede Sprache in P mit logarithmischem Speicherbedarf auf diese reduzierbar ist.
            \item[P-vollständig] Sprache ist P-hart und liegt selbst in P (Beispiel HornSAT).
        \end{description}
\end{description}

\textit{Zusammenfassung aller Themenkomplexe, Hierarchien und Zusammenhänge in \vl{FS 26}.}

    \section{Theoretische Informatik}
\subsection{Turingmaschinen}
    \begin{description}
        \item[Turingmaschine] deterministisch als DTM oder nichtdeterministisch als NTM. \\
            Definiert als Tupel $(Q,\Sigma,\Gamma,\delta,q_0,F)$ mit endlicher Menge von Zuständen $Q$, Eingabealphabet $\Sigma$, Arbeitsalphabet $\Gamma$, Übergangsfunktion $\delta$, Startzustand $q_0$ und Menge von akzeptierenden Endzuständen $F$. Können ein oder mehrere Bänder haben. Siehe auch Church-Turing-These. \vl{FS 18} \vl{TIL 1}

            \begin{description}
                \item[Funktion] Turingmaschine kann eine Funktion von Eingaben auf Ausgabewörter definieren. Wenn eine TM bei Eingabe $w$ anhält und die Ausgabe der Form $v\blank\blank\dots$ entspricht, hat diese TM die Funktion berechnet.
                \item[Sprache] die von einer Turingmaschine erkannte Sprache ist die Menge aller Wörter, die von dieser TM akzeptiert werden (d.h. in einem Endzustand hält).

                \item[Konfiguration]
                    der „Gesamtzustand“ einer TM, bestehend aus Zustand, Bandinhalt und Position des Lese-/Schreibkopfs;
                    geschrieben als Wort (Bandinhalt), in dem der Zustand vor der Position des Kopfes eingefügt ist. Beispiel $ \blank\blank q_0aaba \blank\blank$.
                \item[Übergangsrelation]
                    Beziehung zwischen zwei Konfigurationen wenn die TM von der ersten in die zweite übergehen kann
                    (deterministisch oder nichtdeterministisch)
                \item[Lauf] mögliche Abfolge von Konfigurationen einer TM, beginnend mit der Startkonfiguration; kann endlich oder unendlich sein
                \item[Halten] Ende der Abarbeitung, wenn die TM in einer Konfiguration keinen Übergang mehr zur Verfügung hat.

                \item[Transducer] Ausgabe der Turingmaschine ist Inhalt des Bandes, wenn TM hält, ansonsten undefiniert. Endzustände sind irrelevant.
                \item[Entscheider] Ausgabe der Turingmaschine ist „Akzeptiert“, wenn TM in Endzustand hält, ansonsten „verwirft“ (beinhaltet auch „TM hält nicht“). Bandinhalt ist irrelevant.
                \item[Aufzähler] ist eine DTM, die bei Eingabe des leeren Bandes immer wieder (d.h. bis zum letzten Wort bei endlichen Sprachen) einen Zustand $q_{Ausgabe}$ erreicht, in welchem das aktuelle Band ein Wort aus der Sprache dieser DTM ist. Die Sprache dieser DTM ist dann die Menge der so erzeugten Wörter. Diese DTM muss nicht halten, die Sprache kann unendlich sein. Wörter dürfen mehrfach ausgegeben werden.
            \end{description}

        \item[Berechenbarkeit] bezogen auf Funktionen. Eine Funktion $F$ heißt berechenbar, wenn es eine DTM gibt, die $F$ berechnet. Ist durch geeignete Kodierung (z.B. binär) erweiterbar auf natürliche Zahlen, Wörterlisten und andere Mengen. \vl{TIL 2}
            \begin{description}
                \item[rekursiv] eine berechenbare totale Funktion ist rekursiv.
                \item[partiell rekursiv] eine berechenbare partielle Funktion ist partiell rekursiv.
            \end{description}

        \item[Entscheidbarkeit] bezogen auf Sprachen. \vl{TIL 2}
            \begin{description}
                \item[entscheidbar / berechenbar / rekursiv] es existiert eine Turingmaschine, die das Wortproblem der Sprache entscheidet. D.h. die Turingmaschine ist Entscheider und die Sprache ist gleich der Sprache der TM.
                \item[semi-entscheidbar / Turing-erkennbar / Turing-akzeptierbar / rekursiv aufzählbar] es existiert eine Turingmaschine, deren erzeugte Sprache gleich der Sprache ist, jedoch die TM kein Entscheider ist. \\
                Eine Sprache ist genau dann semi-entscheidbar, wenn es einen Aufzähler für diese Sprache gibt.
                \item[unentscheidbar] sonst. \\
                    „Es gibt Sprachen und Funktionen, die nicht berechenbar sind.“ Beweis anhand der abzählbaren Menge von Turingmaschinen im Vergleich zur Überabzählbarkeit der Menge der Sprachen über jedem Alphabet.
            \end{description}

        \item[Probleme] der Kategorie „Unentscheidbar bzw. unberechenbar, nicht berechenbar“. \vl{TIL 2}
            \begin{description}
                \item[Busy-Beaver-Funktion] ist nicht berechenbar und wächst sehr schnell. Die Funktion nimmt eine natürliche Zahl $n$ und gibt die maximale Anzahl $x$-Symbole, welche eine DTM mit $n$ Zuständen und dem Arbeitsalphabet $\{x,\blank\}$ bis zu ihrem Halt schreiben kann, zurück.
            \end{description}
    \end{description}

\subsection{LOOP und WHILE}
    LOOP und WHILE sind eine Erfindung von Schöning und sind quasi eine pädagogische Brücke zwischen den Ultra-low-level Turingmaschinen und High-level Programmiersprachen. WHILE baut auf LOOP auf. \vl{TIL 3}
    \begin{description}
        \item[LOOP] Besteht aus Variablen, Wertzuweisungen und Schleifen. Die Eingabe einer Menge von natürlichen Zahlen wird in $x_1, x_2, …$ gespeichert. Die Ausgabe ist eine natürliche Zahl, gespeichert in $x_0$. Alle weiteren Variablen haben den Wert $0$. LOOP terminiert immer in endlich vielen Schritten. Berechnet eine totale Funktion.
            \begin{description}
                \item[Variablen] Menge $\{x_0,x_1,…\}$ oder auch $\{x, y, \mathit{myVariable}\}$. Haben als Wert eine natürliche Zahl.
                \item[Wertzuweisungen] in der Form $x := y + n$ oder $x := y - n$, wobei $n$ eine natürliche Zahl ist. Eine Wertzuweisung ist bereits ein LOOP-Programm.
                \item[Schleifen] in der Form LOOP $x$ DO $P$ END, wobei $P$ wieder ein LOOP-Programm ist. Der Wert der Variable $x$ kann in $P$ nicht geändert werden. Daher terminiert ein LOOP-Programm immer in endlich vielen Schritten.
                \item[Hintereinanderausführung] wenn $P_0$ und $P_1$ LOOP-Programme, dann auch $P_0;P_1$.
                \item[Syntax-Erweiterung] Die Syntax lässt sich zur Vereinfachung erweitern.
                    \begin{description}
                        \item[Wertzuweisung \hili{$x:=y$}] $x:=y+0$
                        \item[Rücksetzen \hili{$x:=0$}] LOOP $x$ DO $x:=x-1$ END
                        \item[Wertzuweisung Zahl \hili{$x:=n$}] $x:=0;x:=x+n$. Alternativ $x:=null+n$
                        \item[Variablen-Addition \hili{$x:=y+z$}] $x:=y;$ LOOP $z$ DO $x:=x+1$ END
                        \item[Bedingung \hili{IF $x\neq0$ THEN}] LOOP $x$ DO $y:=1$ END $;$ LOOP $y$ DO $P$ END
                    \end{description}
                \item[Berechenbarkeit] eine Funktion heißt LOOP-berechenbar, wenn es ein LOOP-Programm gibt, welches die Funktion berechnet. Auch hier ist mit geeigneter Kodierung wieder mehr machbar, als nur die natürlichen Zahlen in Betracht zu ziehen (Beispiel Wortproblem, Probleme in NP, gängige Algorithmen). Es gibt berechenbare totale Funktionen, die nicht LOOP-berechenbar sind (vgl. Ackermannfunktion).
            \end{description}

        \item[WHILE] Basiert auf LOOP und erweitert dieses. Jedes LOOP-Programm ist auch ein WHILE-Programm.
            \begin{description}
                \item[Schleifen] in der Form WHILE $x \neq 0$ DO $P$ WHEN, wobei $P$ wieder WHILE-Programm. Im Gegensatz zu LOOP kann in WHILE der Wert von $x$ in $P$ zur Laufzeit geändert werden. Es kann also passieren, dass das Programm nicht terminiert wenn $x$ nie auf $0$ gesetzt wird.
                \item[Konvertierung] LOOP-Schleifen können in WHILE-Schleifen konvertiert werden. Eine DTM kann WHILE-Programme simulieren und ein WHILE-Programm DTMen simulieren.
                \item[Berechenbarkeit] Eine partielle Funktion heißt WHILE-berechenbar, wenn es ein WHILE-Programm gibt, welches bei einem definierten $f(n_0,n_1,…)$ terminiert und bei einem nicht definierten Wertebereich nicht terminiert. Wenn eine partielle Funktion WHILE-berechenbar ist, ist sie \f{Turing-berechenbar}.
            \end{description}
    \end{description}

\subsection{Universalität}
    \begin{description}
        \item[Universalmaschine $U$] eine Turingmaschine, die andere TM als Eingabe kodiert erhält und diese simuliert. Die Kodierung ist dabei z.B. binär, mit dem Trennsymbol $\#$. Hat vier Bänder: Eingabeband von $U$ mit kodierter TM und kodierter Eingabe $w$, Arbeitsband von $U$, Band 3 mit aktuellem Zustand der simulierten TM und Band 4 als Arbeitsband der simulierten TM. \\ Für die Arbeitsweise siehe \vl{TIL 4}
    \end{description}

\newpage
\subsection{Unentscheidbare Probleme und Reduktionen}
    Beweis durch Diagonalisierung, Reduktionen \vl{TIL 4}
    \begin{description}
        \item[Probleme] der Kategorie „unentscheidbar“.
        \begin{description}
            \item[Halteproblem \phalt] Frage: „Gegeben eine Turingmaschine $M$ und ein Wort $w$. Wird die Turingmaschine $M$ für die Eingabe $w$ jemals anhalten?“. Das Halteproblem \phalt der Turingmaschine $M$ für das Wort $w$ kann formal kodiert werden als $enc(M)\#\#enc(w)$ und einer universellen Turingmaschine zur Überprüfung übergeben werden. Beweise für Unentscheidbarkeit anhand Diagonalisierung und Reduktion in \vl{TIL 4}

            \item[Goldbachsche Vermutung] Beispiel für ein auf das Halteproblem reduzierbares Problem. Besagt, dass jede gerade Zahl $n \ge 4$ die Summe zweier Primzahlen ist. Zum Beispiel ist $4 = 2 + 2$ und $100 = 47 + 53$. Lässt man nun eine Turingmaschine diese Vermutung systematisch beginnend bei $4$ testen, würde ein Anhalten bei Misserfolg \phalt und „die Vermutung stimmt nicht“ gleichzeitig lösen. Gäbe es demnach ein Programm, welches \phalt lösen kann (entscheidet), wäre eine separate Überprüfung der Goldbachschen Vermutung nicht nötig. Die Frage der Goldbachschen Vermutung wäre sofort beantwortet.

            \item[$\epsilon$~-Halteproblem] „Gegeben sei eine Turingmaschine. Wird diese TM für die leere Eingabe $\epsilon$ jemals anhalten?“. Unentscheidbar.
        \end{description}

        \item[Beweismethoden] zum Nachweis der Unentscheidbarkeit.
        \begin{description}
            \item[Kardinalität] Beweis von Aussagen anhand der unterschiedlichen Kardinalitäten.
            \item[Diagonalisierung] Berechenbarkeit annehmen und einen paradoxen Algorithmus für das Problem konstruieren.
            \item[Reduktion] Rückführung eines unentscheidbaren Problems auf das gegebene. Die Reduktion ist ein Entscheid\-bar\-keits\-algorithmus. Siehe auch \verweis{subsec:fs-komplexitaet}. \vl{TIL 4}
                \begin{description}
                    \item[Turing-Reduktion] Ein Problem \prob{P} ist Turing-reduzierbar auf ein Problem \prob{Q} (in Symbolen: $\prob{P} \leq_T\prob{Q}$), wenn man \prob{P} mit einem Programm lösen könnte, welches ein Programm für \prob{Q} als Unterprogramm (auch: Subroutine) aufrufen darf. Das Programm für \prob{Q} muss hierbei nicht existieren.
                    \item[Many-One-Reduktion] Eine berechenbare totale Funktion $f: \Sigma^* \to \Sigma^*$ ist eine Many-One-Reduktion von einer Sprache \prob{P} auf eine Sprache \prob{Q} (in Symbolen: $\prob{P} \leq_m \prob{Q}$), wenn für alle Wörter $w \in \Sigma^*$ gilt: $w \in \prob{P}$ gdw. $f(w) \in \prob{Q}$. \\
                    Schwächer als Turing-Reduktion, jede Many-One-Reduktion kann als Turing-Reduktion ausgedrückt werden (dies gilt jedoch nicht andersherum).
                \end{description}
            \item[Satz von Rice] Siehe \verweis{subsec:fs-sprachen-automaten}. \vl{TIL 5} \\
                „Praktisch alle interessanten Fragen zu Sprachen von Turingmaschinen sind unentscheibar“. \\
                Eingabe ist eine Turingmaschine, Ausgabe „hat die Sprache der TM die Eigenschaft?“.
        \end{description}
    \end{description}


\subsection{Semi-Entscheidbarkeit}
    \textit{Hinweis: Hierzu gibt es im Schöning gute graphische Darstellungen}. \vl{TIL 5}
    \begin{description}
        \item[Komplement] einer Sprache $L$: $\overline{L} = \{w \in \Sigma^* ~|~ w \notin L \}$ (Achtung: auf Kontext achten. Komplement des Halteproblems ist z.B. anderer Form). Die Turing-Reduktionen $\overline{L} \leq_T L$ bzw. $L \leq_T \overline{L}$ sind mit einer Turingmaschine überprüfbar. Für eine Eingabe $w$ entscheidet diese, ob $w \in L$ und invertiert das Ergebnis.
        \item[Semi-Entscheidbarkeit] Beispiel anhand des Halteproblems: simuliere eine Turingmaschine und deren Eingabe, kodiert als $enc(M)\#\#enc(w)$. Wenn $M$ hält, hält auch die universelle Turingmaschine und akzeptiert. Eine Sprache $L$ ist entscheidbar, wenn sowohl $L$ als auch $\overline{L}$ semi-entscheidbar sind.
        \item[Co-Semi-Entscheidbarkeit] Wenn eine Sprache $L$ unentscheidbar, jedoch semi-entscheidbar ist, kann $\overline{L}$ nicht semi-entscheidbar sein.
    \end{description}


\newpage
\subsection{Postsches Korrespondenzproblem}
    Auch: \f{PCP}. Ein unentscheidbares Problem ohne direkten Bezug zu einer Berechnung. \vl{TIL 5}
    \begin{description}
        \item[PCP] Bei diesem Problem nimmt man eine Reihe von 2-Tupeln (anschaulich vergleichbar mit Dominosteinen) mit je einem Wert oben und einem unten. Ziel der Lösung ist nun, die gegebenen Tupel so anzuordnen, dass oben und unten die gleiche Wortkette entsteht. Beispiel: wir haben die drei Tupel (AB, B), (B, BBB) und (BB, BA). Eine Anordnung mit zehn Tupeln ergibt dann die Lösung. Es kann vorkommen, dass das Problem keine Lösung besitzt.
        \item[MPCP] Hilfskonstruktion. Wir nutzen MPCP, um das Halteproblem auf MPCP zu reduzieren. Folgend reduzieren wir MPCP auf PCP. Beim MPCP wird PCP verwendet, jedoch das Start-Tupel vorgegeben. Die Lösung eines MPCP ist auch eine Lösung des entsprechenden PCP, welche mit dem gegebenen Start-Tupel beginnt.
    \end{description}


\subsection{Unentscheidbare Probleme formaler Sprachen}
    In diesem Kapitel wird wieder auf \verweis{subsec:fs-sprachen-automaten} zurückgegriffen. Eine durch eine Grammatik $G$ erzeugte Sprache wird als $L(G)$ bezeichnet. Für Beweise der folgenden Sätze siehe Vorlesung. Siehe auch Chomsky-Hierarchie in \verweis{subsec:fs-sprachen-automaten}. \vl{TIL 6}
    \begin{itemize}
        \setlength\itemsep{0em}
        \item Das Schnittproblem regulärer Grammatiken (Typ 3) ist entscheidbar.
        \item Das Schnittproblem kontextfreier Grammatiken (Typ 2, \f{CFG}) ist unentscheidbar. Beweis durch Many-One-Reduktion vom PCP.
        \item Das Leerheitsproblem für kontextfreie Grammatiken ist entscheidbar.
        \item Kontextfreie Sprachen sind unter Vereinigung abgeschlossen.
        \item Deterministische kontextfreie Sprachen sind unter Komplement abgeschlossen.
        \item Das Äquivalenzproblem kontextfreier Grammatiken ist unentscheidbar.
    \end{itemize}


\newpage
\subsection{Komplexitätstheorie}
    Untersuchung von Problemkomplexitäten und Suche nach Methoden zur Bestimmung der Komplexität eines Problems. Klassierung zwischen „leicht lösbar“ bis „schwer lösbar“. Einteilung von berechenbaren Problemen entsprechend der Menge an Ressourcen, die zu ihrer Lösung nötig sind. Einführung anhand von Beispielen. \vl{TIL 7}
    \begin{description}
        \item[Eulerpfad] Ein Eulerpfad ist ein Pfad in einem Graphen, der jede Kante genau einmal durchquert. Ein Eulerkreis ist ein zyklischer Eulerpfad. Ein Graph hat genau dann einen Eulerschen Pfad, wenn er maximal zwei Knoten ungeraden Grades besitzt und zusammenhängend ist.
        \item[Hamiltonpfad] Ein Hamiltonpfad ist ein Pfad in einem Graphen, der jeden Knoten genau einmal durchquert. Ein Hamiltonkreis ist ein zyklischer Hamiltonpfad.

        \item[Schranken von Turingmaschinen] in Zeit und Raum. Siehe \verweis{subsec:fs-komplexitaet}.
            \begin{description}
                \item[Speicher] Zahl der verwendeten Speicherzellen
                \item[Zeit] Zahl der durchgeführten Berechnungsschritte
            \end{description}

        \item[$O$-Notation] Siehe \verweis{subsec:fs-komplexitaet}.
        \item[Linear Speedup Theorem] Sei $M$ eine Turingmaschine mit $k > 1$ Bändern, die bei Eingaben der Länge $n$ nach maximal $f(n)$ Schritten hält. Dann gibt es für jede natürliche Zahl $c > 0$ eine äquivalente $k$-Band Turingmaschine $M'$, die nach maximal $\frac{f(n)}{c} + n + 2$ Schritten hält. \\
        Bedeutet: in der Theorie kann jedes Programm mit Hilfe mehrerer Bänder „beliebig schneller“ gemacht werden. Dies ist praktisch nicht umsetzbar, da eine Turingmaschine nicht beliebig große Datenmengen in einem Schritt lesen und nicht beliebig komplexe Zustandsübergänge in konstanter Zeit realisieren kann.
    \end{description}

\subsection{Beziehungen der Komplexitätsklassen}
    Siehe \verweis{subsec:fs-komplexitaet} für eine Übersicht der Klassen. \vl{TIL 7}
    \begin{description}
        \item[Nichtdeterministische Klassen] $NL \subseteq NP \subseteq NPSpace \subseteq NExp$
        \item[DTM auch als NTM, d.h. nichtdet. stärker] $L \subseteq NL$, $P \subseteq NP$, $\mathit{PSpace} \subseteq \mathit{NPSpace}$, $Exp \subseteq NExp$
        \item[Satz von Savitch] Speicherbeschränkte NTM können durch DTMs nur mit  quadratischen Mehrkosten simuliert werden. Insbesondere gilt damit $\mathit{PSpace} = \mathit{NPSpace}$.
    \end{description}

    Zusammenfassend: $L \subseteq NL \subseteq P \subseteq NP \subseteq \mathit{PSpace} = \mathit{NPSpace} \subseteq Exp \subseteq NExp$. \\
    Jedoch ist zu beachten:

    \begin{itemize}
        \setlength\itemsep{0em}
        \item Wir wissen nicht, ob irgendeines dieser $\subseteq$ sogar $\subsetneq$ ist.
        \item Insbesondere wissen wir nicht, ob $P \subsetneq NP$ oder $P = NP$.
        \item Wir wissen nicht einmal, ob $L \subsetneq NP$ oder $L = NP$.
    \end{itemize}

    \newcommand{\INT}{\mathcal{I}}
\newcommand{\ZUW}{\mathcal{Z}}
\newcommand{\FM}{\mathcal{T}}

\section{Prädikatenlogik}
Die Prädikatenlogik erweitert die Aussagenlogik. Neben den neuen Mengen \f{V}, \f{C} und \f{P} kommen der Allquantor $\forall$ und der Existenzquantor $\exists$ hinzu. \f{Hinweis:} In dieser Vorlesung entfallen Funktionssymbole! \vl{TIL 13}
\subsection{Syntax}
    Im Gegensatz zu der unendlichen Menge von Atomen in der Aussagenlogik gibt es in der Prädikatenlogik mehrere betrachtete Mengen. Diese Mengen sind abzählbar unendlich und die Elemente disjunkt. Formeln sind, ausgenommen genannter Ausnahmen, eindeutig zu klammern. \vl{TIL 13}
    \begin{description}
        \item[Variablen] Die Menge \f{V}, bestehend aus $x, y, z\dots$. Variablen können frei oder gebunden vorkommen (oder bei mehrfachem Auftreten einer Variable in einer Formel auch beides). \f{Freie} Variablen sind durch keinen Quantor gebunden. \f{Gebundene} Variablen befinden sich innerhalb des „Scope“ eines Quantors. Beispiel: in der Formel $p(x) \land \exists x.q(x)$ kommt $x$ sowohl frei ($p(x)$) als auch gebunden ($q(x)$) vor.

        \item[Konstanten] Die Menge \f{C}, bestehend aus $a,b,c,\dots${}

        \item[Prädiktensymbole] Die Menge \f{P}, bestehene aus $p,q,r,\dots${}. Bei nullstelligen Prädiktensymbolen lassen wir die leeren Klammern weg.

        \item[Quantoren] Der Allquantor $\forall$ beschreibt, dass die betreffende Formel für alle möglichen Interpretationen der Variable gelten muss. Der Existenzquantor $\exists$ beschreibt, dass es mindestens eine gültige Interpretation der Variable geben muss. Wenn ein Quantor vor einer Formel mehrere Variablen betrifft, schreiben wir diese als Liste ($\forall x,y.F$ statt $\forall x.\forall y.F$).

        \item[Atom] Ein prädikatenlogisches Atom ist ein Ausdruck $p(t_{1},\dots,t_{n})$ für ein $n$-stelliges Prädiktensymbol $p \in \mathbf{P}$ und \f{Terme} $t_{1},\dots,t_{n} \in \mathbf{V} \cup \mathbf{C}$

        \item[Formel] Jedes Atom ist eine Formel. Wenn nun $x\in\f{V}$ und $F$ und $G$ Formeln, dann sind auch $\neg F$, $(F\land G)$, $(F\lor G)$, $(F\to G)$, $(F\leftrightarrow G)$, $\exists x.F$ und $\forall x.F$ Formeln. Die äußersten Klammern von Formeln dürfen weggelassen werden. Klammern innerhalb von mehrfachen Konjunktionen oder Disjunktionen dürfen weggelassen werden. Hat eine Formel keine freie Variablen ist sie \f{geschlossen} und wird \f{Satz} genannt, ansonsten ist sie eine \f{offene} Formel.

        \item[Teilformel] Teilformeln einer Formel sind alle Teilausdrücke einer Formel, welche selbst Formeln sind.
    \end{description}


\subsection{Semantik}
    Der Wahrheitswert von Formeln ergibt sich aus den Wahrheitswerten der Atome in dieser Formel. \vl{TIL 13}
    \begin{description}
        \item[Interpretation] Interpretation $\INT$  ist ein Paar $\langle\Delta^{\INT}, \cdot^{\INT}\rangle$.\\
            Die nichtleere Menge $\Delta^{\INT}$ wird auch \f{Domäne} genannt. \\
            Die Funktion $\cdot^{\INT}$ heißt \f{Interpretationsfunktion}. Diese bildet jede Konstante $a \in \f{C}$ auf ein Element $a^{\INT} \in \Delta^{\INT}$ und jedes $n$-stellige Prädiktensymbol $p \in \f{P}$ auf eine Relation $p^{\INT} \in (\Delta^{\INT})^{n}$ ab.

        \item[Zuweisung] Zuweisung $\ZUW$ für eine Interpretation $\INT$ ist eine Funktion $\ZUW: \f{V} \to \Delta^{\INT}$, sie bildet also Variablen auf Elemente der Domäne ab.
            Bei $x \in \f{V}$ und $\delta \in \Delta^{\INT}$ schreiben wir für die Zuweisung von $x$ auf $\delta$ und für alle $y \neq x$ auf $\ZUW (y)$: $\ZUW[x \mapsto \delta]$.{}

        \item[Wahrheitsbestimmung] Die Wahrheitsbestimmung von Atomen und Formeln unter einer Interpretation und einer Zuweisung werden rekursiv aufgelöst.
            \begin{itemize}
                \item Für Konstanten $c$ benötigen wir nur die Interpretation: $c^{\INT,\ZUW} = c^{\INT}$
                \item Für Variablen $x$ benötigen wir nur die Zuweisung: $x^{\INT,\ZUW} = \ZUW(x)${}
                \item Für Prädikate/Atome $p(t_{1},\dots,t_{n})$ setzen wir nun rekursiv: \\
                    $p(t_{1},\dots,t_{n})^{\INT,\ZUW} = 1$ wenn $\langle t_{1}^{\INT,\ZUW},\dots,t_{n}^{\INT,\ZUW} \rangle \in p^{\INT}$ bzw. \\
                    $p(t_{1},\dots,t_{n})^{\INT,\ZUW} = 0$ wenn $\langle t_{1}^{\INT,\ZUW},\dots,t_{n}^{\INT,\ZUW} \rangle \notin p^{\INT}$
            \end{itemize}

            Für eine Formel gilt nun: \\
            eine Interpretation $\INT$ und eine Zuweisung $\ZUW$ \f{erfüllen} eine Formel $F$, geschrieben „$\INT, \ZUW \models F$“, wenn die Rekursion mit Atomen, Operationen und Quantoren zu Wahr auflöst.
    \end{description}

\newpage
\subsection{Semantische Grundbegriffe}
    Wir wollen in der Prädikatenlogik wenn möglich nur mit Sätzen arbeiten, d.h. mit geschlossenen Formeln ohne ungebundene Variablen. \vl{TIL 14}
    \begin{description}
        \item[Modelltheorie] Wir unterscheiden grob zwischen der Prädikatenlogik mit und ohne offenen Formeln. Bei der Prädikatenlogik mit Sätzen können wir auf Zuweisungen verzichten. Formeln sind Behauptungen, die wahr oder falsch sein können. Modelle sind mögliche Welten (prädikatenlogische Interpretationen und ggf. Zuweisungen), in denen manche Behauptungen gelten und andere nicht. (\f{Intuition})

        \item[Typen von Formeln]
            Siehe hierzu auch die Graphen „Modelle $\models$ Formeln“ in \vl{TIL 14}
            \begin{itemize}
                \item allgemeingültig (tautologisch): Eine Formel, die in allen Modellen wahr ist
                \item widersprüchlich (inkonsistent): Eine Formel, die in keinem Modell wahr ist
                \item erfüllbar: Eine Formel, die in einem Modell wahr ist
                \item widerlegbar: Eine Formel, die in einem Modell falsch ist
            \end{itemize}

        \item[Logisches Schließen] Bei der Analyse von Modellen für Formeln und andersherum können in Wechselwirkung Konsequenzen hergestellt werden.
            \begin{enumerate}
                \item Wenn $\INT$ die Formel $F$ erfüllt, also $\INT \models F$, dann ist $\INT$ ein Modell für $F$.
                \item $\INT$ kann mehrere Formeln erfüllen, d.h. sie kann Modell für eine Formelmenge $\FM$ sein, wenn $\INT$ alle Formen in $\FM$ erfüllt.

                \item Eine Formel $F$ ist nun eine \f{logische Konsequenz} aus einer Formel bzw. Formelmenge $G$, d.h. $G \models F$, wenn jedes Modell $\INT$ von $G$ auch ein Modell von $F$ ist, d.h. $\INT \models G \implies \INT \models F$. \\
                Sonderfall: Ist $F$ eine Tautologie, dann schreiben wir nur $\models F$. \\

                Beispiel 1: Gegeben sind vier Modelle $\INT_{i}$ und vier Formeln $F_{j}$. $\INT_{2}$ und $\INT_{3}$ sind alle erfüllenden Modelle für $F_{3}$. $\INT_{2}$ und $\INT_{3}$ sind aber u.a. auch Modelle für $F_{2}$. Das bedeutet, wenn $F_{3}$ erfüllt ist, ist auch immer $F_{2}$ erfüllt. Es gilt $F_{3} \models F_{2}$. \\

                Beispiel 2: Im Beispiel der Logelei „Wir sind alle vom gleichen Typ“ haben wir fünf Formeln gegeben. Drei davon ergeben sich aus den gegebenen Aussagen („gegebene Theorie“) und die anderen beiden sind Allquantor-Behauptungen für „alle sagen die Wahrheit“ bzw. „alle lügen.“. Wir können anhand der Modelle „LL“, „WL“ und „WW“ und der Theorie Konsequenzen erstellen und somit über das Modell „WW“ die Behauptung „$\forall x.W(x)$“ als logische Konsequenz für unsere Theorie identifizieren.

                \item Zwei Formelmengen $F$ und $G$ können auch semantisch äquivalent sein, d.h. $F \equiv G$, wenn sie genau die gleichen Modelle haben ($\INT \models F$ gdw. $\INT \models G$ für alle Modelle $\INT$).
            \end{enumerate}
        \item[Semantische Äquivalenz] Eine Äquivalenzrelation $\equiv$ ist reflexiv, symmetrisch und transitiv. Alle Tautologien sind semantisch äquivalent. Alle unerfüllbaren Formeln sind semantisch äquivalent. \\ Äquivalenz $F \equiv G$ gdw. $F\models G$ und $G \models F$.
        \item[Problem logischen Schließens in der Prädikatenlogik] Die zwei Fragen „Model checking“ (Überprüfung eines Modells auf Erfüllung einer Formel) und „Logische Folgerung (Entailment)“ (Überprüfung ob zwei Formeln oder Formelmengen eine logische Konsequenz sind) sind in der Prädikatenlogik schwerer zu lösen als in der Aussagenlogik.
        \item[Monotonie und Tautologie] Aus der Definition von $\models$ folgt die Monotonie: je mehr Sätze in einer logischen Theorie gegeben sind, desto weniger Modelle können die gesamte Theorie erfüllen und desto mehr Schlussfolgerungen kann man aus der logischen Theorie ziehen. D.h. mehr Annahmen führen zu mehr Schlussfolgerungen. Extremfälle sind hierbei Tautologien (sind in jedem Modell wahr und daher logische Konsequenz jeder Theorie) und unerfüllbare Formeln (sind in keinem Modell wahr und haben daher alle anderen Sätze als Konsequenz).

        %\item[Beziehung zur Aussagenlogik] In der Semantik wird nur die Wertzuweisung ersetzt. Hier gibt es nun Interpretationen und Zuweisungen.

        \item[Gleichheit] Es gibt ein spezielles Gleichheitsprädikat $\approx$. In Interpretationen $\INT$ gilt $\approx^{\INT} = \{\langle\delta,\delta\rangle|\delta\in\Delta^{\INT}\}$.
        Dies kann z.B. zum Erzwingen von gleicher Interpretation von Konstanten verwendet werden. Auch gibt es $\not\approx$, Definition $\forall x,y.(x\not\approx y \leftrightarrow \neg x\approx y)$. Man kann aber mit Hilfe anderer Definitionen der Prädikatenlogik sowohl Gleichheit als auch Ungleichheit einsparen. \vl{TIL 14} \vl{TIL 15}
    \end{description}


\subsection{Prädikatenlogik als Universalsprache}
    Die Entwicklung der Logik hat ein zentrales Motiv: Logik als eine universelle, präzise Sprache. Die Entwicklung begann bei Aristoteles als Grundlage der philosophischen Argumentation, ging in Leibniz Sinne in Richtung „rechnen“ und wurde von Hilbert und Russell schließlich zusammen mit der Mathematik formalisiert. Wenn nun die Mathematik in logischen Formeln formuliert wird, wird logisches Schließen zur Kernaufgabe der Mathematik. Eine zentrale Frage des Schließens ist hierbei die Überprüfung auf Erfüllbarkeit einer Formel bzw. einer Formelmenge. \vl{TIL 15}
    \begin{description}
        \item[Strukturelle Induktion] Diese Induktion kann man über jede induktiv definierte syntaktische Struktur durchführen (z.B. Formeln, Terme, Programme,\dots).
            \begin{itemize}
                \item In der „klassischen Induktion“ wird eine Eigenschaft $E$ untersucht, wobei (1) „0 hat $E$“ geprüft und darauf aufbauend (2) für alle $n>0$ im Falle von „$n-1$ hat $E$“ geprüft wird.
                \item In der \f{strukturellen Induktion auf Formeln} prüfen wir nun ob (1) alle atomaren Formeln $E$ haben und (2) alle nicht-atomaren Formeln $F$ ebenfalls $E$ haben, wenn alle ihre echten Teilformeln $E$ haben.
            \end{itemize}

        Im Beispiel „Induktion auf der Insel der Wahrheitssager und Lügner. Ein Einwohner verkündet: 'Was ich jetzt sage, das habe ich schon einmal gesagt.' Welchen Typ hat er?“ muss der Einwohner ein Lügner sein, da er mindestens beim ersten Mal lügt.
    \end{description}


\subsection{Unentscheidbarkeit des logischen Schließens}
    Erinnerung: $F$ ist logische Konsequenz von $G$ ($F\models G$), wenn alle Modelle von $F$ auch Modelle von $G$ sind. (1) Es ist nicht offensichtlich, wie man das überprüfen sollte, denn es gibt unendliche viele Modelle. (2) Ebenso schwer erscheinen die gleichwertigen Probleme der Erfüllbarkeit und Allgemeingültigkeit. \\

    Intuition: prädikatenlogisches Schließen ist unentscheidbar. Beweis durch Reduktion eines bekannten unentscheidbaren Problems, z.B. Halteproblem, PCP, Äquivalenz kontextfreier Sprachen u.a. \\

    Der Beweis in der Vorlesung zeigt die Reduktion vom CFG-Schnittproblem. Hierfür werden Wörter $\omega$ aus der Modellmenge (Modellstruktur) $\INT$ als Ketten von binären Relationen kodiert und untersucht, ob das Wort $\omega$ in der Schnittmenge zweier kontextfreier Grammatiken $G_{1}$ und $G_{2}$ vorkommt. \\
    Beispiel: wir haben auf der Insel z.B. das Modell mit Kombination „LLWWW“ (drei sagen die Wahrheit, zwei lügen), und wir wollen wissen ob $F \models G$. Wir kodieren die erfüllenden Modelle der Formeln $F$ und $G$ wie o.g. und erhalten $G_{1}$ und $G_{2}$. Nach Kodierung müssten also in beiden Grammatiken die Übergänge $\langle L_{1},L_{2} \rangle, \langle L_{2},W_{1} \rangle, \langle W_{1},W_{2} \rangle, \langle W_{2},W_{3} \rangle$ vorkommen. Ist dies der Fall, dann erfüllt dieses Modell beide Formeln.  (\textit{Vergleich und Notation nicht nach VL!}) \\
    Zusammenfassend lassen sich demnach logische Konsequenzen auf diese Probleme reduzieren und da CFG unentscheidbar gilt auch: Logisches Schließen (Erfüllbarkeit, Allgemeingültigkeit, logische Konsequenz) in der Prädikatenlogik ist unentscheidbar. \vl{TIL 15}

\subsection{Gödel}
    Gödelscher Vollständigkeitssatz und Unvollständigkeitssätze. \vl{TIL 15}
    \begin{description}
        \item[Gödelscher Vollständigkeitssatz] „Es gibt ein konsistentes Verfahren, das alle Konsequenzen einer prädikatenlogischen Theorie effektiv beweisen kann.“ (1) Alle wahren Sätze können endlich bewiesen werden. (2) Prädikatenlogisches Schließen ist semi-entscheidbar.
        \item[1. Gödelscher Unvollständigkeitssatz] „Es gibt kein konsistentes Verfahren, das alle Konsequenzen der elementaren Arithmetik effektiv beweisen kann.“ (1) Für jedes Verfahren gibt es Sätze über elementare arithmetische Zusammenhänge, die weder bewiesen noch widerlegt werden können. (2) Die Wahrheit elementarer arithmetischer Zusammenhänge ist nicht semi-entscheidbar.
        \item[2. Gödelscher Unvollständigkeitssatz]
    \end{description}

\subsection{Algorithmen zum logischen Schließen}
    Resolution \vl{TIL 16}

\subsection{Syntaktische Umformungen}
    Ersetzungstheorem, Junktoren wie in Aussagenlogik. \\
    Neu: Äquivalenzen mit Quantoren \vl{TIL 16} \\
    Negationsnormalform (NNF), bereinigte Formel \vl{TIL 16} \\
    Pränexform, Skolemisierung, Einführung Funktionssymbole \vl{TIL 17} \\
    Konjunktive NF, Klauselform \vl{TIL 17}


    \newpage
    \section{Übungen}
Visuelle Hilfen, die in einigen Lösungen verwendet werden, versuche ich schriftlich zu beschreiben. Hier hilft es ggf., die gegebenen Lösungen nochmals selber aufzuschreiben um die logischen Schritte besser nachvollziehen zu können.

\subsection*{Übungsblatt 1 (Berechenbarkeitstheorie)}

\subsubsection*{Aufgabe 1}
    $|\N|$ bezeichnet die Kardinalität (Mächtigkeit) der Menge $\N$.
    \begin{enumerate}
        \item $|\N| = |\N \times \N|$: wir vergleichen hier die Kardinalität zweier Mengen. Um die Gültigkeit der Gleichung zu zeigen, betrachten wir also quasi $|A| = |B|$. Wir vergleichen also die Menge $\{0, 1, 2, 3, \dots\}$ mit der Menge $\{(0,0), (1,0), (0,1), (2,0), \dots\}$ und stellen fest, dass wir $1 \mapsto (0,0), 2\mapsto(1,0), \dots$ „mappen“ können. \\
        Als Hilfe kann man eine Matrix verwenden, die in x- bzw. y-Richtung x bzw. y im (x,y)-Tupel hochzählt. Die Elemente aus $|\N|$ werden nun diagonal darübergelegt.
        Zu zeigen ist, dass zwischen den Mengen $|A|$ und $|B|$ eine Bijektive Abbildung existiert.
        \begin{itemize}
            \item Injektiv: die Aufzählung ist ohne Wiederholung, d.h. in der Matrix steht an verschiedenen Stellen unterschiedliche Tupel
            \item Surjektiv: jedes Tupel der Menge existiert an einer Position in der Matrix.
        \end{itemize}


        \item $|\N| = |\Q|$: wie a), nur dass in $\Q$ negative Zahlen ebenfalls mit in die Matrix einbezogen werden. D.h. x besteht nun aus $\{1,-1,2,-2,\dots\}$. In dieser Menge erscheinen nun Duplikate, z.B. $\frac{1}{1}$ und $\frac{2}{2}$, welche wir einfach löschen. Die Abbildung $f(n)$ trifft also das $n$-te Element der Aufzählung nach Streichen von sich wiederholenden Elementen. Damit ist $f$ sowohl surjektiv als auch injektiv, da der Bruch $\frac{i}{j}$ spätestens an der Position $(i, j)$ vorkommt.

        \item $|\N| = |\R|$: wir zeigen hier nun Überabzählbarkeit. Dazu verwenden wir $|(0,1]| > |\N|$ und nehmen das Gegenteil an, d.h. $|(0,1]| = |\N|$. Dann gäbe es eine Aufzählung $r_{0}, r_{1}, r_{2},\dots$ von $(0,1]$. \\
            Sei $r_{i} = 0,b_{i1}b_{i2}b_{i3}$ (wobei $b_{ij} \in \{0,\dots,9\}$) die Dezimaldarstellung von $r_{i}$, die nicht schließlich $0$ ist. \\
            Nun nehmen wir eine Tabelle und verwenden die Diagonalisierung.

            \begin{tabular}{|l|l|}
                \hline
                $r_{0}$ & $0,b_{11}b_{12}b_{13}\dots$ \\
                $r_{1}$ & $0,b_{21}b_{22}b_{23}\dots$ \\
                $r_{2}$ & $0,b_{31}b_{32}b_{33}\dots$ \\
                \dots & \dots \\
                \hline
            \end{tabular}

            Betrachten wir nun die Zahl $\overline{r_{i}} = 0,\overline{b_{11}}\overline{b_{22}}\overline{b_{33}}$, wobei $\overline{b_{ij}} =
            \begin{cases}
                1\ wenn\ b_{ij} \neq 1 \\
                2\ sonst
            \end{cases}
            $

            Dann ist $\overline{r} \neq r_{i}$ für alle $i$, da sie sich in der $i$-ten Stelle unterscheiden.
    \end{enumerate}

\subsubsection*{Aufgabe 2}
    Sei $M$ eine Menge. Zeigen Sie, dass es keine surjektive Funktion $f: M \to \POT(M)$ gibt. Folgern Sie daraus, dass stets $|M| < |\POT(M)|$ gilt.

    \LOES Wir verwenden hierzu eine Matrix und zeigen anhand der Menge $D = \{x \in M | x \not\in f(x) \}$ einen Widerspruch. Zuerst die beliebig gefüllte Matrix: \\

        \begin{figure}[h]
            \centering{}
            \begin{tabular}{l|llll}
                M & $m_{0}$ & $m_{1}$ & $m_{2}$ & $m_{3}$ \\
                \hline
                $f(m_{0})$ & \cellcolor{light-red} x & & & x \\
                $f(m_{1})$ & & \cellcolor{light-red} & x & \\
                $f(m_{2})$ & & x & \cellcolor{light-red} & \\
                $f(m_{3})$ & x & & & \cellcolor{light-red} x \\
                \hline
                $D$ & & x & x & \\
                \hline
            \end{tabular}
        \end{figure}

    Die $x$ an Stellen $(m_{i}, f(m_{i}))$ bedeuten, dass das jeweilige $m_{i}$ in der abgebildeten Menge von $f(m_{i})$ enthalten ist. Die konstruierte Menge $D$ besteht nun zum Schluss aus genau den gegenteiligen Elementen zu jedem $m_{i}$.
    Angenommen, $f: M \to \POT(M)$ sei surjektiv. Dann gäbe es ein $y \in M$ mit $f(y) = D$. Es gilt dann $y \in D \Rightarrow y \not\in f(y) = D$ und $y \not\in D \Rightarrow y \in f(y) = D$. Widerspruch, es kann $f$ nicht geben.


\subsubsection*{Aufgabe 3}
    Konstruieren Sie eine Turing-Maschine $\mathcal{A}_{mul}$ , welche die Multiplikation zweier natürlicher Zahlen implementiert. Dabei sollen sowohl die Eingaben als auch die Ausgabe unär kodiert sein.

    \LOES Die Maschine erhält eine Eingabe, z.B. $000 \# 0000$ und soll demnach $3 * 4 = 12$ berechnen. Dafür kopiert sie pro Symbol vor dem $\#$ den unären Wert hinter dem Symbol auf das Band dahinter und löscht dann die ursprüngliche Eingabe.

    $000 \# 0000 \blank\blank  \mapsto^{*} \blank\blank 0 \# 0000 \blank 000000000 \blank  \mapsto^{*}  \blank 000000000000$

    \begin{figure}[h]
        \centering
        \begin{tabular}{ll}
            $q_{0}, 0, q_{1}, \blank, R$        & Auftrag: kopiere zweiten Faktor an das Ende \\
            $q_{1}, 0, q_{1}, 0, R$             & überspringe ersten Faktor \\
            $q_{1}, \# q_{2}, \#, R$            & Ende des ersten Faktors erreicht \\
            $q_{2}, 0, q_{3}, \hat{0}, R$       & Markiere das zu kopierende Zeichen \\
            $q_{3}, 0, q_{3}, 0, R$             & 0 überspringen \\
            $q_{3}, \blank, q_{4}, \blank, R$   & Trenner hinter zweiten Faktor \\
            $q_{4}, 0, q_{4}, 0, R$             & Ans Ende des Zwischenergebnisses \\
            $q_{4}, \blank, q_{5}, 0, L$        & Ende erreicht, Null schreiben und zurück \\
            $q_{5}, 0/\blank, q_{5}, 0/\blank, L$ & Zurück zum nächsten zu kopierenden Zeichen \\
            $q_{5}, \hat{0}, q_{2}, 0, R$       & Nächstes Zeichen kopieren \\
            $q_{2}, \blank, q_{6}, \blank, L$   & Zweiten Faktor fertig kopiert, zurück zum ersten \\
            $q_{6}, 0, q_{6}, 0, L$             & Zweiten Faktor überspringen \\
            $q_{6}, \blank, q_{0}, \blank, R$   & Anfang erreicht, bearbeite nächste Ziffer \\
            $q_{0}, \#, q_{7}, \blank, R$       & Erster Faktor aufgebraucht \\
            $q_{7}, 0, q_{7}, \blank, R$        & Löschen des zweiten Faktors \\
            $q_{7}, \blank, q_{f}, \blank, N$   & Alles gelöscht, Berechnung abgeschlossen \\
        \end{tabular}
    \end{figure}

    Daraus ergibt sich die TM: $\mathcal{A}_{mul} = (\{q_{0},\dots,q_{f}\}, \{0,\#\}, \Sigma \cup \{\hat{0}, \blank\}, \delta, q_{0}, \{q_{f}\})$

\subsubsection*{Aufgabe 4}
    Zeigen Sie: Wenn es möglich ist, für zwei beliebige Turing-Maschinen zu entscheiden, ob sie dieselbe Sprache akzeptieren, so ist es auch möglich, für beliebige Turing-Maschinen zu entscheiden, ob sie auf der leeren Eingabe halten. \\

    \LOES Wir konstruieren zwei TM, erstens $\TMM{\emptyset}$, welche die leere Sprache erkennt, und $\TMM{}$. Diese beiden werden dann in den gegebenen Algorithmus eingegeben und auf Äquivalenz getestet.
    Gegeben ist TM $\mathcal{K}$ die entscheidet, ob zwei TM $\TMM{1}$ und $\TMM{2}$ dieselbe Sprache akzeptieren. D.h. \\
    \boxed{\mathcal{K}(enc(\TMM{1}),enc(\TMM{2}))$ akzeptiert $\Leftrightarrow \LANG(\TMM{1}) = \LANG(\TMM{2})}. \\

    Gesucht ist eine TM $\mathcal{K}'$, die entscheidet, ob eine TM $\TMM{}$ auf $\epsilon$ hält, d.h.
    \boxed{\mathcal{K}'(enc(\TMM{}))$ akzeptiert $\Leftrightarrow M$ hält auf $\epsilon}. \\

    Idee: finde für $\TMM{}$ zwei TM $\TMM{1}$ und $\TMM{2}$, so dass \boxed{\TMM{}$ hält auf $\epsilon \Leftrightarrow \LANG(\TMM{1}) = \LANG(\TMM{2})}. \\
    Definiere $\TMM{1}$ als die TM, die alle Eingaben akzeptiert. \\
    Definiere $\TMM{2}$ als TM, die bei Eingabe $\epsilon$ die TM $\TMM{}$ auf $\epsilon$ simuliert und anschließend akzeptiert, und ansonsten akzeptiert. \\
    Simuliere $\mathcal{K}(enc(\TMM{1}), enc(\TMM{2}))$ und gib das Ergebnis zurück. \\

    Ablauf:
    \begin{itemize}
        \item $\mathcal{K}'$ hält auf jeder Eingabe
        \item Hält $\TMM{}$ auf $\epsilon$, dann ist \boxed{\LANG(\TMM{2}) = \Sigma^{*} = \LANG(\TMM{1})} und $\mathcal{K}'(enc(\TMM{}))$ akzeptiert
        \item Hält $\TMM{}$ auf $\epsilon$ nicht, dann ist \boxed{\LANG(\TMM{2}) = \Sigma^{*} \setminus \{\epsilon\} \neq \LANG(\TMM{1})} und $\mathcal{K}'(enc(\TMM{}))$ verwirft
    \end{itemize}

    Gezeigt: Äquivalenz von TM ist nicht entscheidbar.

\lstset{
    frame=single,
    basicstyle=\footnotesize,
    %~ backgroundcolor=\color{light-gray},
    rulecolor=\color{gray},
    %~ linewidth=220pt,
    xleftmargin=0.3cm,
    xrightmargin=3cm,
}

\subsection*{Übung 2}
\subsubsection*{Aufgabe 1}
Zeigen Sie, dass die folgenden Funktionen $f: \N^{2} \to \N$ LOOP-berechenbar sind:

    \begin{enumerate}
        \item $f(x,y) := max(x-y, 0)$
            \begin{lstlisting}
x := x1 + 0; y := x2 + 0
LOOP y DO  x := x - 1  END
x0 := x + 0
            \end{lstlisting}

        \item $f(x,y) := x \cdot y$
            \begin{lstlisting}
x := x1 + 0; y := x2 + 0
LOOP y DO                 # Fuer jedes y
    LOOP x DO             # addiere jedes x
        res := res + 1    # Ist bei Beginn 0
    END
END
x0 := res + 0
            \end{lstlisting}

        \item $f(x,y) := max(x,y)$
            \begin{lstlisting}
x3 := x1 + 0
LOOP x2 DO  x3 := x3 - 1  END
x0 := x2 + 0
LOOP x3 DO  x0 := x1  END
            \end{lstlisting}

            Fallunterscheidung $x1 \leq x2$ (dann ist $x3$ gleich $0$ nach der Schleife und $x0 = x2$) \\ und $x1 > x2$ (dann $x3 \neq 0$ und $x0 = x1$).

        \item $f(x,y) := ggT(x,y)$, wobei $ggT(x,y)$ den größten gemeinsamen Teiler von $x$ und $y$ bezeichnet.
            \begin{lstlisting}
x3 := max(x1, x2); x4 := min(x1, x2)
LOOP x3 DO
    x5 := max(x3-x4, x4)
    x6 := min(x3-x4, x4)
    x3 := x5
    x4 := x6
END
x0 := x3
            \end{lstlisting}

            Wobei $min(x,y)$ einfach $max(x, y)$ mit umgedrehter Ausgabe ist. LOOP statt WHILE, da die Durchläufe beschränkt werden müssen. $x3$ wird verwendet, da $ggT(n,1)$ mindestens einmal laufen soll.
    \end{enumerate}


\subsubsection*{Aufgabe 2}
    Mit $kgV(x_{1}, x_{2})$ bezeichnen wir das kleinste gemeinsame Vielfache zweier natürlicher Zahlen $x_{1}$ und $x_{2}$. Geben Sie ein WHILE-Programm an, das die Funktion $f: \N^{2} \to \N, (x_{1}, x_{2}) \mapsto kgV(x_{1}, x_{2})$ berechnet und erklären Sie seine Arbeitsweise.

    \LOES Wir nehmen hier die Funktion $kgV(x,y) = (x \cdot y) / ggT(x, y)$.
    \begin{lstlisting}
kgV(x1, x2) =
    IF x1 != 0 THEN
        IF x2 != 0 THEN
            x3 := x1 * x2
            x4 := ggT(x1, x2)
            x0 := div(x3, x4)
        END
    END

div(x1, x2) =
    x3 := x1 + 0
    x4 := x3 + 1        # Addiere 1, um Sonderfall x2==0 abzudecken
    x4 := x4 - x2       # Hier wuerde x2 nie dekrementiert, d.h. x4 nie 0
    WHILE x4 != 0 DO
        x0 := x0 + 1
        x3 := x3 - x2
        x4 := x3 + 1    # Wenn x2==0, dann nie x4=0 -> "Fehlercode"
        x4 := x4 - x2   #   fuer Division durch 0
    END
    \end{lstlisting}


\subsubsection*{Aufgabe 3}
    Es sei $\Sigma$ ein fest gewähltes Alphabet mit mindestens zwei Elementen. Wir betrachten eine Programmiersprache $L$ über $\Sigma$, die in der Lage ist, Turing-Maschinen zu simulieren. Für ein Wort $w \in \Sigma^{*}$ ist die Kolmogorov-Komplexität $K_{L}(w)$ die Länge des kürzesten Programms in $L$, welches bei leerer Eingabe das Wort $w$ als Ausgabe produziert. Zeigen Sie folgende Aussagen:
    \begin{enumerate}
        \item Es gibt für jede natürliche Zahl $n \in \N$ ein Wort $w \in \Sigma^{*}$ der Länge $|w| = n$, so dass $K_{L}(w) \geq n$.\\
            \LOES Gegenfrage: wie viele $w$ mit $|w| = n$ gibt es, so dass $K_{L}(w) < n$? \\
            Wir betrachten den Spezialfall $|\Sigma| = 2$, d.h. $\Sigma$ hat die minimal geforderten zwei Elemente.
            Dann gibt es höchstens $\sum\limits_{i=0}^{n-1} 2^{i} = 2^{n} - 1$ Wörter $w$ mit $K_{L}(w) < n$. Es gibt aber $2^{n}$ Wörter $w$ mit $|w| = n$, d.h. mindestens eines dieser $w$ erfüllt $K_{L}(w) \geq n$.

        \item Es gibt eine Konstante $c \in \N$, so dass gilt: Ist $w$ das Ergebnis der Berechnung einer Turing-Maschine $M$ mit Eingabe $x$, dann $K_{L}(w) \leq |\langle M, x \rangle| + c$, wobei $\langle M, x \rangle$ eine (effektive) Kodierung der Maschine $M$ und der Eingabe $x$ als ein Wort über $\Sigma$ ist. \\
            \LOES Idee: simuliere $M$ auf Eingabe $x$ als Programm in $L$. Betrachte folgendes Programm in $L$:
                %~ \begin{lstlisting}
%~ Simuliere $enc(M)$ auf $enc(x)$
%~ Gib das Ergebnis aus
                %~ \end{lstlisting}
                \begin{itemize}
                    \item Simuliere $enc(M)$ auf $enc(x)$
                    \item Gib das Ergebnis aus
                \end{itemize}
            Ist dann $c$ die Länge des Programms ohne $enc(M) + enc(x)$, dann ist $K_{L}(w) \leq |enc(M)\#\#enc(x)| + c$.

        \item Die Abbildung $w \mapsto K_{L}(w)$ ist nicht berechenbar. \\
            \LOES Annahme, $K_{L}(w)$ wäre berechenbar für alle $w \in \Sigma^{*}$. Dann konstruieren wir eine TM $M$, die für Eingaben $n \in \N$ Wörter $w$ mit $|w| = n$ und $K_{L}(w) \geq n$ ausgibt.
            Wir definieren $M$ wie folgt: \\
            $M =$ bei Eingabe $n$
            \begin{itemize}
                \item zähle Wörter $w \in \Sigma^{*}$, $|w| = n$ auf
                \item falls $K_{L}(w) \geq n$, gib $w$ aus
            \end{itemize}
            Wegen a) gibt $M$ für jede Eingabe $n$ ein Wort $w_{n}$ zurück. \\
            Dann ist $[n \leq K_{L}(w_{n}) \leq |enc(M)\#\#enc(n)| + c]$. \\
            $\to n \leq K_{L}(w_{n}) = c' + |enc(n)| = c' + log_{2}n$ (wobei $c' = c + |enc(M)\#\#|$).{}
            Diese Ungleichung gilt jedoch nicht für alle $n$. $n$ wächst schneller als $log_{2}n + c'$, stetig vs. logarithmisch.
    \end{enumerate}
    Damit ist insbesondere gezeigt, dass es niemals einen Compiler geben kann, der ein gegebenes Programm in ein kleinstmögliches übersetzt.


\subsection*{Übung 3 (Berechenbarkeitstheorie)}
\subsubsection*{Aufgabe 1}
    Zeigen Sie, dass es keine Many-One-Reduktion vom Halteproblem $P_{halt}$ von Turing-Maschinen auf des Leerheitsproblem $P_{leer} := \{enc(M)\ |\ \LANG(M) = \emptyset\}$ von Turing-Maschinen gibt. \\

    \LOES Aussage ist demnach $P_{halt} \not\leq_{m} P_{leer}$. Idee zum Beweis ist ein Widerspruch.
    \begin{itemize}
        \item Wenn $A \leq_{m} B$, und $B$ co-semi-entscheidbar (d.h. das Komplement des Problems ist semi-entscheidbar), dann muss auch $A$ co-semi-entscheidbar sein
        \item $P_{halt}$ ist unentscheidbar, jedoch semi-entscheidbar, demnach nicht co-semi-entscheidbar
        \item $P_{leer}$ ist unentscheidbar, aber co-semi-entscheidbar
    \end{itemize}
    Beweis mit Annahme $P_{halt} \leq P_{leer}$. Da $P_{leer}$ co-semi-entscheidbar ist, ist auch $P_{halt}$ co-semi-entscheidbar. Widerspruch!
    Es bleibt zu zeigen, dass $P_{leer}$ co-semi-entscheidbar ist. Dazu geben wir eine TM $M$ an, die $\overline{P_{leer}}$ erkennt. \textit{(Anmerkung: Es handelt sich hier um das Komplement des Problems, nicht um das Komplement einer Menge)} \\

\newpage
    $M$ erhält Eingabe $enc(N)$, ist NTM (andere Eingaben verwerfen) \\
    Sei $w_{0}, w_{1}, w_{2}, \dots$ eine Aufzählung von $\Sigma^{*}$. Definiere $M$ wie folgt:
    \begin{itemize}
        \item Für $i = 0, 1, 2, \dots${}
        \item Simuliere $N$ auf $w_{0}, w_{1}, \dots, w_{i}$ für $i$ Schritte
        \item akzeptiere, falls $N$ eines dieser Wörter akzeptiert
    \end{itemize}
    Durch das Probieren aller $w_{i}$ muss irgendwann ein Wort akzeptiert werden. Entsprechend ist $\exists w \in \overline{\LANG(N)}$ bewiesen und somit das Komplement entschieden. \\
    Simuliert $M$ die TM $N$ ohne die Begrenzung der Ausführungsschritte, kann es passieren, dass $N$ in eine Endlosschleife gerät bevor $M$ ein gültiges Wort testen konnte.


\subsubsection*{Aufgabe 2}
    Es sei $T := \{\ enc(M)\ |\ \mathit{M\ ist\ eine\ TM,\ welche\ w^{R}\ akzeptiert,\ falls\ sie\ w\ akzeptiert}\}$, wobei $w^{R}$ das zu $w$ umgekehrte Wort ist. Zeigen Sie, dass $T$ nicht entscheidbar ist. \\

    \LOES Wir verwenden hier den Satz von Rice und das Wissen, dass $P_{leer}$ unentscheidbar ist. \\
    Sei E eine nicht-triviale Eigenschaft (d.h. eine Eigenschaft die sowohl zutreffen kann als auch nicht zutreffen kann) semi-entscheidbarer Sprachen (d.h. nicht Turingmaschinen!). Dann ist $\{enc(M)\ |\ \LANG(M)\ erfüllt\ E\}$ unentscheidbar. \\
    Für $P_{leer}$ bedeutet dies: $L$ erfüllt $E \Longleftrightarrow L = \emptyset$ \\
    Für $T$: $L$ erfüllt $E \Longleftrightarrow (\forall w: w \in L \Leftrightarrow w^{R} \in L)$ \\
    $E$ ist nicht-trivial: $L = \emptyset, L = \{a\}$ erfüllen $E$; $L = \{ab\}$ erfüllt $E$ jedoch nicht


\subsubsection*{Aufgabe 3}
    Es sei $L := \{enc(G)\#\#enc(x)\ |\ \mathit{G\ kontextfreie\ Grammatik\ und\ x\ Teilwort\ eines\ Wortes\ aus\ \LANG(G)}\}$, wobei $enc(G)$ eine Kodierung von $G$ ist. Zeigen Sie, dass $\LANG$ auf das Komplement des Leerheitsproblems kontextfreier Grammatiken many-one-reduziert werden kann. Hinweis: der Schnitt einer regulären (Typ 3) mit einer kontextfreien Sprache (Typ 2) ist wieder kontextfrei. \\

    \LOES Hier $w_{1} \times w_{2} \in \LANG(G) \longrightarrow \SIGS \times \SIGS \cap \LANG(G) \neq \emptyset$. \\
    Also $enc(G)\#\#enc(x) \in \LANG \Longleftrightarrow \SIGS \times \SIGS \cap \LANG(G) \neq \emptyset$ \\
    Reduktion: $f(enc(G) \#\# enc(x)) = enc(G')$


\subsubsection*{Aufgabe 4}
    Zeigen Sie, dass jede semi-entscheidbare Sprache $L$ auf das Halteproblem $P_{halt}$ many-one-reduziert werden kann. \\

    \LOES Aussage demnach: ist $L$ semi-entscheidbar, dann gilt $L \leq_{m} P_{halt}$. Es gibt also eine TM $M$ mit $\LANG(M) = L$ und für die many-one-Reduktion muss es eine berechenbare Funktion $f$ von $L$ nach $P_{halt}$ geben. \\

    Idee: $w \in \LANG \Leftrightarrow M\ akzeptiert\ w \longrightarrow M'\ h"alt\ auf\ w'$ . \\
    Ziel: $f(w) = enc(M')\#\#enc(w')$ so dass $w \in L \Leftrightarrow M'$ hält auf $w'$. \\
    Definiere $M' =$ bei Eingabe $x$
    \begin{itemize}
        \item Simuliere $M$ auf $x$
        \item Falls $M$ akzeptiert, akzeptiere (halte)
        \item Ansonsten loop (Endlosschleife)
    \end{itemize}

    Beweis: Sei $L$ eine semi-entscheidbare Sprache. Sei $M$ TM mit $\LANG(M) = L$. Definiere für $w \in \SIGS$ den Wert $f(w) = enc(M')\#\#enc(w)$, mit $M'$ wie oben.
    Dann ist $f$ berechenbar. Es gilt $w \in L \Leftrightarrow f(w) \in P_{halt}$ und damit ist $L \leq_{m} P_{halt}$.

\newpage

\subsection*{Übung 4 (Berechenbarkeitstheorie)}
\subsubsection*{Aufgabe 1}
    Besitzen folgende Instanzen $P_{i}$ des Postschen Korrespondenzproblems Lösungen oder nicht?
    \begin{enumerate}
        \item Ja, einfach zu zeigen.
        \item Nein, denn der erste Stein ist der einzige, mit dem begonnen werden kann. Folgend passt nur der dritte Stein und nach diesem ebenfalls immer nur der dritte. Das untere Wort ist demnach immer länger als das obere.
        \item Ja, hat Lösung mit 66 Steinen.
    \end{enumerate}


\subsubsection*{Aufgabe 2}
    Zeigen Sie, dass das Postsche Korrespondenzproblem über einem einelementigen Alphabet entscheidbar ist.

    \LOES Dies lässt sich mit Hilfe eines Algorithmus lösen, welcher die Länge der Wortpaare untersucht.
    Sei $P = (a^{u_{1}}, a^{v_{1}}), \dots, (a^{u_{n}}, a^{v_{n}})$ über $\Sigma = \{ a \}$. Wir schreiben nun $(u_{i}, v_{i})$ statt $(a^{u_{i}}, a^{v_{i}})$, betrachten also nur die jeweilige Länge. Fallunterscheidung:
    \begin{itemize}
        \item 1. Fall: Es gibt ein $u_{i} = v_{i}$. Dann ist Paar $i$ die Lösung.
        \item 2. Fall: Alle $i$ sind derart, dass $u_{i} < v_{i}$ bzw. $u_{i} > v_{i}$, d.h. alle oberen bzw. unteren Wörter sind länger als das jeweils andere. Dann ist $P$ unlösbar.
        \item 3. Fall: Es gibt $i, j$ mit $u_{i} > v_{i}$ und $u_{j} < v_{j}$. Eine Lösung hat dann die Form $(\overbrace{i,i,\dots}^\text{k-mal},\overbrace{j,j,\dots}^\text{l-mal})$ (sofern Lösung existiert). Dann muss gelten $k \cdot u_{i} + l \cdot u_{j} = k \cdot v_{i} + l \cdot v_{j}$, also $k \cdot (u_{i} - v_{i}) = l \cdot (v_{j} - u_{j})$. Wähle $l = (u_{i} - v_{i})$, $k = (v_{j} - u_{j})$. Dann ist $(k \cdot i, l \cdot j)$ tatsächlich eine Lösung.
    \end{itemize}
    In jedem Fall ist also entscheidbar, ob es eine Lösung gibt. Damit ist die Aussage gezeigt.


\subsubsection*{Aufgabe 3}
    Zeigen Sie, dass folgendes Problem unentscheidbar ist: gegeben eine Turing-Maschine $M$ und ein $k \in \N$, kann die Sprache $L(M)$ durch eine Turing-Maschine mit höchstens $k$ Zuständen erkannt werden?
    Zeigen Sie dazu, dass für $k = 1$ die Menge $T_{k} := \{\ enc(M)\ |\ L(M)\ \text{wird von einer TM mit höchstens}\ k\ \text{Zuständen erkannt} \}$ nicht entscheidbar ist. Warum zeigt dies die ursprüngliche Behauptung? \\

    \LOES Wir setzen hier den Satz von Rice über die Unentscheidbarkeit von Eigenschaften von Sprachen an. (\textit{Wichtig! Nicht Eigenschaften von Maschinen!}) \\
    Wir betrachten den Fall $T_{1}$. $T_{1}$ ist nach Satz von Rice unentscheidbar. \\
    Was ist in diesem Falle die Eigenschaft $E$? Definition für $L \subset \SIGS$. \\
    \hili{$L$ erfüllt $E \Longleftrightarrow$ es gibt eine TM $N$ mit einem Zustand, so dass $L(N) = L$}. Dann ist $E$ Eigenschaft von Sprachen. Bemerkung: ist $L$ nicht semi-entscheidbar (benötigt für Satz von Rice!), dann gibt es keine TM $N$  mit $L = L(M)$. Also erfüllt $L$ die Eigenschaft $E$ nicht. \\
    $E$ ist nicht-trivial: $L = \emptyset$ funktioniert, TM hat keinen Endzustand. $L = \{ aa \}$ jedoch funktioniert nicht, da mit nur einem Zustand nicht gezählt werden kann.
    Also ist nach Satz von Rice die Maschine $T_{1}$ unentscheidbar. \\
    Da das Problem bereits für $k=1$ unentscheidbar ist, ist es auch für beliebige $k$ unentscheidbar. \\
    Der Sonderfall $k=0$ führt zu $T_{0} = \emptyset$. Ist $\emptyset$ entscheidbar? Ja, die TM lehnt einfach immer ab. Da dieser Fall jedoch trivial ist, fällt er nicht unter den Satz von Rice.

\newpage
\subsubsection*{Aufgabe 4}
    Zeigen Sie, dass weder das Äquivalenzproblem $\prspec{"aquiv}$ für Turing-Maschinen noch dessen Komplement $\overline{\prspec{"aquiv}}$ semi-entscheidbar ist, wobei
    \begin{itemize}
        \item $\prspec{"aquiv} := \{ enc(M_{1}) \#\# enc(M_{2}) | L(M_{1}) = L(M_{2}) \}$
        \item $\overline{\prspec{"aquiv}} := \{ enc(M_{1}) \#\# enc(M_{2}) | L(M_{1}) \neq L(M_{2}) \}$
    \end{itemize}
    Zeigen Sie dazu, dass $\phalt \leq_{m} \prspec{"aquiv}$ und $\phalt \leq_{m} \overline{\prspec{"aquiv}}$ gilt. Weshalb zeigt dies die Aussage? \\

    \LOES Angenommen $\paq$ wäre semi-entscheidbar. Dann $\overline{\paq}$ co-semi-entscheidbar. \\ Da $\phalt \leq_{m} \overline{\paq}$ muss auch $\phalt$ co-semi-entscheidbar. Widerspruch! Also ist $\overline{\paq}$ nicht semi-entscheidbar. \\
    Selbige Heransgehensweise gilt für die entsprechenden Komplemente. \\

    Wir zeigen $\phalt \leq_{m} \paq$. Dafür geben wir eine berechenbare Funktion $f: \SIGS \to \SIGS$ an, so dass \hili{$enc(M) \#\# enc(w) \in \phalt \Leftrightarrow f(enc(M) \#\# enc(w)) \in \paq$} für $M$ Turingmaschine und $w$ Eingabe.
    Dafür müssten wir zwei TM $M_{1}$ und $M_{2}$ finden, so dass gilt: \hili{$M$ hält auf $w \Leftrightarrow L(M_{1}) = L(M_{2})$}.
    Seien also $M$ und $w$ wie oben. \\

    Dafür setzen wir $M_{1} =$ bei Eingabe $w$
    \begin{itemize}
        \item akzeptiere ($L(M_{1}) = \SIGS$)
    \end{itemize}
    Und $M_{2} = $ bei Eingabe $y$
    \begin{itemize}
        \item simuliere $M$ auf $w$ (nimmt $y$, codiert dies als $w$)
        \item akzeptiere (bedeutet $M$ hat gehalten. Andernfalls würde $M$ nicht halten)
    \end{itemize}
    Bedeutet: $L(M_{2})$ ist $\SIGS$, falls $M$ auf $w$ hält. Sonst $L(M_{2}) = \emptyset$. \\
    Dann gilt: \hili{$M$ hält auf $w \Leftrightarrow L(M_{2}) = \SIGS = L(M_{1})$}. \\
    Daher ist \hili{$f(enc(M) \#\# enc(w)) := enc(M_{1})\#\# enc(M_{2})$} eine Reduktion von $\phalt$ auf $\paq$. \\
    Die Reduktion $\phalt \leq_{m} \overline{\paq}$ verläuft analog.

\subsection*{Übung 5 (Komplexitätstheorie)}
\subsubsection*{Aufgabe 1}
    Welche der folgenden Aussagen sind wahr? Begründen Sie Ihre Antwort.
    \begin{enumerate}
        \item Falls $P \neq \NP$ gilt, dann auch $P \cap \NP = \emptyset$.{}
        \item Es gibt Probleme, die NP-hart, aber nicht NP-vollständig sind.
        \item Polynomielle Reduzierbarkeit ist nicht transitiv.
        \item Ist $L_{2} \in P$ und $L_{1} \leq_{p} L_{2}$, dann ist auch $L_{1} \in P$.
        \item Ist $L_{1}$ eine NP-vollständige Sprache und gilt $L_{1} \leq_{p} L_{2}$, dann ist auch $L_{2}$ NP-vollständig.
        \item Ist $L_{2}$ eine NP-vollständige Sprache und gilt $L_{1} \leq_{p} L_{2}$, dann ist auch $L_{1}$ NP-vollständig.
    \end{enumerate}

    \LOES
    \begin{enumerate}
        \item Richtig. Derzeitiger Kenntnissstand: $P \subseteq \NP$, d.h. $P \cap \NP = P \neq \emptyset$.{}
        \item Richtig. Jedes NP-Problem ist bspw. in polynomieller Zeit auf das Halteproblem $\phalt$ reduzierbar, aber $\phalt$ ist nicht in NP (da unentscheidbar).
        \item Falsch. Reduktion ist transitiv, die Komposition von polynomiell-zeitberechenbaren Funktionen ist wieder polynomiell-zeitberechenbar. Formell: $C \leq_{p} B \leq_{p} A \Rightarrow C \leq_{p} A$.
        \item Richtig. Ein Entscheidungsverfahren für $L_{1}$, welches in polynomieller Zeit läuft, reduziert zuerst die Eingabe $w$ auf eine Instanz $f(w)$ für $L_{2}$ und prüft dann, ob $f(w) \in L_{2}$.
        \item Falsch. $L_{2}$ muss nur NP-hart sein. Beispiel $\phalt$.
        \item Falsch. Beispiel $L = \emptyset$, $\emptyset \leq SAT$.
    \end{enumerate}


\newpage
\subsubsection*{Aufgabe 2}
    Zeigen Sie, dass das Wortproblem deterministischer endlicher Automaten in $L$ liegt: ist \\
    $\prspec{DFA} := \{\ enc(A) \#\# enc(w)\ |\ A\ \text{ist ein DFA, der $w$ akzeptiert} \}$, dann gilt $\prspec{DFA} \in L$. \\

    \LOES Die Klasse $L$ ist $LogSpace$, d.h. der Automat hat zusätzlich zur Eingabe logarithmisch viel Platz für seine Berechnung. Die Klasse $L$ ist somit die Klasse von Problemen, die mit einer konstanten Anzahl von Zählern und Zeigern gelöst werden können. \\

    Für die Simulation von $A$ auf $w$ brauchen wir
    \begin{itemize}
        \item einen Zeiger, der auf den aktuellen Zustand zeigt
        \item einen Zeiger in die Eingabe $w$
        \item 2-3 Hilfszähler
        \item 1-2 Zähler, um Eingabe zu überprüfen
    \end{itemize}
    Wichtig: die Anzahl der Zähler/Zeiger hängt nicht von der Länge der Eingabe ab.
    Die Anzahl der für die Simulation benötigten Zähler und Zeiger liegt demnach in LogSpace.


\subsubsection*{Aufgabe 3}
    Es sei $L := \{\ a^{n}\ |\ n \in \N\ \MT{ist\ keine\ Primzahl} \}$. Zeigen Sie, dass $L \in \NP$ gilt. \\

    \LOES Demnach ist $L = \{ \epsilon, a, aaaa, aaaaaa, \dots \}$. Wir nutzen den Teiler von $n$ als Zertifikat. \\
    Ein nicht-deterministisches Entscheidungsverfahren für $L$, welches in polynomieller Zeit läuft ist folgendes: $M =$ bei Eingabe $a^{n}$:
    \begin{itemize}
        \item rate $p \in \N$ mit $1 < p < n$ (es gibt $\sqrt{n}$ viele $p$)
        \item prüfe ob $p$ ein Teiler von $n$ ist
        \item falls ja, akzeptiere, ansonsten verwerfe
    \end{itemize}

    Warum $L(M) = L$? Für jedes $a^{n} \in L$ gibt es mindestens einen akzeptierenden Lauf von $M$ auf $a^{n}$ und für $a^{n} \not\in L$ verwirft sie stets.
    Ist $M$ polynomiell zeitbeschränkt? Ja, denn Test lässt sich in polynomieller Zeit ausführen. Damit ist $L \in \NP$. Sogar $L \in P$, wenn einfach alle Zahlen durchprobiert werden.
    Der Primzahltest ist in $P$, wird jedoch komplexer bei der Kodierung ($log\ n \to n^{2}$).


\subsubsection*{Aufgabe 4}
    Zeigen Sie: ist $P = \NP$, dann gibt es einen Algorithmus, der in polynomieller Zeit für jede erfüllbare aussagenlogische Formel eine erfüllende Belegung findet. \\

    \LOES Idee ist die binäre Suche mit Teilformeln. \\
    Sei $\varphi$ eine aussagenlogische Formel mit Variablen $x_{1}, \dots, x_{n}$. Angenommen $\varphi$ ist erfüllbar. Betrachte die Formel $\varphi [x_{1} \leftarrow True ]$. Ist diese Formel erfüllbar (da $P = \NP$ kann hier SAT verwendet werden), setze $\beta(x_{1}) := \MT{True}$, ansonsten setze $\beta(x_{1}) := \MT{False}$. Berechne dann rekursiv eine erfüllende Belegung $\beta'$ für $\varphi [x_{1} \leftarrow \beta(x_{1})]$. Dann ist $\beta$ erweitert um $\beta'$ eine erfüllende Belegung für $\varphi$. \\

    Was ist die Laufzeit dieses Algorithmus? Da $P = \NP$ gibt es ein Polynom $p(n)$, welches die Laufzeit für den Erfüllbarkeitstest nach oben abschätzt. Dann läuft der Algorithmus oben in Zeit $O(n \cdot p(|\varphi|)) = O(|\varphi| \cdot p(|\varphi|))$, also in polynomieller Zeit in der Größe von $\varphi$. \\

    Wichtig: Backtracking ist hier nicht notwenig, da SAT alle weiteren Belegungen nach einer Belegung prüft.
    Klappt auch für 3SAT und CLIQUE.

\newpage
\subsection*{Übung 6 (Komplexitätstheorie)}
\subsubsection*{Aufgabe 1}
    Wir betrachten das folgende Problem $K$: Gegeben eine aussagenlogische Formel $\varphi$ mit $n$ Variablen. Gibt es eine erfüllbare Belegung von $\varphi$, bei der mindestens die Hälfte aller in $\varphi$ vorkommenden Variablen mit $\MT{True}$ belegt sind?
    \begin{enumerate}
        \item Formalisieren Sie dieses Problem als Sprache und zeigen Sie, dass $K \in \NP$ gilt.
        \item Zeigen Sie, dass $K$ ein NP-hartes Problem ist.
    \end{enumerate}

    \LOES
    \begin{enumerate}
        \item $K = \{\ enc(\varphi)\ |\ \varphi\ \text{aussagenlogische\ Formel,\ die\ eine\ erf"ullende\ Belegung\ hat,}$ \\
            $\text{\ \ die\ mindestens\ die\ Hälfte\ der\ in\ $\varphi$\ vorkommenden\ Variablen\ auf\ $\MT{True}$\ setzt} \}$. \\

            Was ist $\NP$? \boxed{\NP = \{\ L \subseteq \SIGS\ |\ L $ akzeptiert von NTM in polynomieller Zeit $\}}. \
            Es gilt $K \in \NP$, da eine entsprechende erfüllende Belegung geraten und in polynomieller Zeit geprüft werden kann.

        \item Wir müssen zeigen: alle Sprachen $L \in \NP$ können in polynomieller Zeit auf K reduziert werden, also $L \leq_{p} K$. Dazu genügt es zu zeigen, dass $SAT \leq_{p} K$ gilt. Dazu müssen wir in polynomieller Zeit zu einer aussagenlogischen Formel $\varphi$ eine aussagenlogische Formel $\varphi$ konstruieren, so dass $enc(\varphi) \in SAT \Leftrightarrow enc(\psi) \in K$.

        Seien $x_{1}, \dots, x_{n}$ Variablen in $\varphi$ und $y_{1}, \dots, y_{n}$ neue Variablen. Definiere $\psi := \varphi \land y_{1} \land \dots \land y_{n}$.

        Wir zeigen:

        $\Rightarrow$: Sei $\phi$ erfüllbar. Sei $\beta$ eine erfüllende Belegung von $\varphi$. Dann ist $\beta'(z):{}
        \begin{cases}
            \beta(z)\ falls\ z \in \{ x_{1}, \dots, x_{n} \} \\
            true\ falls\ z \in \{ y_{1}, \dots, y_{n} \}
        \end{cases}$

        Dann ist $\beta'(\psi) = True$ und $\beta'$ setzt mindestens die Hälfte der in $\psi$ vorkommenden Variablen auf $True (y_{1},\dots,y_{n})$. Also ist $enc(\psi) \in K$.

        $\Leftarrow$: Sei $enc(\psi) \in K$. Dann ist $\psi$ erfüllbar, also auch $\varphi$ erfüllbar. Schließlich kann $\psi$ aus $\varphi$ in polynomieller Zeit konstruiert werden und es folgt $SAT \leq_{p} K$.
    \end{enumerate}


\subsubsection*{Aufgabe 2}
    Im folgenden Solitaire-Spiel haben wir ein Spielbrett der Größe $m \times m$ gegeben. Als Ausgangsposition liegt auf jeder der $m^{2}$ Positionen entweder ein blauer Stein, ein roter Stein, oder gar nichts. Das Spiel wird nun so gespielt, dass Steine vom Brett genommen werden bis in jeder Spalte nur noch Steine einer Farbe liegen, und in jeder Zeile mindestens ein Stein liegen bleibt. In diesem Fall ist das Spiel gewonnen. Es ist möglich, dass man ausgehend von einer Ausgangsposition das Spiel nicht gewinnen kann.
    \begin{enumerate}
        \item Formalisieren Sie das Problem, für eine gegebene Ausgangsposition im Solitaire-Spiel zu entscheiden, ob es möglich ist, das Spiel zu gewinnen, als ein Entscheidungsproblem SOLITAIRE.
        \item Zeigen Sie, dass $\MT{SOLITAIRE} \in \NP$ gilt.
        \item Zeigen Sie, dass SOLITAIRE ein NP-hartes Problem ist, indem Sie zeigen, dass 3SAT in polynomieller Zeit auf SOLITAIRE reduzierbar ist.
    \end{enumerate}

    \LOES
    \begin{enumerate}
        \item $\MT{SOLITAIRE} = \{\ enc(f)\ |\ f: \{ 1,\dots,m \} \times \{ 1,\dots,m \} \to \{ blau, rot, nil \},$ \\
            $\text{\ \ so,\ dass\ das\ Spiel\ mit\ Anfangsbelegung\ f\ gewinnbar\ ist} \}$.

        \item $\MT{SOLITAIRE} \in \NP$, da als Zertifikat eine „Unterbelegung“ der Anfangsbelegung geraten werden kann, die eine Gewinnposition ist.
            Der Test, ob dabei eine Gewinnbelegung vorliegt, lässt sich in polynomieller Zeit durchführen.

\newpage
        \item Zeigen SOLITAIRE NP-hart, Reduktion von 3SAT: $\MT{3SAT} \leq_{p} \MT{SOLITAIRE}$.

            Sei $\varphi = \bigwedge\limits_{i=1}^{k} C_{i}$ eine 3CNF-Formel mit Klauseln $C_{1}, \dots, C_{k}$ und Variablen $x_{1}, \dots, x_{n}$.
            Wir konstruieren eine Anfangsbelegung $f$ für SOLITAIRE wie folgt:
            $f(i,j) =
            \begin{cases}
                \MT{blau\ falls\ } x_{j} \in C_{i} \\
                \MT{rot\ falls\ } \neg x_{j} \in C_{i} \\
                \MT{nil\ sonst}
            \end{cases}$

            Hierbei bezeichnet $i \in \{ 1,\dots,k \}$ die Klausel und $j \in \{ 1,\dots,n \}$ die Variable. Wir nehmen auch an, dass keine Klausel gleichzeitig $x$ und $\neg x$ enthält.

            Dies ergibt ein $k \times n$ Spielbrett. Dieses Brett kann quadratisch gemacht werden durch Duplizieren von Zeilen oder Hinzufügen leerer Spalten.

            Dann gilt: $\varphi$ ist erfüllbar gdw. das Spiel mit $f$ als Anfangsbelegung gewinnbar ist, \\
            d.h. $enc(\varphi) \in 3SAT \Leftrightarrow enc(f) \in \MT{SOLITAIRE}$.

            Behauptung: \boxed{\varphi$ erfüllbar $\Leftrightarrow f$ gewinnbar$}

            $\Rightarrow$: Sei $\varphi$ erfüllbar. Sei $\beta$ eine erfüllende Belegung von $\varphi$. Falls $\beta(x_{j}) = True$, entferne alle Steine aus der $j$-ten Spalte von $f$, die rot sind, andernfalls entferne alle blauen Steine. Sei $f'$ die daraus resultierende Brettposition.
            Dann ist $f'$ eine Gewinnposition, da in jeder Zeile immer noch ein Stein liegt. Betrachte dazu Zeile $i$.
            Dann ist $\beta(C_{i}) = True$, also gibt es ein Literal $l \in C_{i}$, mit $\beta(l) = True$. Ist $l$ wahr, d.h. $l = x_{j}$, dann liegt in $f$ auf Position $(i,j)$ ein blauer Stein.
            Da $\beta(x_{j}) = \beta(l) = True$, liegt dieser Stein auch in $f'$ auf Position $(i,j)$. Also liegt in Zeile $i$ ein blauer Stein. Der Fall, dass $l$ nicht wahr ist, liefert analog, dass in Zeile $i$ ein roter Stein liegt. In jedem Fall liegt in Zeile $i$ ein Stein und damit ist $f'$ eine Gewinnposition.

            $\Leftarrow$: Sei $f$ gewinnbar und sei $f'$ eine Gewinnposition für $f$.

            Definiere Variablenbelegung $\beta$ mit $\beta(x_{j})
            \begin{cases}
                True\ \text{falls in Spalte $j$ ein blauer Stein liegt} \\
                False\ \text{sonst (leere Spalten sind egal)}
            \end{cases}$

            Dann gilt $\beta(\varphi) = True$, da für jede Klausel $C_{i}$ gilt $\beta(C_{i}) = True$. Dies gilt, da in Zeile $i$ mindestens eine Position $(i,j)$ in $f'$ existiert, auf der ein Stein liegt. Ist dieser blau, dann ist $\beta(x_{j}) = True$ und $x_{j} \in C_{i}$ also $\beta(C_{i}) = True$, ist der Stein rot, so folgt analog $\beta(C_{i}) = True$.
    \end{enumerate}


\subsubsection*{Aufgabe 3}
    Sei $\Sigma$ ein Alphabet und $A, B \subseteq \SIGS$. Wir sagen, dass $A$ auf $B$ in logarithmischen Platz reduzierbar ist, und schreiben $A \leq_{l} B$, falls es eine Many-One-Reduktion von $A$ nach $B$ gibt, die in logarithmischen Platz berechenbar ist. Zeigen Sie: gilt $A \leq_{l} B$ und $B \leq_{l} C$, dann gilt auch $A \leq_{l} C$. \\

    \textit{Für diese Aufgabe ist eine Musterlösung gegeben, die Aufgabe wurde nicht in der Übung besprochen.}


%~ \newpage
\subsubsection*{Aufgabe 4}
    Wir betrachten das Problem SET-SPLITTING, welches für eine gegebene endliche Menge $S$ und eine Menge $\C = \{ C_{1}, \dots, C_{k}\}$ von Teilmengen von $S$ fragt, ob die Elemente von $S$ derart mit den Farben blau oder rot gefärbt werden können, so dass niemals alle Elemente einer Menge $C_{i}$ die gleiche Farbe bekommen. Zeigen Sie, dass SET-SPLITTING ein NP-vollständiges Problem ist. \\

    \LOES Frage: ist $(S, \C)$ färbbar? \\
    Behauptung: SET-SPLITTING ist NP-vollständig. \\
    Beweis: $\SESP \in \NP$ da eine korrekte Färbung als Zertifikat in polynomieller Zeit geraten und überprüft werden kann. Wir zeigen $\MT{CNFSAT} \leq_{p} \SESP$.{}

    Sei $\varphi = \bigwedge\limits_{i=1}^{k} C_{i}$ eine Formel in KNF. Wir konstruieren in polynomieller Zeit eine Instanz $(S_{\varphi}, \C_{\varphi})$ von SET-SPLITTING so, dass \boxed{\varphi$ erfüllbar $\Leftrightarrow (S_{\varphi}, \C_{\varphi})$ färbbar$}.
    Seien $x_{1}, \dots, x_{n}$ die in $\varphi$ vorkommenden Variablen. Definiere:
    \begin{itemize}
        \item $S_{\varphi} = \{ x_{1}, \dots, x_{n}, \neg x_{1}, \dots, \neg x_{n}, False \}$
        \item $\C_{varphi} = \{ \{ x_{1}, \neg x_{1} \}, \dots, \{ x_{n}, \neg x_{n} \}, C_{1}', \dots C_{k}' \}$
        \item $C_{i}' = C_{i} \cup \{ False \}$
        \item $\varphi = (x_{1} \lor x_{2} \lor x_{3}) = (x_{1} \lor x_{2} \lor x_{3} \lor False)$
    \end{itemize}

    $\Leftarrow$: Sei $(S_{\varphi}, \C_{\varphi})$ färbbar und sei $f$ eine entsprechende Färbung (ohne Einschränkung sei $f(False) = rot$).

    Definiere $\beta(x_{j}) =
    \begin{cases}
        True\ \text{falls $f(x_{j}) = blau$} \\
        False\ \text{sonst}
    \end{cases}$

    Behauptung: $\beta(\varphi) = True$. Sei $C_{i}$ eine Klausel von $\varphi$. Dann gibt es in $C_{i}'$ ein Element, welches Blau gefärbt ist. Da $f(farbe) = rot$ muss also ein Literal $l \in C_{i}$ existieren mit $f(l) = blau$. Ist $l = \neg x_{j}$ dann ist wegen $\{ x_{j}, \neg x_{j} \} \in \C_{\varphi}$ $x_{j}$ rot gefärbt.
    Also ist $\beta(x_{j}) = False$, $\beta(\neg x_{j}) = True$ und $\beta(C_{i}) = true$. % l = x_{j}

    $\Rightarrow$: analog mit Tauschen der Farbe. Aufgabe wie 6.2 lösen.

\subsection*{Übung 7 (Komplexitätstheorie)}
\subsubsection*{Aufgabe 1}
    Zeigen Sie, dass PSpace unter Komplement, Durchschnitt, Vereinigung, Konkatenation und Kleene-Stern abgeschlossen ist. \\

    \LOES \\
    \hili{PSpace = es gibt eine DTM, die in polynomiellem Platz ein Problem entscheidet.} (kann auch Loop beinhalten)
    \begin{enumerate}
        \item Durchschnitt: sind $L_{1}, L_{2} \in PSpace$ dann auch $L_{1} \cap L_{2} \in PSpace$. Seien dazu $M_{1}$ und $M_{2}$ zwei platzbeschränkte DTM, die $L_{1}$ bzw. $L_{2}$ entscheiden.
            Definiere M = bei Eingabe $w$
            \begin{itemize}
                \item simuliere $M_{1}$ auf $w$; verwerfe falls Simulation verwirft
                \item simuliere $M_{2}$ auf $w$; verwerfe falls Simulation verwirft
                \item akzeptiere
            \end{itemize}
            Dann ist $M$ polynomiell-platzbeschränkt, entscheidet $L_{1} \cap L_{2}$ und ist deterministisch.

        \item Vereinigung: wie Durchschnitt, nur „verwirft“ und „akzeptiere“ vertauscht.

        \item Komplement: $L \in PSpace$, $L = L(M)$, $M$ determinischer polynomiell-platzbeschränkter Entscheider.
            Konstruiere neue Maschine $M'$ durch Vertauschen der Final- und Nichtfinalzustände. Dann ist $M'$ ein deterministisch polynomiell-platzbeschränkter Entscheider für $\SIGS \setminus L$ also $\SIGS \setminus L \in PSpace$.

        \item Konkatenation: $L_{1}, L_{2} \in PSpace \Rightarrow L_{1} \circ L_{2} \in PSpace$ \\
            $L_{1} = L(M_{1})$, $L_{2} = L(M_{2})$

            $M =$ bei Eingabe $w$
            \begin{itemize}
                \item Für alle Belegungen $w = w_{1} w_{2}$ (Raten einer Zerlegung ist NPSpace, laut Savitch gleich PSpace)
                \item Simuliere $M_{1}$ auf $w_{1}$
                \item Simuliere $M_{2}$ auf $w_{2}$
                \item Akzeptiere, falls beide akzeptieren
                \item Verwerfe, sonst
            \end{itemize}

        \item Kleene-Stern: wie Konkatenation, nur mit $w=w_{1},\dots,w_{n}$ (wobei $n$ beliebig, maximal $|w|$), d.h. alle Teilstrings durchprobieren mit maximal $n$ DTM.
    \end{enumerate}


\subsubsection*{Aufgabe 2}
    Gomoku ist auch bekannt als „Fünf gewinnt“.

    %~ Wir betrachten das japanische Spiel Gomoku, welches von zwei Spielern X und O auf einem
    %~ 19×19-Brett gespielt wird. Die Spieler setzen abwechselnd ihre Steine auf das Brett, und
    %~ derjenige Spieler, der zuerst fünf Steine in einer Reihe (horizontal, vertikal, oder diagonal)
    %~ gelegt hat, gewinnt. Spieler X beginnt.
    %~ Verallgemeinertes Gomoku wird statt auf einem Brett fester Größe auf einem beliebigen
    %~ n×n-Brett gespielt. Eine Position in diesem Spiel ist eine Belegung der Felder des Spielbretts
    %~ 1mit Steinen der Spieler X und O, wie sie in einem wirklichen Spiel auftreten könnte. Sei

    \hili{$GM := \{\ enc(B)\ |\ B \text{ ist eine Position im verallgemeinerten Gomoku, in der X eine Gewinnstrategie hat } \}$}, wobei $enc(B)$ die zeilenweise Kodierung der Position $B$ über einem festen Alphabet ist. Zeigen Sie $GM \in PSpace$. \\

    \LOES Wir verwenden hierzu einen Baum, der vom Knotenpunkt B abgehend die möglichen Züge von X darstellt. Diese erste Ebene von Positionen, welche nach dem entsprechenden X-Zug entstehen, ist mit dem Existenzquantor markiert. Abgehend von einer Position gibt es nun weitere Baumblätter für die Züge von O, markiert mit dem Allquantor. Danach wieder X/$\exists$ und so weiter. All diese Positionen bedürfen maximal $\leq n^{2}$ Platz.
    \begin{itemize}
        \item Rekursive Definition finden für „gewinnbare Position“
            \begin{itemize}
                \item ist die Position für X bereits gewonnen, dann ist sie gewinnbar
                \item wenn X am Zug, dann ist Position gewinnbar, falls es einen Zug für X gibt, der in einer gewinnbaren Position mündet
                \item wenn O am Zug, dann ist Position gewinnbar, falls alle Züge zu einer gewinnbaren Position führen
            \end{itemize}

        \item Rekursive Auswertung, ob die Wurzel B gewinnbar ist mittels "Tiefensuche". Dies benötigt höchstens so viel Platz, wie der längste Ast im Spielbaum von B lang ist und das ist höchstens $O(n^{2} \cdot log\ n)$.

    \end{itemize}


\subsubsection*{Aufgabe 3}
    Welche der folgenden QBF-Formeln (quantified boolean formula) sind erfüllbar? Begründen Sie Ihre Antwort.
    \begin{enumerate}
        \item $W(\exists p_{1}.p_{1}) = W(p_{1}[p_{1}/\top] \lor W(p_{1}[p_{1}/\bot]))${}
        \item $W(\forall p_{1}.p_{1}) = \dots \land \dots = 1 \land 0 = 0$
        \item $W(\exists p_{1}.\bot) = 0$
        \item $\forall p_{1}.\exists p_{2}. p_{2} \to p_{1}$: Baum zeichnen, ist erfüllbar
        \item $\forall p_{1}. \exists p_{2}. \forall p_{3}. (p_{1} \lor p_{2}) \land p_{3}$: Baum zeichnen
        \item Trivial, nur $\lor \neg p_{3}$ betrachten.
    \end{enumerate}


\subsubsection*{Aufgabe 4}
    \textit{Diese Aufgabe war optional und wurde in der Übung nicht behandelt. Es gibt eine Musterlösung.}


%~ a) ∃p 1 .p 1
%~ b) ∀p 1 .p 1
%~ c) ∃p 1 .⊥
%~ d) ∀p 1 .∃p 2 .p 2 → p 1
%~ e) ∀p 1 .∃p 2 .∀p 3 .(p 1 ∨ p 2 ) ∧ p 3
%~ f) ∀p 1 .∀p 2 .∃p 3 .∀p 4 .(p 1 ∧ p 2 → p 4 ) ∨ ¬p 3
%~ Aufgabe 4
%~ Ein linear bounded automaton (LBA) ist eine deterministische Turing Maschine M , die bei
%~ jeder Berechnung niemals mehr Platz benutzt als bereits durch die Eingabe belegt ist. Dies
%~ erreicht M dadurch, dass sie niemals ein überschreibt und den Lesekopf nach links bewegt,
%~ sobald sie ein liest.
%~ Zeigen Sie, dass das Wortproblem für LBA
%~ P LBA := { enc(M)## enc(w ) | M ein LBA, der w akzeptiert }.
%~ ein PSpace-vollständiges Problem ist.

\subsection*{Übung 8 (Komplexitätstheorie)}
\subsubsection*{Aufgabe 1}
    \begin{enumerate}
        \item Ist Ist $P = \NP$, dann ist $\NP = \MT{coNP}$. \\
            \LOES Da deterministischer Automat für Komplement einfach invertiert werden kann gilt $P = coP$. Demnach $\NP = P = coP = \MT{coNP}$.

        \item Ist $P \neq \NP$, dann gilt $P \neq \MT{coNP}$, $L \neq \NP$ und $P \neq \MT{PSpace}$. \\
            \LOES Für $P = \MT{coNP}$ müsste gelten $P = coP = \MT{cocoNP} = NP$. \\
            Für $L = \NP$: $L \subseteq P \subseteq \NP \Rightarrow P = \NP$ \\
            Für $P = \MT{PSpace}$: $P \subseteq \NP \subseteq \MT{PSpace} \Rightarrow P = \NP$
    \end{enumerate}


\subsubsection*{Aufgabe 2}
    %~ Wir betrachten folgendes Scheduling-Problem: gegeben sind Prüfungen P 1 , ... , P k und
%~ Studierende S 1 , ... , S ` , so dass jede Prüfung von einer bestimmten Menge von Studierenden
%~ abgelegt wird. Die Aufgabe ist, die Prüfungen so in Zeitslots zu legen, dass niemand zwei
%~ Prüfungen im selben Zeitslot ablegen muss. Formalisieren Sie die Frage, ob solch ein
%~ Prüfungsplan mit höchstens h Zeitslots möglich ist, als eine formale Sprache und zeigen Sie,
%~ dass diese NP-vollständig ist. Nutzen Sie dazu die Tatsache, dass Färbbarkeit von Graphen
%~ NP-vollständig ist.

    Scheduling-Problem $SP$, Prüfungen $P_{1}, \dots, P_{k}$, Studierende $S_{1}, \dots, S_{l}$, maximale Anzahl Zeitslots $h$. \\
    Zeigen: Färbbarkeit von Graphen NP-vollständig. \\

    \LOES \\
    Formalisierung des Problems: \hili{$SP = \{\ enc(P, S, f, h)\ |\ \text{es existiert ein Plan für $(P, S, f)$ mit $h$ Zeitslots}\}$}. Hierbei Funktion $f: P \mapsto \POT(S)$, Zuordnung Studierende auf jeweilige Prüfungen.

    $SP \in \NP$. Rate Plan $g: P \mapsto \{\ 1, \dots, h\ \}$. Prüfe, ob Plan gültig.

    SP ist NP-hart. Angewandt auf Färbbarkeit: $\MT{F"arb} = \{\ enc(G, k)\ |\ G \text{ ist $k$-färbbar } \}$

    $\MT{3SAT} \leq_{p} \MT{F"arb} \leq_{p} SP$. Gesucht: $enc(G,k) \to enc(P,S,f,h)$. $G = (V,E) \to (V,E,f,k)$ \\
    $f(v) = \{\ (u_{1}, u_{2}) \in E\ |\ u_{1} = v \text{ oder } u_{2} = v\ \}$.

    $g(w) =
    \begin{cases}
        enc(V,E,f,k) \text{ falls } w = enc(G,k) \\
        \epsilon sonst
    \end{cases}$

    Reduktion: \hili{$G$ ist $k$-färbbar $\Leftrightarrow$ es existiert ein Plan für $(V,E,f)$ mit $k$ Zeitslots.}

    „$\Rightarrow$“ $\exists h$: $V \to \{\ 1, \dots, k\ \}$ gültige Färbung von $G$. $\to h$ ist Plan für $(V,E,f)$, da $P_{1} \neq P_{2}$. $S \in f(P_{1})$ und $S \in f(P_{2})$. $\Rightarrow S = (P_{1}, P_{2})$ oder $S=(P_{2}, P_{1})$. $\Rightarrow h(P_{1}) \neq h(P_{2})$

    „$\Leftarrow$“ $\exists h$: $V \to \{\ 1, \dots, k\ \}$ gültiger Plan für $(V,E,f)$ $\Rightarrow h$ ist gültige Färbung für $G$, da $(u,v) \in E$ (wobei $(u,v) = e$)
        $\Rightarrow e \in f(v)$ und $e \in f(u)$ $\Rightarrow h(v) \neq h(u)$ .

    Hier: Übersetzung eines Färbbarkeitsproblems in ein $SP$. D.h. es existiert nur der Fall, „Student schreibt zwei Prüfungen“ ($(u_{1}, u_{2}) \in E$).
    $SP \leq_{p} \MT{F"arb}$ würde ein allgemeines Problem $SP$ mit beliebigen Prüfungen auf ein $\MT{F"arb}$ reduzieren (komplexer, nicht immer möglich).


\subsubsection*{Aufgabe 3}
    Zeigen Sie, dass folgendes Problem unentscheidbar ist: gegeben eine Turing-Maschine $M$ und eine Zahl $k \in \N$, ist $M$ eine $O(n^{k})$-zeitbeschränkte Turing-Maschine? \\

    \LOES \\
    $\prspec{poly} = \{\ enc(M) \#\# enc(k)\ |\ M \text{ läuft in } O(n^{k})\ \}$. Wobei $n$ Länge der Eingabe.\\
    Wir suchen nun die Reduktion $\phalt \leq_{m} \prspec{poly}$ um \hili{$M$ hält auf $w \Leftrightarrow M'$ läuft in $O(n^{2})$} zu konstruieren. \\
    TM: $enc(M) \#\# enc(w) \mapsto enc(M') \#\# enc(2)$ (hier $k = 2 \in \N$ fest gewählt). \\

    $M'(U) = $ simuliere $M$ auf $w$ in $|U|^{2}$ Schritten ($|U| = n$).
    \begin{itemize}
        \item Wenn $M$ hält auf $w$ dann akzeptiere
        \item Sonst loope für $|U|^{3}$ Schritte und akzeptiere
    \end{itemize}

\subsection*{Übung 9 (Prädikatenlogik)}
\subsubsection*{Aufgabe 1}
Welche der angegebenen Strukturen sind Modelle der folgenden Formel?
\begin{equation*}
\forall x.p(x,x) \land \forall x,y.((p(x,y) \land p(y,x)) \to x \approx y) \land \forall x, y, z.((p(x,y) \land p(y,z)) \to p(x,z))
\end{equation*}
\begin{enumerate}
\item $\INT_1$ mit Grundmenge $\N$ und $p^{\INT_1} = \{(m,n) \mid m < n\}$;
\item $\INT_2$ mit Grundmenge $\N$ und $p^{\INT_2} = \{(m,n + 1) \mid n \in \N \}$;
\item $\INT_3$ mit Grundmenge $\N$ und $p^{\INT_3} = \{(m,n) \mid m teilt n\}$;
\item $\INT_4$ mit Grundmenge $\SIGS$ für ein Alphabet $\Sigma$ und $p^{\INT_4} = \{(x,y) \mid x \text{ ist Präfix von } y \}$;
\item $\INT_5$ mit Grundmenge $\POT(M)$ für eine Menge $M$ und $p^{\INT_5} = \{(X,Y) \mid X \subseteq Y \}$;
\end{enumerate}
\LOES 
\begin{equation*}
\underbrace{\forall x.p(x,x)}_{\substack{{p(x,x) = \top} \\ \\ \text{p wird als reflexive} \\ \text{Relation interpretiert}}} \land \underbrace{\forall x,y.((p(x,y) \land p(y,x)) \to x \approx y)}_{\substack{{x \leq y, y \leq x, x=y} \\ \\ \text{p wird als antisymmetrische} \\ \text{Relation interpretiert}}} \land \underbrace{\forall x, y, z.((p(x,y) \land p(y,z)) \to p(x,z))}_{\substack{\text{transitivität} \\ \\ \text{p wird als transitive} \\ \text{Relation interpretiert}}}
\end{equation*}
$\Rightarrow$ Die gesamte Formel beschreibt die Theorie der Ordnungsrelation.
\begin{enumerate}[leftmargin=1cm]
\item[Zu a)] Kein Modell, denn $(2,2) \notin p^{\INT_1}$
\item[Zu b)] Kein Modell, denn $(1,2),(2,3) \in p^{\INT_2}$, aber $(1,3) \notin p^{\INT_2}$.
\begin{align*}
\text{\f{Genauer:}} \\
\text{Sei } \ZUW: &x \mapsto 1 \text{, dann gilt } \INT_2, \ZUW\#p(x,y) \land p(y,z) \to p(x,z) \\
& y \mapsto 2 \text{, also folgt } \INT_2\#\forall x, y, z.(p(x,y) \land p(y,z) \to p(x,z)) \\
& z \mapsto 3 \text{, und damit ist } \INT_2 \text{ kein Modell der Formel}  
\end{align*}
\item[Zu c)] Ist ein Modell, denn Teilbarkeit ist eine Ordnungsrelation auf $\N$.
\item[Zu d)] Ist ein Modell, denn Präfixrelation ist eine Ordnungsrelation auf $\N$.
\item[Zu e)] Ist ein Modell, denn $\subseteq$ ist eine Ordnungsrelation auf $\N$.
\end{enumerate}

\subsubsection*{Aufgabe 2}
\begin{enumerate}
\item Geben Sie eine erfüllbare Formel in Prädikatenlogik mit Gleichheit an, so dass alle Modelle 
	\begin{enumerate}[label=\roman*)]
	\item höchstens drei, \\
	\LOES $F_{\leq 3} := \exists x,y,z.\forall w.(w \approx x \lor w \approx y \lor w \approx z)$
	\item mindestens drei,\\
	\LOES $F_{\geq 3} := \exists x,y,z.(\underbrace{x \not\approx y}_{= \neg (x \approx y)} \land y \not\approx z \land x \not\approx z)$
	\item genau drei \\
	\LOES $F_{=3} := F_{\leq 3} \land F_{\geq 3}$
	\end{enumerate}
	Elemente in der Grundmenge besitzen.
\item Geben Sie je eine erfüllbare Formel in Prädikatenlogik mit Gleichheit an, so dass das zweistellige Relationensymbol $p$ in jedem Modell als der Graph einer

	\begin{enumerate}[label=\roman*)]
	\item injektiven Funktion, \\
	\LOES $F_{fun} := \underbrace{\forall x.\exists y. p(x,y)}_{\substack{{p \text{ wird als linkstotale}} \\ \text{Relation interpretiert}}} \land \underbrace{\forall x, y, z.(p(x,y) \land p(x,z) \to y \approx z)}_{\substack{{p \text{ wird als rechtseindeutige}} \\ \text{Relation interpretiert}}}$ \\
	$F_{inj} := F_{fun} \land \forall x,y,z.(p(x,z) \land p(y,z) \to x \approx y)$
	\item surjektiven Funktion, \\
	\LOES $F_{sur} := F_{fun} \land \forall y. \exists x.p(x,y)$
	\item bijektiven Funktion \\	
	\LOES $F_{bij} := F_{inj} \land F_{sur}$
	\end{enumerate}
	interpretiert wird. \\
	(Der Graph einer Funktion $f: A \to B$ ist die Relation $\{(x,y) \in A \times B \mid f(x) = y \}$.)
\end{enumerate}

\subsubsection*{Aufgabe 3}
Welche der folgenden Aussagen sind wahr? Begründen Sie Ihre Antwort.
\begin{enumerate}
\item Sind $\Gamma$ und $\Gamma'$ Mengen von prädikatenlogischen Formeln, dann folgt aus $\Gamma \subseteq \Gamma'$ und $\Gamma \models F$ auch $\Gamma' \models F$. \\
\LOES \textcolor{green}{Ja}, $\Gamma \subseteq \Gamma'$ und $\overbrace{\Gamma \models F}^{\forall \text{ Strukturen } \INT: \INT \models \Gamma \Rightarrow \INT \models F}$  impliziert $\overbrace{\Gamma' \models F}^{\forall \text{ Strukturen } \INT: \INT \models \Gamma' \Rightarrow \INT \models F}$. \\
Die Aussage gilt: Sei $\INT \models \Gamma'$. Wegen $\Gamma \subseteq \Gamma'$ folgt $\INT \models \Gamma$. \\
Mit $\Gamma \models F$ folgt $\INT \models F$. Also insgesamt haben wir $\Gamma' \models F$ gezeigt.
\item Jede aussagenlogische Formel ist eine prädikatenlogische Formel. \\
\LOES \textcolor{green}{Ja}, mit der in der VL gezeigten Einbettung von Aussagenlogik in Prädikatenlogik.
\item Eine prädikatenlogische Formel $F$ ist genau dann allgemeingültig, wenn $\neg F$ unerfüllbar ist. \\
\LOES \textcolor{green}{Ja}, denn: 
\begin{align*}
F \text{ allgemeingültig } &\Leftrightarrow \forall \text{ Strukturen } \INT:\INT \models F \\
& \Leftrightarrow \forall \text{ Strukturen } \INT: \INT \not\models \neg F \\
& \Leftrightarrow \neg F \text{ unerfüllbar }
\end{align*}
\item Es gilt 
\begin{equation*}
\{\forall x,y.(p(x,y) \to p(y,x)), \forall x,y,z.((p(x,y) \land p(y,z)) \to p(x,z))\} \models \forall x.p(x,x).
\end{equation*}
\LOES \textcolor{orange}{Falsch!} Übersetzt wird hier gefragt, ob aus Symmetrie und Transitivität einer binären Relation stets ihre Reflexivität folgt. \\
Gegenbeispiel: $\INT := (\{d\},\{p \mapsto \emptyset \})$
\end{enumerate}
\subsubsection*{Aufgabe 4}
\label{U9-4}
Formalisieren Sie Bertrand Russells Barbier-Paradoxon 
\begin{center}
\textit{Der Barbier rasiert genau diejenigen Personen, die sich nicht selbst rasieren.}
\end{center}
als eine prädikatenlogische Formel und zeigen Sie, dass diese unerfüllbar ist. \\\\
\LOES Wir verwenden die Menge $C := \{Barbier\}$ als Menge der Konstanten und die Menge $\f{P} := \{rasiert\}$ als Menge der Prädikatensymbole. Nun formulieren wir die gegebene Aussage in Prädikatenlogik wie folgt:
\begin{equation*}
F := \forall x.(\neg rasiert(x,x) \leftrightarrow rasiert(barbier,x))
\end{equation*}
Wir zeigen: $F$ ist unerfüllbar. \\\\
Sei $\INT$ eine Interpretation. Dann ist zu zeigen, dass $\INT \not\models F$. 
\begin{align*}
\INT \models F &\Leftrightarrow \INT \models \forall x.(rasiert(barbier,x) \leftrightarrow \neg rasiert(x,x)) \\
&\Leftrightarrow \text{ Für alle } \delta_x \in \Delta^{\INT} \text{ gilt, dass } \INT, \{ x \mapsto \delta_x \} \models rasiert(barbier, x)  \leftrightarrow \neg rasiert(x,x)
\end{align*}
Für das Element $\delta_x$ mit $barbier^{\INT} = \delta_x$ gilt:
\begin{align*}
\INT, \{x \mapsto \delta_x\} \models rasiert(barbier,x) \leftrightarrow \neg rasiert(x,x) 
\Leftrightarrow \underbrace{(barbier^{\INT}, \delta_x}_{=\delta_x} \in rasiert^{\INT} \Leftrightarrow (\delta_x, \delta_x) \not\in rasiert^{\INT}
\end{align*}
Es ergibt sich der Widerspruch $(\delta_x, \delta_x) \in rasiert^{\INT} \Leftrightarrow (\delta_x, \delta_x) \not\in rasiert^{\INT}$ und damit ist $\INT$ kein Modell von $F$. Weil $\INT$ beliebig, folgt die Unerfüllbarkeit von $F$.

\newcommand{\HRule}[2]{\par
  \vspace*{\dimexpr-\parskip-\baselineskip+#2}
  \noindent\rule{#1}{0.2mm}\par
  \vspace*{\dimexpr-\parskip-.5\baselineskip+#2}}

\subsection*{Übung 10 (Skolemform)}
\subsubsection*{Aufgabe 1}
Bestimmen Sie zu jeder der folgenden Formeln eine äquivalente bereinigte Formel in Pränexform.
\begin{enumerate}
\item $\forall x.(p(x,x) \leftrightarrow \neg \exists y.q(x,y))$ \\
\LOES 
\begin{align*}
& \forall x.(p(x,x) \leftrightarrow \neg \exists y.q(x,y)) \\ 
\equiv\, & \forall x.(p(x,x) \to \neg \exists y.q(x,y)) \land (\neg \exists y.q(x,y) \to p(x,x))) \\
\equiv\, & \forall x.((\neg p(x,x) \lor \underbrace{\neg \exists y}_{\forall y.\neg}.q(x,y)) \land (\exists y'.q(x,y') \lor p(x,x))) \\
\equiv\, & \forall x,y.\exists y'.((\neg p(x,x) \lor \neg q(x,y)) \land (q(x,y') \lor p(x,x)))
\end{align*}
\item $\forall x.p(f(x,x)) \lor (q(x,z) \to \exists x.p(g(x,y,z)))$ \\
\LOES 
\begin{align*}
& \forall x.p(f(x,x)) \lor (q(x,z) \to \exists x.p(g(x,y,z))) \\
\equiv\, & \forall x'.(p(f(x',x')) \lor (\neg q(x,z) \lor \exists x''.p(g(x'',y,z)))) \\
\equiv\, & \forall x'.\exists x''.(p(f(x',x')) \lor \neg q(x,z) \lor p(g(x'',y,z)))
\end{align*}
\item $\forall x.p(x) \land (\forall y.\exists x.q(x,g(y)) \to \exists y.(r(f(y)) \lor \neg q(y,x)))$ \\
\LOES
\begin{align*}
& \forall x.p(x) \land (\forall y.\exists x.q(x,g(y)) \to \exists y.(r(f(y)) \lor \neg q(y,x))) \\
\equiv\, & \forall x''.p(x'') \land (\underbrace{\neg\forall y.\exists x'.q(x',g(y))}_{\equiv \exists y.\forall x'.\neg q(x',g(y))} \lor \, \exists y'.(r(f(y')) \lor \neg q(y',x)) \\
\equiv\, & \forall x''.\exists y.\forall x'.\exists y'.(p(x'') \land (q(x',g(y)) \lor r(f(y') \lor \neg q(y',x)))
\end{align*}
(Tipp für Skolemform: $\exists$-Quantoren möglichst nach links ziehen.)
\end{enumerate}
\subsubsection*{Aufgabe 2}
Bestimmen Sie zu jeder der folgenden Formeln eine erfüllbarkeitsäquivalente bereinigte Formel in Skolemform.
\begin{enumerate}
\item $p(x) \lor \exists x.q(x,x) \lor \forall x.p(f(x))$ \\
\LOES \begin{align*}
& p(x) \lor \exists x.q(x,x) \lor \forall x.p(f(x)) \\
\equiv\, & \exists u.\forall v.(p(x) \lor q(u,u) \lor p(f(v))) \\
\rightarrow_{skolemform}\, & \forall v.(p(x) \lor q(c,c) \lor p(f(v))) \text{,}
\end{align*}
wobei $c$ eine neue Konstante ist.
\item $\forall x. \exists y.q(f(x),g(y)) \land \forall x.(p(x,y,y) \lor q(h(y),x))$ \\
\LOES 
\begin{align*}
& \forall x. \exists y.q(f(x),g(y)) \land \forall x.(p(x,y,y) \lor q(h(y),x)) \\
\equiv\, & \forall x.\exists u.\forall v.(q(f(x),g(u)) \land (p(v,y,y) \lor q(h(y),v))) \\
\rightarrow_{skolemform}\, & \forall x,v.(q(f(x), g(l(x)) \land (p(v,y,y) \lor q(h(y),v))) \text{,}
\end{align*}
wobei $l$ ein neues ein-stelliges Funktionssymbol ist.
\item $\forall x. \forall x.(p(x) \leftrightarrow q(x,x)) \lor \exists x.\forall y.(q(x,g(y,z)) \land \exists z.q(z,z))$ \\
\LOES 
\begin{align*}
& \forall x. \forall x.(p(x) \leftrightarrow q(x,x)) \lor \exists x.\forall y.(q(x,g(y,z)) \land \exists z.q(z,z)) \\
\equiv\, & \forall x.\exists u.\forall v.\exists w(((p(x) \land q(x,x)) \lor (\neg p(x) \land \neg q(x,x) \lor (q(n,g(v,z)) \land q(w,w))) \\
\rightarrow_{skolemform}\, & \forall x,v.(((p(x) \land q(x,x)) \lor (\neg p(x) \land \neg q(x,x))) \lor (q(f(x), g(v,z)) \land q(h(x,v), h(x,v)))) \text{,}
\end{align*}
wobei $f$ und $h$ neue Funktionssymbole sind.
\end{enumerate}

\subsubsection*{Aufgabe 3}
Gegeben sind die folgenden Formeln in Skolemform. 
\begin{align*}
F &= \forall x,y,z.p(x,f(y),g(z,x)), \\
G &= \forall x,y.(p(a,f(a,x,y)) \lor q(b)),
\end{align*}
wobei $a$ und $b$ Konstanten sind.
\begin{enumerate}
\item Geben Sie die zugehörigen Herbrand-Universen $\Delta_F$ und $\Delta_G$ an. \\
\LOES 
\begin{align*}
\Delta_F &= \{a, f(a), g(a,a), f(f(a)), g(f(a),f(a)), \dots \} \\
\Delta_G &= \{a, b, f(a,a,a), f(a,a,b), f(a,b,a), f(b,a,a), \dots, f(f(a,a,a),f(b,a,b),f(b,a,a)), \dots \} \\
\text{Wir können } &\Delta_F \text{ und } \Delta_G \text{ auch rekursiv wie folgt charakterisieren:} \\
\Delta_F &= \{a\} \cup \{f(t),g(t,u) \mid t,u \in \Delta_F \} \\
\Delta_G &= \{a,b\} \cup \{f(s,t,u) \mid s,t,u \in \Delta_G\}
\end{align*}
\item Geben Sie je ein Herbrand-Modell an oder begründen Sie, warum kein solches existiert. \\
\LOES Für $F: a^{\INT} := a,\quad f^{\INT}(t) := f(t),\quad g^{\INT}(s,t) := g(s,t) \text{ mit } s,t \in \Delta_F$ \\
Definiere noch: $p^{\INT} := \Delta_F^3$, alternativ: $p^{\INT} := \{(r,f(s),g(t,r)) \mid r,s,t \in \Delta_F \}$. \\
Dann ist die Herbrand-Interpretation $(\Delta_F, \cdot^{\INT})$ ein Modell von $F$.
\HRule{3cm}{3mm}
Für $G: a^{\INT} := a,\quad b^{\INT} := b,\quad f^{\INT}(v,s,t) := f(v,s,t)$. \\\
Definiere noch: $p^{\INT} := \{(a,f(a,s,t)) \mid s,t \in \Delta_G \}$, $q^{\INT} := \{b\}$. \\
Dann ist $(\Delta_G, \cdot^{\INT})$ ein Herbrand Modell von $G$.
\item Geben Sie die Herbrand-Expansion $\HE(F)$ und $\HE(G)$ an. \\
\LOES 
\begin{align*}
\HE(F) &= \{p(a,f(a),g(a,a)), \dots \} \\
&= \{p(r,f(s),g(t,r)) \mid r,s,t \in \Delta_F \} \\
\HE(G) &= \{ p(a,f(a,f(a,a,a),b)) \lor q(b), \dots \} \\
&= \{ p(a,f(a,s,t)) \lor q(b) \mid s,t \in \Delta_G \}
\end{align*}
\end{enumerate}

\subsubsection*{Aufgabe 4}
Zeigen Sie, dass Allgemeingültigkeit von Formeln der Prädikatenlogik erster Stufe in Skolemform entscheidbar ist. \\
\LOES Es sei $F$ eine quantorenfreie Formel mit Variablen $x_1, \dots, x_n$. Dann gilt
\begin{align*}
\forall x_1, \dots, x_n. F ist allgemeingültig \Leftrightarrow & \exists x_1, \dots, x_n.\neg F \text{ ist unerfüllbar} \\
\Leftrightarrow & \neg F[x_1/a_1, \dots, x_n/a_n] \text{ ist unerfüllbar} \\
& \text{(Skolemisierung mit Konstanten } a_1, \dots, a_n \text{).}
\end{align*}
Es ist also $\forall x_1, \dots, x_n.F$ allgemeingültig genau dann, wenn $\neg F[x_1/a_1, \dots, x_n/a_n]$ unerfüllbar ist. \\
Letzteres ist aber essentiell eine aussagenlogische Formel, und deren Erfüllbarkeit ist entscheidbar.
\subsection*{Übung 11 (Allgemeinster Unifikator, Resolution)}
\subsubsection*{Aufgabe 1}
Bestimmen Sie jeweils einen allgemeinsten Unifikator der folgenden Gleichungsmengen, oder begründen Sie, warum kein allgemeinster Unifikator existiert. Verwenden Sie hierfür den Algorithmus aus der Vorlesung. Dabei sind $x,y$ Variablen und $a,b$ Konstanten.
\begin{enumerate}
\item $\{ f(x) \dot{=} g(x,y), y \dot{=} f(a) \}$ \\
\LOES Der Algorithmus besteht aus 4 Regeln: 
\begin{itemize}
\item Löschen: $t=t$
\item Orientieren: $t=x \mapsto x=t$
\item Zerlegen: $f(t_1,\dots,t_n) = f(s_1,\dots,s_n) \mapsto t_1=s_1,\dots, t_n=s_n$
\item Einsetzen (Eliminieren): $x=t$ in anderer Gleichung einsetzen, falls $x$ nicht vorkommt.
\end{itemize}
\begin{equation*}
\{ f(x) = g(x,y), y = f(a) \} \rightarrow_{einsetzen} \{ f(x) = g(x,f(a)), y = f(a) \}
\end{equation*}
Keine weitere Regel anwendbar und Menge nicht in gelöster Form. \\
Also gibt es keinen (allgemeinsten) Unifikator.
\item $\{ f(g(x,y)) \dot{=} f(g(a,h(b))) \}$ \\
\LOES 
\begin{equation*}
\{ f'(g(x,y)) = f'(g(a,h(b))) \} \rightarrow_{zerlegen} \{g(x,y) = g(a,h(b)) \rightarrow_{zerlegen} \{x=a, y=h(b)\}
\end{equation*}
Keine weitere Regel anwendbar und Menge in gelöster Form. \\
Ein allgemeinster Unifikator ist $\{x \mapsto a, y \mapsto h(b) \}$
\item $\{ f(x,y) \dot{=} x, y \dot{=} g(x) \}$ \\
\LOES
\begin{equation*}
\{ f(x,y) = x, y = g(x) \} \rightarrow_{orientieren} \{x=f(x,y), y=g(x)\} \rightarrow_{einsetzen} \{x = f(x,g(x)), y= g(x) \}
\end{equation*}
Keine weitere Regel anwendbar. \\
Menge nicht in gelöster Form.
\item $\{ f(g(x),y) \dot{=} f(g(x),a), g(x) \dot{=} g(h(a)) \}$ \\
\LOES
\begin{equation*}
\{ f(g(x),y) = f(g(x),a), g(x) = g(h(a)) \} \rightarrow_{zerlegen} \{ g(x) = g(x), y=a, g(x) = g(h(a)) \} \rightarrow_{zerlegen} \{ g(x) = g(x), y=a, x=h(a) \} \rightarrow_{löschen} \{y=a, x=h(a) \}
\end{equation*}
Fertig, in gelöster Form, ein allg. Unifikator ist $\{y \mapsto a, x \mapsto h(a) \}$.
\item Zusatz. $\{x\dot{=}a, x\dot{=}h(a)\}$  \\
\LOES
\begin{equation*}
\{x=a, x=h(a)\} \rightarrow_{einsetzen} \{h(a) = a, x=h(a) \}
\end{equation*}
Nicht in gelöster Form.
\item Zusatz. $\{x\dot{=}z, y\dot{=}h(z)\}$  \\
\LOES
In gelöster Form. Unifikator $\{ x \mapsto z, y \mapsto h(z) \}$
\end{enumerate}
\subsubsection*{Aufgabe 2}
Zeigen Sie mittels prädikatenlogischer Resolution folgende Aussagen: 
\begin{enumerate}
\item Die Aussage \glqq Der Professor ist glücklich, wenn alle seine Studenten Logik mögen\grqq \\
hat als Folgerung \glqq Der Professor ist glücklich, wenn er keine Studenten hat\grqq. \\
\LOES Wir verwenden folgendes Vokabular: $\{\text{glücklich}/1, \text{magLogik}/1, \text{student}/1, \text{prof}(Konstante)\}$ 
\begin{align*}
F_1 &= \forall x.(\text{student}(x) \to \text{magLogik}(x)) \to \text{glücklich}(prof) \\
F_2 &= \neg \exists x.\text{student}(x) \to \text{glücklich}(prof)
\end{align*} 
\underline{Ziel}: Zeige $F_1 \models F_2$. Zeige dazu, $\{F_1, \neg F_2\}$ ist unerfüllbar. \\
\underline{Normalformen}: 
\begin{align*}
F_1 &= \forall x.(\neg\text{student}(x) \lor \text{magLogik}(x) \to \text{glücklich}(prof) \\
&\equiv \neg \forall x.(\neg \text{student}(x) \lor \text{magLogik}(x) \lor \text{glücklich}(prof) \\
&\equiv \exists x.((\text{student}(x) \land \neg \text{magLogik}(x)) \lor \text{glücklich}(prof) \qquad \text{Pränexform} \\
&=_{Skolem} (\text{student}(c) \land \neg \text{magLogik}(c)) \lor \text{glücklich(prof)} \\ 
&\equiv (\text{student}(c) \lor \text{glücklich}(prof)) \land (\neg \text{magLogik}(c) \lor \text{glücklich}(prof)) \\
\neg F_2 &= \neg (\neg \exists x.\text{student}(x) \to \text{glücklich}(prof)) \\
&\equiv \neg (\exists x.\text{student}(x) \lor \text{glücklich}(prof)) \\
&\equiv \forall x.(\neg \text{student}(x) \land \neg \text{glücklich}(prof))
\end{align*}
Klauselmenge: 
\begin{align*}
\{&\{\text{student}(c), \text{glücklich}(prof)\}^{(1)}, \{\neg\text{magLogik}(c),\text{glücklich}(prof)\}^{(2)}, \\ 
&\{\neg\text{student}(x)\}^{(3)}, \{\neg\text{glücklich}(prof)\}^{(4)}\}
\end{align*}
Resolution:
\begin{align*}
(5) &= (1) + (4) \text{ ergibt } \{\text{student}(c)\} \\
(6) &= (3) + (5) \text{ mit Unifikator } \{x \mapsto c\} \text{ ergibt } \bot
\end{align*}
Also ist $\{F_1, \neg F_2\}$ unerfüllbar und es gilt $F_1 \models F_2$
\item Die Formulierung des Barbier-Paradoxons aus Aufgabe 4 von Blatt 9 [\ref{U9-4}] ist unerfüllbar. \\
\LOES 
\begin{align*}
F &= \forall x.(rasiert(barbier, x) \leftrightarrow \neg rasiert(x,x) \\
&= \forall x.((rasiert(barbier,x) \lor rasiert(x,x)) \land (\neg rasiert(barbier,x) \lor \neg rasiert(x,x)))
\end{align*}
Klauseln:
\begin{equation*}
\{\{rasiert(barbier,x),rasiert(x,x)\}^{(1)}, \{\neg rasiert(barbier, x), \neg rasiert(x,x)\}^{(2)}\}
\end{equation*}
Resolution: $(1) + (2)$ mit Unifikator $\{x \mapsto barbier\}$ ergibt $\bot$
\item In Aufgabe V [\ref{REP3-V}]folgt die letzte Aussage aus den ersten drei. (Zur Vereinfachung darf hier angenommen werden, dass alle Individuen Drachen sind.) \\
\LOES
\begin{align*}
F_1 &:= \forall x.(\forall y.(\text{kind}(x,y) \to \text{fliegen}(y)) \to \text{glücklich}(x)) \\
F_2 &:= \forall x.(\text{grün}(x) \to \text{fliegen}(x)) \\
F_3 &:= \forall x.(\exists y(\text{kind}(y, x) \land \text{grün}(y)) \to \text{grün}(x)) \\
F_4 &:= \forall x.(\text{grün}(x) \to \text{glücklich}(x))
\end{align*}
Klauselmenge (nach umformulieren):
\begin{align*}
\{&\{\text{kind}(x,f(x)), \text{glücklich}(x)\}^{(1)},\{\neg\text{fliegen}(f(x)), \text{glücklich}(x)\}^{(2)}, \\
&\{\neg\text{grün}(x), \text{fliegen}(x)\}^{(3)}, \{\neg\text{kind}(y,x), \neg\text{grün}(y), \text{grün}(x) \}^{(4)}, \\
&\{\text{grün}(c)\}^{(5)}, \{\neg\text{glücklich}(c)\}^{(6)}\}
\end{align*}
Resolution: 
\begin{align*}
(7) = (1)+(4) & \text{ Variante von } (4): \{\neg \text{kind}(z,w), \neg\text{grün}(z),\text{grün}(w)\} \\
& \text{ Unifikator } \{z \mapsto x, w \mapsto f(x)\} \text{ ergibt Resolvente } \{\text{glücklich}(x),\neg\text{grün}(x),\text{grün}(f(x))\} \\
(8) = (5)+(7) &,\{x \mapsto c\} \text{ ergibt } \{\text{glücklich}(c),\text{grün}(f(c))\} \\
(9) = (8)+(6) & \text{ ergibt } \{\text{grün}(f(c))\} \\
(10) = (3)+(9) & \text{ mit } \{x \mapsto f(c)\} \text{ ergibt } \{\text{fliegen}(f(c))\} \\
(11) = (10)+(2) & \text{ mit } \{x \mapsto c\} \text{ ergibt } \{\text{glücklich}(c)\} \\
(12) = (11)+(6) & \text{ ergibt } \bot
\end{align*}
Also gilt $\{F_1, F_2, F_3\} \models F_4$
\end{enumerate}

\subsection*{Übungen 9-11 (Prädikatenlogik)}
    \textit{Die Übungen 9, 10 und 11 sind dem Themenkomplex der Prädikatenlogik gewidmet.}


    \section{Repetitorien}

\subsection*{Repetitorium I}
\subsubsection*{Aufgabe A}
    Wiederholung von Begriffen Einband Turing-Maschine, Mehrband Turing-Maschine, Entscheidungsproblem, Unentscheidbarkeit, Aufzählbarkeit, Abzählbarkeit und Halteproblem.

\subsubsection*{Aufgabe B}
    Zeigen Sie: Wenn es möglich ist, für zwei beliebige Turing-Maschinen zu entscheiden, ob sie dieselbe Sprache akzeptieren, so ist es auch möglich, für beliebige Turing-Maschinen zu entscheiden, ob sie die leere Sprache akzeptieren. Seien $K, M_1, M_2$ Turingmaschinen, so dass $K(enc(M_1)\#\#enc(M_2))$ akzeptiert $\Leftrightarrow$ $L(M_1) = L(M_2)$ und $K$ hält auf jeder Eingabe. \\
    \textbf{Lösung:} Sei $M$ Turingmaschine und sei $M_\emptyset$ eine Turingmaschine, so dass $L(M_\emptyset) = \emptyset$. \\
    Dann gilt $K(enc(M)\#\#enc(M_\emptyset))$ akzeptiert $\Leftrightarrow$ $L(M) = \emptyset$, also $\prspec{leer} \leq_{m} \prspec{"aquiv}$.

\subsubsection*{Aufgabe C}
    Zeigen Sie, dass $\{1\}^*$ unentscheidbare Teilmengen besitzt. \\
    \textbf{Lösung:} $\{1\}^*$ ist abzählbar unendlich, also ist $\mathfrak{P}(\{1\}^*)$ überabzählbar. Es gibt aber nur abzählbar unendlich viele entscheidbare Sprachen (auch: abzählbar viele nicht-äquivalente Turingmaschinen). Also sind einige (fast alle) dieser Sprachen unentscheidbar.

\subsubsection*{Aufgabe D}
    \begin{enumerate}
        \item „Jedes LOOP-Programm terminiert.“ – Richtig. Definition von LOOP sagt, dass Anzahl Durchläufe nicht mehr während der Laufzeit geändert werden kann, demnach gibt es eine endliche Anzahl Durchläufe.

        \item „Zu jedem WHILE-Programm gibt es ein äquivalentes LOOP-Programm.“ – Falsch, nicht zu jedem WHILE-Programm gibt es immer ein äquivalentes LOOP-Programm. Dies liegt daran, dass LOOP keine partiellen Funktionen verarbeiten kann. \\
        Beispiel anhand von Division: LOOP terminiert immer, jedoch wäre Division durch $0$ (ebenfalls in $\mathbb{N}$) undefiniert. Kann demnach nur mit WHILE gelöst werden (Fall $x_2 = 0$ für $div(x_1, x_2)$ landet in Endlosschleife).

        \item „Die Anzahl der Ausführungen von $P$ in der LOOP-Schleife LOOP $x_i$ DO $P$ END kann beeinflusst werden, indem $x_i$ in $P$ entsprechend modifiziert wird.“ – Falsch, Anzahl Schleifen kann laut Definition von LOOP nicht während Laufzeit geändert werden.

        \item „Die Ackermannfunktion ist total und damit LOOP-berechenbar.“ – Falsch, die Ackermannfunktion ist zwar total, jedoch nicht LOOP-berechenbar (jedoch berechenbar). Die Funktion wurde gezielt gesucht und gefunden, um genau diesen Fall zu zeigen.
    \end{enumerate}

\subsubsection*{Aufgabe E}
    Geben Sie eine Turing-Maschine $A_{mod2}$ an, die die Funktion  $f: \mathbb{N} \to \mathbb{N}$ mit $f(x) = (x~mod~ 2)$ berechnet. Stellen Sie dabei die Zahlen in unärer Kodierung dar. \\
    \textbf{Lösung:} $A_{mod2} = (\{q_0,…,q_f\}, \{x\}, \{x, \blank\}, \delta, q_0, \{q_f\})$. Die Turingmaschine liest die eingegebenen $x$ (unäre Kodierung) und wechselt zwischen $q_0$ und $q_1$. Sobald auf ein $\blank$ gestoßen wird, weiß die TM, ob eine gerade oder ungerade Anzahl $x$ eingegeben wurde. Wenn gerade, lösche alle $x$ und ende in leerem Band. Wenn ungerade, lösche alle $x$ und schreibe zum Abschluss ein $x$ auf das Band.

\subsubsection*{Aufgabe F}
    Es sei $f:\N\to\N$ mit $f(x)=\lfloor\log_{10}(x)\rfloor$. Geben Sie ein WHILE-Programm an, welches $f$ berechnet. \\
    \textbf{Lösung:} Erst eine endlose WHILE-Schleife für die Eingabe $x=0$. Dann Lösung mit $div(x, 10)$.

\subsubsection*{Aufgabe G}
    \begin{enumerate}
        \item „Die Menge der Instanzen des Postschen Korrespondenzproblems, welche eine Lösung haben, ist semi-entscheidbar.“ – Richtig. Wenn eine Lösung existiert, kann diese (z.B. durch Breitensuche) auch gefunden werden.

        \item „Das Postsche Korrespondenzproblem ist bereits über dem Alphabet $\Sigma = \{a,b\}$ nicht entscheidbar.“ – Richtig, denn Instanzen können über $\Sigma$ kodiert werden, ohne die Entscheidbarkeit zu beeinflussen.

        \item „Es ist entscheidbar, ob eine Turingmaschine nur Wörter akzeptiert, die Palindrome sind.“ – Falsch, Satz von Rice (die Akzeptanz von Palindromen ist eine \f{Eigenschaft}). Eigenschaft $E$ ist „$L$ besteht nur aus Palindromen“, diese Eigenschaft ist nicht-trivial: erfüllt z.B. durch $L = \emptyset$, jedoch nicht durch $L = \{ab\}$.

        \item „$\prspec{halt}$ ist semi-entscheidbar“ – Richtig, da es universelle Turingmaschinen gibt, die beliebige TM simulieren können.

        \item „Es ist nicht entscheidbar, ob die von einer deterministischen Turing-Maschine berechnete Funktion total ist.“ – Richtig, denn sonst wäre das Halteproblem entscheidbar ($\prspec{halt} \leq_{m} \prspec{total}$). \\
        Reduktion: $M, w$ gegeben, baue Turingmaschine $M'$ mit \\
        $M' =$ bei Eingabe $x$ \\
        $\to$ simuliere $M$ auf $w$ \\
        $\to$ akzeptiere mit leerem Band \\
        M hält auf $w \Rightarrow M'$ berechnet $f(x)= \epsilon$ \\
        M hält nicht auf $w \Rightarrow M'$ berechnet Abbildung, die nirgends definiert ist. \\
        Reduktion demnach $enc(M)\#\#enc(w) \mapsto enc(M')$.

        \item „Es gibt reguläre Sprachen, die nicht semi-entscheidbar sind.“ – Falsch. Reguläre Sprachen sind immer entscheidbar, da Turingmaschinen endliche Automaten simulieren können.
    \end{enumerate}

\subsubsection*{Aufgabe H}
    Sei $L$ eine unentscheidbare Sprache. Zeigen Sie:

    \begin{enumerate}
        \item „$L$ hat eine Teilmenge $T \subseteq L$, die entscheidbar ist.“: $T = \emptyset$.
        \item „$L$ hat eine Obermenge $O \supseteq L$, die entscheidbar ist.“: $O = \Sigma^*$.
        \item „Es gibt jeweils nicht nur eine sondern unendlich viele entscheidbare Teilmengen bzw. Obermengen wie in (a) und (b).“ – Es gilt: $L$ ist unendlich. Dann ist die Menge der endlichen Teilmengen von $L$ unendlich. Alles diese sind entscheidbar. Genauso für \textbf{b)}, z.B. muss $\Sigma^*\setminus L$ unendlich sein. Die Menge der endlichen Teilmengen $E$ von $\Sigma^*\setminus L$ ist unendlich, für jede ist $\Sigma^*\setminus E$ entscheidbar.
    \end{enumerate}

    \subsection*{Repetitorium II (02.06.2017)}
\subsubsection*{Aufgabe $\alpha$}
    \begin{enumerate}
        \item $P$: Menge aller Entscheidungsprobleme, die von einer deterministischen TM in polynomieller Zeit entschieden werden können. (entschieden: der Algorithmus hält immer)
        \item $NP$: Wie $P$, nur mit NTM. Menge aller Entscheidungsprobleme, so dass für Instanzen ein Zertifikat in polynomieller Zeit geraten und überprüft werden kann. (NP ist Menge aller Suchprobleme, bei denen ich weiß wann ich angekommen bin)
        \item $PSpace$: Menge aller Entscheidungsprobleme, die von einer DTM in polynomiellem Platz entschieden werden können.
        \item $P \subseteq NP \subseteq PSpace$: DTM „sind“ auch NTM $\to P \subseteq NP$. NTP, die polynomiell-zeitbeschränkt sind, können in deterministisch polynomiellem Platz simuliert werden $\to NP \subseteq PSpace$. (Auch anhand der Anzahl möglicher Lesevorgänge begründbar). Außerdem Satz von Savitch: $NP \subseteq NPSpace \subseteq PSpace$.
        \item $\mathcal{C}$-hart: ein Entscheidungsproblem ist $\mathcal{C}$-hart, wenn alle Probleme in in $\mathcal{C}$ in polynomieller Zeit auf dieses reduzierbar sind. Es ist $\mathcal{C}$-vollständig, wenn es $\mathcal{C}$-hart ist und selbst in $\mathcal{C}$ liegt. Am Beispiel von SAT sehen wir, dass SAT $\mathcal{C}$-vollständig ist, da es selbst in $NP$ liegt und kein Problem in $NP$ schwerer ist (und somit alle auf SAT reduzierbar sind) als SAT. Es kann vorkommen, dass mehrere Probleme $\mathcal{C}$-vollständig sind, wenn diese in polynomiell äquivalenter Zeit lösbar sind.
    \end{enumerate}

\subsubsection*{Aufgabe $\beta$}
    Zeigen, dass $NP$ unter Kleene-Stern abgeschlossen. $\forall L \in \mathit{NP}: L^{*} \in \mathit{NP}$ \\
    Sei $L \in \mathit{NP}$ und sei $M$ eine polynomiell-zeitbeschränkte TM, so dass $L = L(M)$. \\
    Definiere $N = $ bei Eingabe $\omega$
    \begin{itemize}
        \item rate Zerlegung $\omega = \omega_{1},\dots,\omega_{n}$ (beim leeren Wort: $n=0$) (nicht-deterministisch)
        \item simuliere $M$ auf $\omega_{i}$ für $i = 1,\dots,n$ (nicht-deterministisch)
        \item akzeptiere, falls alle Simulationen akzeptieren
    \end{itemize}
    $N$ ist polynomiell-zeitbeschränkt und $L(N) = L^{*}$


\subsubsection*{Aufgabe $\gamma$}
    Aufgabe mit Problem \prob{K}: zwei gerichtete Graphen $G_{1}$ und $G_{2}$ sowie eine Zahl $k \in \N$. Gesucht: Teilmengen und Bijektion.
    \begin{enumerate}
        \item $K \in \mathit{NP}$ da Teilmengen $V^{'}_{1}$ und $V^{'}_{2}$ und die Zuordnung $f$ geraten werden kann und in polynomieller Zeit überprüfbar ist ob $f: V_{1} \to V_{2}$ eine Bijektion ist, so dass $(u,v) \in E_{1} \implies (f(u), f(v)) \in E_{2}$.

        \item Sei $G$ ein Graph und $n \in \N$. Gefragt ist dann, ob $G$ eine \prob{CLIQUE} der Größe $n$ als Untergraph enthält.\\
        Sei $f(enc(G)\#\#enc(n)) := enc(G)\#\#enc(K_{n})\#\#enc(n)$ wobei $K_{n}$ der vollständige Graph auf $n$ Knoten ist. Dann gilt: $f$ ist polynomiell-zeitbeschränkt und $G$ hat \prob{CLIQUE} der Größe \\ $n \Longleftrightarrow enc(G)\#\#enc(K_{n})\#\#enc(n) \in K$. Also ist $f$ eine polynomiell-zeitbeschränkte Many-One-Reduktion von \prob{CLIQUE} auf $K$ und damit ist $K$ auch $NP$-hart. \\
        Liste bekannter Probleme: SAT/3SAT/CNFSAT, CLIQUE/IndepententSet/HamiltonCircle, 3-Färbbarkeit
    \end{enumerate}

\newpage
\subsubsection*{Aufgabe $\delta$}
    \begin{enumerate}
        \item Entscheider für $L_{1}$: $N =$ bei Eingabe $\omega$
            \begin{itemize}
                \item berechne Reduktion $f(\omega)$ (polynomielle Zeit)
                \item entscheide, ob $f(\omega) \in L_{2}$
            \end{itemize}
        $N$ ist polynomiell-platzbeschränkt, da:
        \begin{itemize}
            \item $f$ polynomiell-zeitbeschränkt
            \item $f(\omega) \in L_{2}$ kann in polynomiellem Platz entschieden werden.
        \end{itemize}
        $N$ ist auch Entscheider, da es einen polynomiell-platzbeschränkten Entscheider für $L_{2}$ gibt. Also ist $L_{1} \in \mathit{PSpace}$.

        \item Sei $L \in \mathit{PSpace}$. Dann ist $L \leq_{p} L_{1} \leq_{p} L_{2}$ also $L \leq_{p} L_{2}$ (transitiv). Also ist jedes Problem in PSpace auf $L_{2}$ in polynomieller Zeit reduzierbar und $L_{2}$ damit PSpace-hart.
    \end{enumerate}


\subsubsection*{Aufgabe $\epsilon$}
    \begin{enumerate}
        \item „Jedes PSpace-harte Problem ist NP-hart“ – Richtig, da $\mathit{NP} \subseteq \mathit{PSpace}$.
        \item „Es gibt kein NP-hartes Problem, welches in PSpace liegt“ – Falsch, z.B. gilt $\mathit{SAT} \in \mathit{PSpace}$ und SAT ist NP-hart.
        \item „Jedes NP-vollständige Problem liegt in PSpace“ – Richtig, da $\mathit{NP} \subseteq \mathit{PSpace}$ und alle NP-vollständigen Probleme liegen in NP.

        \item „Es gilt $\mathit{NP} = \mathit{PSpace}$, wenn es ein PSpace-hartes Problem in NP gibt“ – Richtig, $\mathit{NP} \subseteq \mathit{PSpace}$ ist bekannt. Sei $L$ ein PSpace-hartes Problem in NP. Sei $L^{'} \in \mathit{PSpace}$. Dann gilt $L^{'} \leq_{p} L$ und da NP unter polynomieller Zeitreduktion abgeschlossen ist, folgt $L^{'} \in \mathit{NP}$. Also gilt $\mathit{PSpace} \subseteq \mathit{NP}$ und damit auch $\mathit{NP} = \mathit{PSpace}$.

        \item „Wenn $P \neq NP$ gilt, dann gibt es kein NP-hartes Problem in P“ – Richtig, sonst wäre $P = NP$.

        \item „Sei L ein PSpace-vollständiges Problem. Dann gilt $L \in P \Longleftrightarrow P = PSpace$“ – Richtig. % TODO Lösung fehlt
    \end{enumerate}

\subsubsection*{Aufgabe $\zeta$}
    Tic-Tac-Toe-Spiel. Die Beschreibung einer Gewinnstrategie erfolgt mit Hilfe eines Baumes, auf dem die möglichen Abläufe skizziert werden. Alle Möglichen Spielzüge von X und O führen zum Sieg von X.

\subsubsection*{Aufgabe $\eta$}
    „Zeigen Sie, dass für jedes PSpace-vollständige Problem \prob{L} auch das Komplement $\overline{\prob{L}}$ ein PSpace-vollständiges Problem ist.“
    $\prob{L} \in \mathit{PSpace} \to \overline{\prob{L}} \in \mathit{PSpace}$. $\overline{\prob{L}}$ ist PSpace-hart:

    \hspace{.52cm} $\prob{H} \in PSpace \Longrightarrow \overline{\prob{H}} \in PSpace$

    \hspace{2.55cm} $\Longrightarrow \overline{\prob{H}} \leq_{p} L$

    \hspace{2.55cm} $\Longrightarrow \prob{H} \leq_{p} \overline{\prob{L}}$

    Also ist $\overline{\prob{L}}$ PSpace-vollständig.


\subsubsection*{Aufgabe $\theta$}
    „Zeigen Sie: ist $P = NP$, dann sind alle Sprachen $\prob{L} \in P\setminus\{\emptyset, \Sigma^{*}\}$ NP-vollständig.“ \\
    Sei $\prob{L} \in P\setminus\{\emptyset, \Sigma^{*}\}$. Sei $K \in NP$. Wir zeigen, dass $\prob{K} \leq_{p} \prob{L}$, unter der Annahme, dass $P=NP$.\\

    Seien $x_{1} \in \prob{L}, x_{2} \in \Sigma^{*}\setminus\prob{L}$.
    Definiere $
        f(\omega) =
        \begin{cases}
            x_{1}\ \mathit{falls}\ \omega \in \prob{K}\\
            x_{2}\ \mathit{sonst}
        \end{cases}
    $ \\

    Da $\prob{K} \in P$ ist die Abbildung $f$ in polynomieller Zeit berechenbar. Es gilt $\omega \in \prob{K} \Longleftrightarrow f(\omega) \in \prob{L}$. Also ist $f$ eine polynomiell-zeitbeschränkte Many-One-Reduktion von \prob{K} auf \prob{L}. Also ist \prob{L} auch NP-vollständig.


\end{document}
