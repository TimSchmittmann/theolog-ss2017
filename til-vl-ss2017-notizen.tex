\documentclass[a4paper,10pt]{article}

\usepackage[utf8]{inputenc}
\usepackage[ngerman]{babel}
\usepackage[top=2.5cm,bottom=2.5cm,left=2.5cm,right=2.5cm]{geometry}
\usepackage[T1]{fontenc}
\usepackage{graphicx}
\usepackage{color}
\usepackage{fancyhdr}
\usepackage{tgpagella}
\usepackage{marginnote}
\usepackage{nameref}
\usepackage{amsmath,amssymb,amsfonts}

\usepackage[
    pdfauthor={Dominik Pataky},
    pdftitle={Theoretische Informatik und Logik},
    pdfsubject={Notizen und Begriffsklärungen aus dem Sommersemester 2017, TU Dresden},
    colorlinks=true,linkcolor=black,urlcolor=link]{hyperref}

\definecolor{light-gray}{gray}{0.9}
\definecolor{vl}{RGB}{84,200,70}
\definecolor{link}{RGB}{84,100,220}

\setlength\parindent{0cm} % indentation of paragraphs

\newcommand{\vl}[1]{\colorbox{vl}{\textcolor{white}{\small\textbf{#1}}}}
\newcommand{\f}[1]{\textbf{#1}}
\newcommand{\blank}{\text{\textvisiblespace}}
\newcommand{\hili}[1]{\colorbox{light-gray}{\textcolor{black}{#1}}}
\newcommand{\verweis}[1]{\textit{\ref{#1} \nameref{#1}}}
\newcommand{\prob}[1]{\textbf{#1}}
\newcommand{\phalt}{$~\prob{P}_{Halt}~$}


\begin{document}

    % set pagestyle to fancy and set header chapter to lowercase
    \pagestyle{fancy}

    % Header and footer styles
    \lhead{\slshape Theoretische Informatik und Logik SS2017}
    \rhead{\slshape\nouppercase{\rightmark}}
    \cfoot{\thepage}
    \lfoot{\footnotesize Dozent: Prof. Krötzsch, Autor: Pataky}
    \rfoot{}

    \begin{titlepage}
        \centering

        {\scshape\Large Notizen zur Vorlesung \par}
        {\scshape\LARGE Theoretische Informatik und Logik \par}
        \vspace{1cm}

        \begin{abstract}
            Notizen zur Vorlesung \url{https://iccl.inf.tu-dresden.de/web/Theoretische_Informatik_und_Logik_(SS2017)}. \par
            Ich versuche möglichst ohne formelle Symbole und Definitionen zu arbeiten, daher verweisen die Markierungen jeweils auf die Vorlesungsnummer in \vl{FS} bzw. \vl{TIL}.
            Obwohl der Schwerpunkt auf TheoLog liegt, habe ich ein paar Definitionen aus Formale Systeme mit einbezogen, da TheoLog diese weiterverwendet. \\
            Einige Formulierungen habe ich aus den hervorragenden Folien von Prof. Krötzsch geliehen. Quellen dieser Folien sind auf Github zu finden unter \url{https://github.com/mkroetzsch} und sind unter der Lizenz CC BY 3.0 DE verwendbar. Für diese gilt: „(C) Markus Krötzsch, \url{https://iccl.inf.tu-dresden.de/web/TheoLog2017}, CC BY 3.0 DE“.
        \end{abstract}

        \tableofcontents

        \vfill
        \begin{tabular}{p{3cm} p{10cm}}
            Autor & Dominik Pataky \\
            Dozent & Prof. Markus Krötzsch \\
            Ort & Fakultät Informatik, TU Dresden \\
            Zeit & Sommersemester 2017 \\
            Letztes Update & \today \\
            Lizenz & CC BY-SA 4.0
        \end{tabular}

    \end{titlepage}

    %\section{Motivation}
    % Hier Motivationstext über theoretische Informatik einfügen
    % Hilbert wollte Anfang des 20. Jahrhunderts die Mathematik auf eine formelle Basis stellen.

    %\item[Logik] Eine kurze Motivationsübersicht zur Einleitung.
    %            \begin{itemize}
    %                \item „Wer rechnende Systeme verstehen und konstruieren will, der benötigt passende mathematische Modelle. Dieser Weg führt oftmals zur Logik.“ Ziel: Verifikation
    %                \item Logik ist die Wissenschaft vom folgerichtigen Denken. Ziel: Künstliche Intelligenz
    %                \item Logisches Schließen ist Problemlösen
    %                \item Logiken sind kodierte Beschreibungssprachen. Ziel: Wissensrepräsentation und Datenbanken
    %                \item „Logik“ ist ein allgemeiner Oberbegriff für viele mathematische und technische Formalismen
    %            \end{itemize}

    \section{Formale Systeme}
    \subsection{Sprachen und Automaten}
    \label{subsec:fs-sprachen-automaten}

    \begin{description}
        \item[(formale) Sprache] Menge von Wörtern/Symbolen/Tokens, z.B. Programmiercode oder natürliche Sprache.
            Zusätzliche Begriffe: Konkatenation, Präfix/Suffix/Infix, leeres Wort \vl{FS 1}
            \begin{description}
                \item[Symbol] Token der Sprache, z.B. if/else, +/-, True/False, ''Hello World''-String
                \item[Alphabet] nichtleere, endliche Menge von Symbolen
                \item[Wort] endliche Sequenz von Symbolen
                \item[Grammatik] formelle Spezifikation einer Sprache. Aus einer Grammatik kann man widerum eine Sprache erzeugen \vl{FS 2}
                \item[Rechenoperationen] Vereinigung, Schnitt, Komplement, Produkt, Potenz, Kleene-Abschluss
                \item[Abschlusseigenschaft] Beispiel: Wenn Sprache A und Sprache B regulär sind, wäre dann auch der Schnitt der beiden Sprachen wieder regulär? \vl{FS 5}
            \end{description}

        \item[Automat]
            Beginnt von einem Startzustand und folgt je nach Eingabe seinen Übergängen in die jeweiligen Zustände.
            Akzeptiert, wenn letzter Zustand ein akzeptierter Endzustand.
            \begin{description}
                \item[Deterministischer endlicher Automat (DFA)] erkennen reguläre Sprachen \vl{FS 3}
                \item[nichtdeterministischer endlicher Automat (NFA)] „rät“ die richtigen Übergänge, arbeitet parallel. Nichtdeterminismus sinnvoll? Kompaktere Darstellungen, Start für Entwicklung DFA, kann bei Untersuchung Komplexität/Berechenbarkeit helfen \vl{FS 4}
                \item[Kellerautomat (PDA)]
                    PDA erweitern endliche Automaten um einen unbeschränkt großen Speicher, der aber nur nach dem LIFO-Prinzip verwendet werden
                    kann. PDAs erkennen genau die kontextfreien Sprachen. \vl{FS 15}
                \item[Turingmaschine (TM)]
                    liefert allgemeines Modell der Berechnung. Liest und schreibt in einem Schritt, hat unendlichen
                    Speicher, kann beliebig auf Speicher zugreifen (im Gegensatz zu LIFO bei PDA).
                    Kann ein Band oder mehrere Bänder haben.
                    Kann deterministisch (DTM) oder nichtdeterministisch (NTM) sein.
                    Alle Varianten der TM können die selben Funktionen berechnen - einzig der Aufwand ist unterschiedlich (NTM kann DTM darstellen, NTM kann durch DTM simuliert werden etc.).
                    Siehe auch Church-Turing-These. \vl{FS 18}
            \end{description}


        \item[Kardinalität] Unterscheidung abzählbar (mit natürlichen Zahlen) und überabzählbar

        \item[Chomsky-Hierarchie]
            Kategorische Einteilung von Sprachen je nach Komplexität ihrer Grammatik. \\
            Hierarchie 0 > 1 > 2 > 3.
            These: „Die meisten Sprachen können nicht mit Grammatiken beschrieben
            werden (abzählbar viele Grammatiken vs. überabzählbar viele Sprachen)“. \vl{FS 2}
            \begin{description}
                \item[Typ 0] beliebige Grammatiken (Turingmaschinen)
                \item[Typ 1] kontextsensitive Grammatiken
                \item[Typ 2] kontextfreie Grammatiken (CYK, Kellerautomaten)
                \item[Typ 3] reguläre Grammatiken (DFA, NFA, Pumping Lemma)
            \end{description}

        \item[Probleme] Probleme formulieren Berechnungsfragen.
            \begin{description}
                \item[Wortproblem] Wortproblem für eine Sprache über einem Alphabet ist die Berechnung der Ausgabe „ja, Wort ist in Sprache“ oder „nein, Wort ist nicht in Sprache“, für die Eingabe eines Wortes gebildet aus dem Alphabet \vl{FS 3}
                \item[Leerheitsproblem (DFA, NFA)] Entscheidung für „ja, Automat erzeugt Sprache“ oder „nein, durch den Automaten erzeugte Sprache ist leer“ (es wird nie ein Endzustand erreicht). \vl{FS 10}
                \item[Inklusionsproblem (DFA, NFA)] Entscheidung für „ja, Sprache A eines Automaten ist eine Teilmenge der Sprache B eines anderen Automaten“ oder „nein, Sprache A ist keine Teilmenge der Sprache B“. \vl{FS 10}
                \item[Äquivalenzproblem (DFA, NFA)] Entscheidung für „ja, Sprache A eines Automaten ist gleich der Sprache B eines anderen Automaten“ oder „nein, Sprache A unterscheidet sich von Sprache B“. \vl{FS 10}
                \item[Endlichkeitsproblem (DFA, NFA)] Entscheidung für „ja, erzeugte Sprache eines Automaten ist endlich“ oder „nein, erzeugte Sprache ist nicht endlich“ (z.B., wenn Zyklen auf dem Pfad von Start- zu Endzustand existieren). \vl{FS 10}
                \item[Universalitätsproblem (DFA, NFA)] Entscheidung für „ja, erzeugte Sprache eines Automaten ist $\Sigma^*$“ oder „nein, erzeugte Sprache ist nicht $\Sigma^*$“ (heißt, Komplement der erzeugten Sprache ist leer). \vl{FS 10}
                \item[Halteproblem (TM)] Entscheidet, ob eine Turingmaschine für eine Eingabe jemals hält oder nicht. Unentscheidbar. \vl{FS 19}
            \end{description}

        \item[Church-Turing-These]
            Die Turingmaschine kann alle Funktionen berechnen, die intuitiv berechenbar sind. Auch: „Eine Funktion ist genau dann
            im intuitiven Sinne berechenbar, wenn es eine Turingmaschine gibt, die für jede mögliche Eingabe den Wert der Funktion auf
            das Band schreibt und anschließend hält.“ \vl{FS 18}

        \item[Entscheidbarkeit]
            Eine Sprache L ist entscheidbar / berechenbar / rekursiv, wenn es eine Turingmaschine gibt, die das Wortproblem der Sprache L entscheidet.
            D.h. die Turingmaschine ist Entscheider und die Sprache L ist gleich der durch die Turingmaschine erzeugten Sprache.
            Andernfalls heißt die Sprache L unentscheidbar. \\
            Die Sprache L ist semi-entscheidbar / Turing-erkennbar / rekursiv aufzählbar, wenn es eine Turingmaschine gibt, deren erzeugte Sprache zwar L ist, jedoch die Turingmaschine kein Entscheider ist. \vl{FS 19}

        \item[Satz von Rice]
            %Informell: „Jede nicht-triviale Frage über die von einer TM ausgeführte Berechnung ist unentscheidbar.“ \\
            Sei E eine Eigenschaft von Sprachen, die für manche Turing-erkennbare Sprachen gilt und für manche Turing-erkennbare Sprachen nicht gilt (= „nicht-triviale Eigenschaft“). \\ Dann ist das folgende Problem unentscheidbar: die Eingabe besteht aus einer Turingmaschine. Anhand der Ausgabe wollen wir prüfen, ob die durch diese Turingmaschine erzeugte Sprache die Eigenschaft E besitzt. \vl{FS 20} \\
            Der Beweis für die Unentscheidbarkeit dieses Problems ist eine Reduktion auf das Halteproblem.

    \end{description}

    \newpage
    \subsection{Aussagenlogik}
        Die Aussagenlogik untersucht logische Verknüpfungen von atomaren Aussagen. \vl{FS 21}

        \begin{description}
            \item[Atomare Aussage] Behauptungen, die wahr oder falsch sein können. \\ Auch: aussagenlogische Variablen, Propositionen, Atome
            \item[Operatoren, Junktoren] Verknüpfung von atomaren Aussagen. \\ Negation, Konjunktion, Disjunktion, Implikation, Äquivalenz. \\
                Können auch äquivalent durch andere Junktoren ausgedrückt werden (de Morgan). \vl{FS 22} \\
                Eine Disjunktion von Literalen nennt man \f{Klausel}. \\
                Eine Konjunktion von Literalen nennt man \f{Monom}.

            \item[Formel] Jedes Atom ist eine Formel, jede durch Junktoren verknüpfte Formeln sind wieder Formeln.
                Diese zusammengesetzten Formeln bestehen dann wieder aus Teilformeln (auch: Unterformeln, $Sub(F)$).
                Eine Formel, die nur aus einem Atom besteht, nennt man auch \f{Literal}.
                \begin{description}
                    \item[unerfüllbar] Formel hat keine Modelle
                    \item[erfüllbar] Formel hat mindestens ein Modell
                    \item[allgemeingültig] alle Interpretationen sind Modelle für Formel. Auch: \f{Tautologie}, $\models F$
                    \item[widerlegbar] Formel ist nicht allgemeingültig
                \end{description}

            \item[Syntax] Sprache einer Logik (Formeln mit logischen Operatoren). Wichtig: Klammerung.
            \item[Semantik] Definition der Bedeutung. Wertzuweisung von Wahrheitswerten zu Atomen mit Hilfe der Interpretation.
                „Die Bedeutung einer Formel besteht darin, dass sie uns Informationen darüber liefert, welche Wertzuweisungen möglich sind, wenn die Formel wahr sein soll.“
                \begin{description}
                    \item[Interpretation] eine Funktion w, die von einer Menge Atome auf die Menge \{0, 1\} abbildet.
                    \item[Wahrheitstabelle] Schrittweise Auflösung einer Formel durch Lösen ihrer Teilformeln.
                    \item[Modell] eine Interpretation, dessen Abbildung eine Formel nach 1 löst.
                    \item[Logische Konsequenz] eine Formel G ist eine logische Konsequenz einer Formel F ($F \models G$), wenn jedes Modell von F auch ein Modell für G ist.
                    \item[Logische Äquivalenz] zwei Formeln F und G sind semantisch äquivalent ($F \equiv G$), wenn sie genau dieselben Modelle haben \vl{FS 22}
                \end{description}

            \item[Normalform] jede Formel lässt sich in eine äquivalente Formel in Normalform umformen. \\ Für die Umformungen gibt es Algorithmen, siehe \vl{FS 22}
                \begin{description}
                    \item[Negationsnormalform (NNF)] enthält nur UND, ODER und Negation, wobei Negation nur direkt vor Atomen vorkommt.
                    \item[Konjunktive Normalform (KNF)] Formel ist eine Konjunktion von Disjunktionen von Literalen.
                    \item[Disjunktive Normalform (DNF)] Disjunktion von Konjunktionen von Literalen.
                \end{description}

            % Logisches Schließen: Resolution
            % Horn-Klauseln

        \end{description}


    \newpage
    \subsection{Komplexität}
    \label{subsec:fs-komplexitaet}
    Turingmaschinen sind begrenzt durch die Anzahl ihrer Speicherzellen (Speicher) und der Anzahl möglicher Berechnungsschritte (Zeit).
    Schranken sind Funktionen gerichtet nach der Länge der Angabe. \vl{FS 24}

    \begin{description}
        \item[$O$-Notation] charakterisiert Funktionen nach ihrem Verhalten und versteckt lineare Faktoren.
        \item[$O(f)$-zeitbeschränkt] es gibt eine Funktion, so dass eine DTM bei beliebiger Eingabe nach einer maximalen Anzahl Schritte anhält
        \item[$O(f)$-speicherbeschränkt] es gibt eine Funktion, os dass eine DTM bei beliebiger Eingabe nur eine maximale Anzahl Speicherzellen verwendet

        \item[Sprachklassen] Einteilung von Sprachen nach der Möglichkeit der Entscheidbarkeit. \\
            „Klasse aller Sprachen, welche…“
            \begin{description}
                \item[DTIME$f(n)$] …durch eine $O(f)$-zeitbeschränkte DTM entschieden werden können
                \item[DSPACE$f(n)$] …durch eine $O(f)$-speicherbeschränkte DTM entschieden werden können

                \item[NTIME$f(n)$] …durch eine $O(f)$-zeitbeschränkte NTM entschieden werden können
                \item[NSPACE$f(n)$] …durch eine $O(f)$-speicherbeschränkte NTM entschieden werden können
            \end{description}

        \item[Komplexitätsklassen] erfassen Sprachklassen je nach ihrer Komplexität.
            Stehen untereinander in Beziehung und bilden quasi Hierarchie. \vl{FS 24}
            \begin{description}
                \item[PTime (P)] deterministisch, polynomielle Zeit
                \item[ExpTime (Exp)] deterministisch, exponentielle Zeit
                \item[LogSpace (L)] deterministisch, logarithmischer Speicher
                \item[PSpace] deterministisch, polynomieller Speicher

                \item[NPTime (NP)] nichtdeterministisch, polynomielle Zeit
                \item[NEexpTime (NExp)] nichtdeterministisch, exponentielle Zeit
                \item[NLogSpace (NL)] nichtdeterministisch, logarithmischer Speicher
                \item[NPSpace] nichtdeterministisch, polynomieller Speicher
            \end{description}

        \item[SAT] Boolean Satisfiability Problem. Problem, welches nach einem Modell für eine Formel sucht. In $NP$.
            Interessant für Untersuchung, da SAT ein Problem darstellt, für welches es schwierig ist eine Lösung zu finden,
            jedoch sehr einfach ist eine Lösung auf Korrektheit zu prüfen. \vl{FS 25}

        \item[Reduktion] Rückführung eines Problems auf ein anderes. Beispiel Drei-Farben-Problem ist auf SAT reduzierbar, da sich die Farb-Zustände als
            Formeln ausdrücken kodieren lassen. „Alle Probleme in NP können polynomiell auf SAT reduziert werden“ (\f{Cook, Levin})

        \item[Härte und Vollständigkeit] für P und NP \vl{FS 25}
            \begin{description}
                \item[NP-hart] Sprache ist NP-hart, wenn jede Sprache in NP polynomiell darauf reduzierbar ist \\
                    (Beispiel Halteproblem und jedes weitere unentscheidbare Problem).
                \item[NP-vollständig] Sprache ist NP-hart und liegt selbst in NP (Beispiel SAT).

                \item[P-hart] Sprache ist P-hart, wenn jede Sprache in P mit logarithmischem Speicherbedarf auf diese reduzierbar ist.
                \item[P-vollständig] Sprache ist P-hart und liegt selbst in P (Beispiel HornSAT).
            \end{description}
    \end{description}

    \textit{Zusammenfassung aller Themenkomplexe, Hierarchien und Zusammenhänge in \vl{FS 26}.}


    \newpage
    \section{Theoretische Informatik}
    \subsection{Turingmaschinen}
    \begin{description}
        \item[Turingmaschine] deterministisch als DTM oder nichtdeterministisch als NTM. \\
            Definiert als Tupel $(Q,\Sigma,\Gamma,\delta,q_0,F)$ mit endlicher Menge von Zuständen $Q$, Eingabealphabet $\Sigma$, Arbeitsalphabet $\Gamma$, Übergangsfunktion $\delta$, Startzustand $q_0$ und Menge von akzeptierenden Endzuständen $F$. Können ein oder mehrere Bänder haben. Siehe auch Church-Turing-These. \vl{FS 18} \vl{TIL 1}

            \begin{description}
                \item[Funktion] Turingmaschine kann eine Funktion von Eingaben auf Ausgabewörter definieren. Wenn eine TM bei Eingabe $w$ anhält und die Ausgabe der Form $v\blank\blank$ entspricht, hat diese TM die Funktion berechnet.
                \item[Sprache] die von einer Turingmaschine erkannte Sprache ist die Menge aller Wörter, die von dieser TM akzeptiert werden (d.h. in einem Endzustand hält).

                \item[Konfiguration]
                    der „Gesamtzustand“ einer TM, bestehend aus Zustand, Bandinhalt und Position des Lese-/Schreibkopfs;
                    geschrieben als Wort (Bandinhalt), in dem der Zustand vor der Position des Kopfes eingefügt ist. Beispiel $ \blank\blank q_0aaba \blank\blank$.
                \item[Übergangsrelation]
                    Beziehung zwischen zwei Konfigurationen wenn die TM von der ersten in die zweite übergehen kann
                    (deterministisch oder nichtdeterministisch)
                \item[Lauf] mögliche Abfolge von Konfigurationen einer TM, beginnend mit der Startkonfiguration; kann endlich oder unendlich sein
                \item[Halten] Ende der Abarbeitung, wenn die TM in einer Konfiguration keinen Übergang mehr zur Verfügung hat.

                \item[Transducer] Ausgabe der Turingmaschine ist Inhalt des Bandes, wenn TM hält, ansonsten undefiniert. Endzustände sind irrelevant.
                \item[Entscheider] Ausgabe der Turingmaschine ist „Akzeptiert“, wenn TM in Endzustand hält, ansonsten „verwirft“ (beinhaltet auch „TM hält nicht“). Bandinhalt ist irrelevant.
                \item[Aufzähler] ist eine DTM, die bei Eingabe des leeren Bandes immer wieder einen Zustand $q_{Ausgabe}$ erreicht, in welchem das aktuelle Band ein Wort aus der Sprache dieser DTM ist. Die Sprache dieser DTM ist dann die Menge der so erzeugten Wörter. Diese DTM muss nicht halten, die Sprache kann unendlich sein. Wörter dürfen mehrfach ausgegeben werden.
            \end{description}

        \item[Berechenbarkeit] bezogen auf Funktionen. Eine Funktion $F$ heißt berechenbar, wenn es eine DTM gibt, die $F$ berechnet. Ist durch geeignete Kodierung (z.B. binär) erweiterbar auf natürliche Zahlen, Wörterlisten und andere Mengen. \vl{TIL 2}
            \begin{description}
                \item[rekursiv] eine berechenbare totale Funktion ist rekursiv.
                \item[partiell rekursiv] eine berechenbare partielle Funktion ist partiell rekursiv.
            \end{description}

        \item[Entscheidbarkeit] bezogen auf Sprachen. \vl{TIL 2}
            \begin{description}
                \item[entscheidbar / berechenbar / rekursiv] es existiert eine Turingmaschine, die das Wortproblem der Sprache entscheidet. D.h. die Turingmaschine ist Entscheider und die Sprache ist gleich der Sprache der TM.
                \item[semi-entscheidbar / Turing-erkennbar / Turing-akzeptierbar / rekursiv aufzählbar] es existiert eine Turingmaschine, deren erzeugte Sprache gleich der Sprache ist, jedoch die TM kein Entscheider ist. \\
                Eine Sprache ist semi-entscheidbar, wenn es einen Aufzähler für diese Sprache gibt.
                \item[unentscheidbar] sonst. \\
                    „Es gibt Sprachen und Funktionen, die nicht berechenbar sind.“ Beweis anhand der abzählbaren Menge von Turingmaschinen im Vergleich zur Überabzählbarkeit der Menge der Sprachen über jedem Alphabet.
            \end{description}

        \item[Probleme] der Kategorie „Unentscheidbar“. \vl{TIL 2}
            \begin{description}
                \item[Busy-Beaver-Funktion] ist nicht berechenbar und wächst sehr schnell. Die Funktion nimmt eine natürliche Zahl $n$ und gibt die maximale Anzahl $x$-Symbole, welche eine DTM mit $n$ Zuständen und dem Arbeitsalphabet $\{x,\blank\}$ bis zu ihrem Halt schreiben kann, zurück.
            \end{description}
    \end{description}


    \subsection{LOOP und WHILE}
    LOOP und WHILE sind eine Erfindung von Schöning und sind quasi eine pädagogische Brücke zwischen den Ultra-low-level Turingmaschinen und High-level Programmiersprachen. WHILE baut auf LOOP auf. \vl{TIL 3}
    \begin{description}
        \item[LOOP] Besteht aus Variablen, Wertzuweisungen und Schleifen. Die Eingabe einer Menge von natürlichen Zahlen wird in $x_1, x_2, …$ gespeichert. Die Ausgabe ist eine natürliche Zahl, gespeichert in $x_0$. Alle weiteren Variablen haben den Wert $0$. LOOP terminiert immer in endlich vielen Schritten. Berechnet eine totale Funktion.
            \begin{description}
                \item[Variablen] Menge $\{x_0,x_1,…\}$ oder auch $\{x, y, myVariable\}$. Haben als Wert eine natürliche Zahl.
                \item[Wertzuweisungen] in der Form $x := y + n$ oder $x := y - n$, wobei $n$ eine natürliche Zahl ist. Eine Wertzuweisung ist bereits ein LOOP-Programm.
                \item[Schleifen] in der Form LOOP $x$ DO $P$ END, wobei $P$ wieder ein LOOP-Programm ist. Der Wert der Variable $x$ kann in $P$ nicht geändert werden. Daher terminiert ein LOOP-Programm immer in endlich vielen Schritten.
                \item[Hintereinanderausführung] wenn $P_0$ und $P_1$ LOOP-Programme, dann auch $P_0;P_1$.
                \item[Syntax-Erweiterung] Die Syntax lässt sich zur Vereinfachung erweitern.
                    \begin{description}
                        \item[Wertzuweisung \hili{$x:=y$}] $x:=y+0$
                        \item[Rücksetzen \hili{$x:=0$}] LOOP $x$ DO $x:=x-1$ END
                        \item[Wertzuweisung Zahl \hili{$x:=n$}] $x:=0;x:=x+n$. Alternativ $x:=null+n$
                        \item[Variablen-Addition \hili{$x:=y+z$}] $x:=y;$ LOOP $z$ DO $x:=x+1$ END
                        \item[Bedingung \hili{IF $x\neq0$ THEN}] LOOP $x$ DO $y:=1$ END $;$ LOOP $y$ DO $P$ END
                    \end{description}
                \item[Berechenbarkeit] eine Funktion heißt LOOP-berechenbar, wenn es ein LOOP-Programm gibt, welches die Funktion berechnet. Auch hier ist mit geeigneter Kodierung wieder mehr machbar, als nur die natürlichen Zahlen in Betracht zu ziehen (Beispiel Wortproblem, Probleme in NP, gängige Algorithmen). Es gibt berechenbare totale Funktionen, die nicht LOOP-berechenbar sind (vgl. Ackermannfunktion).
            \end{description}

        \item[WHILE] Basiert auf LOOP und erweitert dieses. Jedes LOOP-Programm ist auch ein WHILE-Programm.
            \begin{description}
                \item[Schleifen] in der Form WHILE $x \neq 0$ DO $P$ WHEN, wobei $P$ wieder WHILE-Programm. Im Gegensatz zu LOOP kann in WHILE der Wert von $x$ in $P$ zur Laufzeit geändert werden. Es kann also passieren, dass das Programm nicht terminiert wenn $x$ nie auf $0$ gesetzt wird.
                \item[Konvertierung] LOOP-Schleifen können in WHILE-Schleifen konvertiert werden. Eine DTM kann WHILE-Programme simulieren und ein WHILE-Programm DTMen simulieren.
                \item[Berechenbarkeit] Eine partielle Funktion heißt WHILE-berechenbar, wenn es ein WHILE-Programm gibt, welches bei einem definierten $f(n_0,n_1,…)$ terminiert und bei einem nicht definierten Wertebereich nicht terminiert. Wenn eine partielle Funktion WHILE-berechenbar ist, ist sie \f{Turing-berechenbar}.
            \end{description}

    \end{description}

    \subsection{Universalität}
    \begin{description}
        \item[Universalmaschine $U$] eine Turingmaschine, die andere TM als Eingabe kodiert erhält und diese simuliert. Die Kodierung ist dabei z.B. binär, mit dem Trennsymbol $\#$. Hat vier Bänder: Eingabeband von $U$ mit kodierter TM und kodierter Eingabe $w$, Arbeitsband von $U$, Band 3 mit aktuellem Zustand der simulierten TM und Band 4 als Arbeitsband der simulierten TM. \\ Für die Arbeitsweise siehe \vl{TIL 4}
    \end{description}

    \newpage
    \subsection{Unentscheidbare Probleme und Reduktionen}
    Beweis durch Diagonalisierung, Reduktionen \vl{TIL 4}
    \begin{description}
        \item[Probleme] der Kategorie „unentscheidbar“.
        \begin{description}
            \item[Halteproblem \phalt] Frage: „Gegeben eine Turingmaschine $M$ und ein Wort $w$. Wird die Turingmaschine $M$ für die Eingabe $w$ jemals anhalten?“. Das Halteproblem \phalt der Turingmaschine $M$ für das Wort $w$ kann formal kodiert werden als $enc(M)\#\#enc(w)$ und einer universellen Turingmaschine zur Überprüfung übergeben werden. Beweise für Unentscheidbarkeit anhand Diagonalisierung und Reduktion in \vl{TIL 4}

            \item[Goldbachsche Vermutung] Beispiel für ein unentscheidbares Halteproblem. Besagt, dass jede gerade Zahl $n \ge 4$ die Summe zweier Primzahlen ist. Zum Beispiel ist $4 = 2 + 2$ und $100 = 47 + 53$. Lässt man nun eine Turingmaschine diese Vermutung systematisch beginnend bei $4$ testen, würde ein Anhalten bei Misserfolg \phalt und „die Vermutung stimmt nicht“ gleichzeitig lösen. Gäbe es demnach ein Programm, welches \phalt lösen kann (entscheidet), wäre eine Überprüfung der Goldbachschen Vermutung nicht nötig. Die Frage von \phalt wäre sofort beantwortet.

            \item[$\epsilon$~-Halteproblem] „Gegeben sei eine Turingmaschine. Wird diese TM für die leere Eingabe $\epsilon$ jemals anhalten?“. Unentscheidbar.
        \end{description}

        \item[Beweismethoden] zum Nachweis der Unentscheidbarkeit.
        \begin{description}
            \item[Kardinalität] Beweis von Aussagen anhand der unterschiedlichen Kardinalitäten.
            \item[Diagonalisierung] Berechenbarkeit annehmen und einen paradoxen Algorithmus für das Problem konstruieren.
            \item[Reduktion] Rückführung eines Problems auf ein anderes. Die Reduktion ist ein Entscheidbarkeitsalgorithmus. Siehe auch \verweis{subsec:fs-komplexitaet}. \vl{TIL 4}
                \begin{description}
                    \item[Turing-Reduktion] Ein Problem \prob{P} ist Turing-reduzierbar auf ein Problem \prob{Q} (in Symbolen: $\prob{P} \leq_T\prob{Q}$), wenn man \prob{P} mit einem Programm lösen kann, welches ein Programm für \prob{Q} als Unterprogramm (auch: Subroutine) aufrufen darf.
                    \item[Many-One-Reduktion] Eine berechenbare totale Funktion $f: \Sigma^* \to \Sigma^*$ ist eine Many-One-Reduktion von einer Sprache $P$ auf eine Sprache $Q$ (in Symbolen: $P \leq_m Q$), wenn für alle Wörter $w \in \Sigma^*$ gilt: $w \in P$ gdw. $f(w) \in Q$. \\
                    Schwächer als Turing-Reduktion, jede Many-One-Reduktion kann als Turing-Reduktion ausgedrückt werden.
                \end{description}
            \item[Satz von Rice] Siehe \verweis{subsec:fs-sprachen-automaten}. \vl{TIL 5} \\
                „Praktisch alle interessanten Fragen zu Turingmaschinen sind unentscheibar“. \\
                Eingabe ist eine Turingmaschine, Ausgabe „hat die Sprache der TM die Eigenschaft?“.
        \end{description}

    \end{description}


    \subsection{Semi-Entscheidbarkeit}
    \textit{Hinweis: Hierzu gibt es im Schöning gute graphische Darstellungen}. \vl{TIL 5}
    \begin{description}
        \item[Komplement] einer Sprache $L$: $\overline{L} = \{w \in \Sigma^* ~|~ w \notin L \}$. Die Turing-Reduktionen $\overline{L} \leq_T L$ bzw. $L \leq_T \overline{L}$ sind mit einer Turingmaschine überprüfbar. Für eine Eingabe $w$ entscheidet diese, ob $w \in L$ und invertiert das Ergebnis.
        \item[Semi-Entscheidbarkeit] Beispiel anhand des Halteproblems: simuliere eine Turingmaschine und deren Eingabe, kodiert als $enc(M)\#\#enc(w)$. Wenn $M$ hält, hält auch die universelle Turingmaschine und akzeptiert. Eine Sprache $L$ ist entscheidbar, wenn sowohl $L$ als auch $\overline{L}$ semi-entscheidbar sind.
        \item[Co-Semi-Entscheidbarkeit] Wenn eine Sprache $L$ unentscheidbar, jedoch semi-entscheidbar ist, kann $\overline{L}$ nicht semi-entscheidbar sein.
    \end{description}


    \newpage
    \subsection{Postsches Korrespondenzproblem}
    Auch: \f{PCP}. Ein unentscheidbares Problem ohne direkten Bezug zu einer Berechnung. \vl{TIL 5}
    \begin{description}
        \item[PCP] Bei diesem Problem nimmt man eine Reihe von 2-Tupeln (anschaulich vergleichbar mit Dominosteinen) mit je einem Wert oben und einem unten. Ziel der Lösung ist nun, die gegebenen Tupel so anzuordnen, dass oben und unten die gleiche Wortkette entsteht. Beispiel: wir haben die drei Tupel (AB, B), (B, BBB) und (BB, BA). Eine Anordnung mit zehn Tupeln ergibt dann die Lösung. Es kann vorkommen, dass das Problem keine Lösung besitzt.
        \item[MPCP] Hilfskonstruktion. Wir nutzen MPCP, um das Halteproblem auf MPCP zu reduzieren. Folgend reduzieren wir MPCP auf PCP. Beim MPCP wird PCP verwendet, jedoch das Start-Tupel vorgegeben. Die Lösung eines MPCP ist auch eine Lösung des entsprechenden PCP, welche mit dem gegebenen Start-Tupel beginnt.
    \end{description}


    \subsection{Unentscheidbare Probleme formaler Sprachen}
    In diesem Kapitel wird wieder auf \verweis{subsec:fs-sprachen-automaten} zurückgegriffen. Eine durch eine Grammatik $G$ erzeugte Sprache wird als $L(G)$ bezeichnet. Für Beweise der folgenden Sätze siehe Vorlesung. Siehe auch Chomsky-Hierarchie in \verweis{subsec:fs-sprachen-automaten}. \vl{TIL 6}
    \begin{itemize}
        \setlength\itemsep{0em}
        \item Das Schnittproblem regulärer Grammatiken (Typ 3) ist entscheidbar.
        \item Das Schnittproblem kontextfreier Grammatiken (Typ 2, \f{CFG}) ist unentscheidbar. Beweis durch Many-One-Reduktion vom PCP.
        \item Das Leerheitsproblem für kontextfreie Grammatiken ist entscheidbar.
        \item Kontextfreie Sprachen sind unter Vereinigung abgeschlossen.
        \item Deterministische kontextfreie Sprachen sind unter Komplement abgeschlossen.
        \item Das Äquivalenzproblem kontextfreier Grammatiken ist unentscheidbar.
    \end{itemize}


    \newpage
    \subsection{Komplexitätstheorie}
    Untersuchung von Problemkomplexitäten und Suche nach Methoden zur Bestimmung der Komplexität eines Problems. Klassierung zwischen „leicht lösbar“ bis „schwer lösbar“. Einteilung von berechenbaren Problemen entsprechend der Menge an Ressourcen, die zu ihrer Lösung nötig sind. Einführung anhand von Beispielen. \vl{TIL 7}
    \begin{description}
        \item[Eulerpfad] Ein Eulerpfad ist ein Pfad in einem Graphen, der jede Kante genau einmal durchquert. Ein Eulerkreis ist ein zyklischer Eulerpfad. Ein Graph hat genau dann einen Eulerschen Pfad, wenn er maximal zwei Knoten ungeraden Grades besitzt.
        \item[Hamiltonpfad] Ein Hamiltonpfad ist ein Pfad in einem Graphen, der jeden Knoten genau einmal durchquert. Ein Hamiltonkreis ist ein zyklischer Hamiltonpfad.

        \item[Schranken von Turingmaschinen] in Zeit und Raum. Siehe \verweis{subsec:fs-komplexitaet}.
            \begin{description}
                \item[Speicher] Zahl der verwendeten Speicherzellen
                \item[Zeit] Zahl der durchgeführten Berechnungsschritte
            \end{description}

        \item[$O$-Notation] gibt Ressourcenschranken für Turingmaschinen an, versteckt konstante Faktoren.
        \item[Linear Speedup Theorem] Sei $M$ eine Turingmaschine mit $k > 1$ Bändern, die bei Eingaben der Länge $n$ nach maximal $f(n)$ Schritten hält. Dann gibt es für jede natürliche Zahl $c > 0$ eine äquivalente $k$-Band Turingmaschine $M'$, die nach maximal $\frac{f(n)}{c} + n + 2$ Schritten hält. \\
        Bedeutet: in der Theorie kann jedes Programm mit Hilfe mehrerer Bänder beliebig schnell gemacht werden. Dies ist praktisch nicht umsetzbar, da eine Turingmaschine nicht beliebig große Datenmengen in einem Schritt lesen und nicht beliebig komplexe Zustandsübergänge in konstanter Zeit realisieren kann.
    \end{description}

    \subsection{Beziehungen der Komplexitätsklassen}
    Siehe \verweis{subsec:fs-komplexitaet} für eine Übersicht der Klassen. \vl{TIL 7}
    \begin{description}
        \item[Nichtdeterministische Klassen] $NL \subseteq NP \subseteq NPSpace \subseteq NExp$
        \item[DTM auch als NTM, d.h. nichtdet. stärker] $L \subseteq NL$, $P \subseteq NP$, $PSpace \subseteq NPSpace$, $Exp \subseteq NExp$
        \item[Satz von Savich] Speicherbeschränkte NTM können durch DTMs nur mit  quadratischen Mehrkosten simuliert werden. Insbesondere gilt damit $PSpace = NPSpace$.
    \end{description}

    Zusammenfassend: $L \subseteq NL \subseteq P \subseteq NP \subseteq PSpace = NPSpace \subseteq Exp \subseteq NExp$. \\
    Jedoch ist zu beachten:

    \begin{itemize}
        \setlength\itemsep{0em}
        \item Wir wissen nicht, ob irgendeines dieser $\subseteq$ sogar $\subsetneq$ ist.
        \item Insbesondere wissen wir nicht, ob $P \subsetneq NP$ oder $P = NP$.
        \item Wir wissen nicht einmal, ob $L \subsetneq NP$ oder $L = NP$.
    \end{itemize}


    \newpage
    \section{Übungen}


\end{document}
