\section{Repetitorien}

\subsection*{Repetitorium I}
\subsubsection*{Aufgabe A}
    Wiederholung von Begriffen Einband Turing-Maschine, Mehrband Turing-Maschine, Entscheidungsproblem, Unentscheidbarkeit, Aufzählbarkeit, Abzählbarkeit und Halteproblem.

\subsubsection*{Aufgabe B}
    Zeigen Sie: Wenn es möglich ist, für zwei beliebige Turing-Maschinen zu entscheiden, ob sie dieselbe Sprache akzeptieren, so ist es auch möglich, für beliebige Turing-Maschinen zu entscheiden, ob sie die leere Sprache akzeptieren. Seien $K, M_1, M_2$ Turingmaschinen, so dass $K(enc(M_1)\#\#enc(M_2))$ akzeptiert $\Leftrightarrow$ $L(M_1) = L(M_2)$ und $K$ hält auf jeder Eingabe. \\
    \textbf{Lösung:} Sei $M$ Turingmaschine und sei $M_\emptyset$ eine Turingmaschine, so dass $L(M_\emptyset) = \emptyset$. \\
    Dann gilt $K(enc(M)\#\#enc(M_\emptyset))$ akzeptiert $\Leftrightarrow$ $L(M) = \emptyset$, also $\prspec{leer} \leq_{m} \prspec{"aquiv}$.

\subsubsection*{Aufgabe C}
    Zeigen Sie, dass $\{1\}^*$ unentscheidbare Teilmengen besitzt. \\
    \textbf{Lösung:} $\{1\}^*$ ist abzählbar unendlich, also ist $\mathfrak{P}(\{1\}^*)$ überabzählbar. Es gibt aber nur abzählbar unendlich viele entscheidbare Sprachen (auch: abzählbar viele nicht-äquivalente Turingmaschinen). Also sind einige (fast alle) dieser Sprachen unentscheidbar.

\subsubsection*{Aufgabe D}
    \begin{enumerate}
        \item „Jedes LOOP-Programm terminiert.“ – Richtig. Definition von LOOP sagt, dass Anzahl Durchläufe nicht mehr während der Laufzeit geändert werden kann, demnach gibt es eine endliche Anzahl Durchläufe.

        \item „Zu jedem WHILE-Programm gibt es ein äquivalentes LOOP-Programm.“ – Falsch, nicht zu jedem WHILE-Programm gibt es immer ein äquivalentes LOOP-Programm. Dies liegt daran, dass LOOP keine partiellen Funktionen verarbeiten kann. \\
        Beispiel anhand von Division: LOOP terminiert immer, jedoch wäre Division durch $0$ (ebenfalls in $\mathbb{N}$) undefiniert. Kann demnach nur mit WHILE gelöst werden (Fall $x_2 = 0$ für $div(x_1, x_2)$ landet in Endlosschleife).

        \item „Die Anzahl der Ausführungen von $P$ in der LOOP-Schleife LOOP $x_i$ DO $P$ END kann beeinflusst werden, indem $x_i$ in $P$ entsprechend modifiziert wird.“ – Falsch, Anzahl Schleifen kann laut Definition von LOOP nicht während Laufzeit geändert werden.

        \item „Die Ackermannfunktion ist total und damit LOOP-berechenbar.“ – Falsch, die Ackermannfunktion ist zwar total, jedoch nicht LOOP-berechenbar (jedoch berechenbar). Die Funktion wurde gezielt gesucht und gefunden, um genau diesen Fall zu zeigen.
    \end{enumerate}

\subsubsection*{Aufgabe E}
    Geben Sie eine Turing-Maschine $A_{mod2}$ an, die die Funktion  $f: \mathbb{N} \to \mathbb{N}$ mit $f(x) = (x~mod~ 2)$ berechnet. Stellen Sie dabei die Zahlen in unärer Kodierung dar. \\
    \textbf{Lösung:} $A_{mod2} = (\{q_0,…,q_f\}, \{x\}, \{x, \blank\}, \delta, q_0, \{q_f\})$. Die Turingmaschine liest die eingegebenen $x$ (unäre Kodierung) und wechselt zwischen $q_0$ und $q_1$. Sobald auf ein $\blank$ gestoßen wird, weiß die TM, ob eine gerade oder ungerade Anzahl $x$ eingegeben wurde. Wenn gerade, lösche alle $x$ und ende in leerem Band. Wenn ungerade, lösche alle $x$ und schreibe zum Abschluss ein $x$ auf das Band.

\subsubsection*{Aufgabe F}
    Es sei $f:\N\to\N$ mit $f(x)=\lfloor\log_{10}(x)\rfloor$. Geben Sie ein WHILE-Programm an, welches $f$ berechnet. \\
    \textbf{Lösung:} Erst eine endlose WHILE-Schleife für die Eingabe $x=0$. Dann Lösung mit $div(x, 10)$.

\subsubsection*{Aufgabe G}
    \begin{enumerate}
        \item „Die Menge der Instanzen des Postschen Korrespondenzproblems, welche eine Lösung haben, ist semi-entscheidbar.“ – Richtig. Wenn eine Lösung existiert, kann diese (z.B. durch Breitensuche) auch gefunden werden.

        \item „Das Postsche Korrespondenzproblem ist bereits über dem Alphabet $\Sigma = \{a,b\}$ nicht entscheidbar.“ – Richtig, denn Instanzen können über $\Sigma$ kodiert werden, ohne die Entscheidbarkeit zu beeinflussen.

        \item „Es ist entscheidbar, ob eine Turingmaschine nur Wörter akzeptiert, die Palindrome sind.“ – Falsch, Satz von Rice (die Akzeptanz von Palindromen ist eine \f{Eigenschaft}). Eigenschaft $E$ ist „$L$ besteht nur aus Palindromen“, diese Eigenschaft ist nicht-trivial: erfüllt z.B. durch $L = \emptyset$, jedoch nicht durch $L = \{ab\}$.

        \item „$\prspec{halt}$ ist semi-entscheidbar“ – Richtig, da es universelle Turingmaschinen gibt, die beliebige TM simulieren können.

        \item „Es ist nicht entscheidbar, ob die von einer deterministischen Turing-Maschine berechnete Funktion total ist.“ – Richtig, denn sonst wäre das Halteproblem entscheidbar ($\prspec{halt} \leq_{m} \prspec{total}$). \\
        Reduktion: $M, w$ gegeben, baue Turingmaschine $M'$ mit \\
        $M' =$ bei Eingabe $x$ \\
        $\to$ simuliere $M$ auf $w$ \\
        $\to$ akzeptiere mit leerem Band \\
        M hält auf $w \Rightarrow M'$ berechnet $f(x)= \epsilon$ \\
        M hält nicht auf $w \Rightarrow M'$ berechnet Abbildung, die nirgends definiert ist. \\
        Reduktion demnach $enc(M)\#\#enc(w) \mapsto enc(M')$.

        \item „Es gibt reguläre Sprachen, die nicht semi-entscheidbar sind.“ – Falsch. Reguläre Sprachen sind immer entscheidbar, da Turingmaschinen endliche Automaten simulieren können.
    \end{enumerate}

\subsubsection*{Aufgabe H}
    Sei $L$ eine unentscheidbare Sprache. Zeigen Sie:

    \begin{enumerate}
        \item „$L$ hat eine Teilmenge $T \subseteq L$, die entscheidbar ist.“: $T = \emptyset$.
        \item „$L$ hat eine Obermenge $O \supseteq L$, die entscheidbar ist.“: $O = \Sigma^*$.
        \item „Es gibt jeweils nicht nur eine sondern unendlich viele entscheidbare Teilmengen bzw. Obermengen wie in (a) und (b).“ – Es gilt: $L$ ist unendlich. Dann ist die Menge der endlichen Teilmengen von $L$ unendlich. Alles diese sind entscheidbar. Genauso für \textbf{b)}, z.B. muss $\Sigma^*\setminus L$ unendlich sein. Die Menge der endlichen Teilmengen $E$ von $\Sigma^*\setminus L$ ist unendlich, für jede ist $\Sigma^*\setminus E$ entscheidbar.
    \end{enumerate}
